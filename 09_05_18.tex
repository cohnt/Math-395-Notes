\documentclass[10pt,letterpaper]{article}
\usepackage[utf8]{inputenc}
\usepackage{amsmath}
\usepackage{amsfonts}
\usepackage{amssymb}
\usepackage{ragged2e}
\usepackage[letterpaper, margin=1in]{geometry}
\usepackage{graphicx}
\usepackage{cancel}
\usepackage{mathtools}
\usepackage{tabularx}
\usepackage{arydshln}
\usepackage{tensor}
\usepackage{array}
\usepackage{xcolor}

%%%%%%%%%%%%%%%%%%%%%%%%%%%%%
% Formatting commands
%%%%%%%%%%%%%%%%%%%%%%%%%%%%%
\newcommand{\n}{\hfill\break}
\newcommand{\lemma}[1]{\par\noindent\settowidth{\hangindent}{\textbf{Lemma: }}\textbf{Lemma: }#1\n}
\newcommand{\defn}[1]{\par\noindent\settowidth{\hangindent}{\textbf{Defn: }}\textbf{Defn: }#1\n}
\newcommand{\thm}[1]{\par\noindent\settowidth{\hangindent}{\textbf{Thm: }}\textbf{Thm: }#1\n}
\newcommand{\ex}[1]{\par\noindent\settowidth{\hangindent}{\textbf{Ex: }}\textbf{Ex: }#1\n}
\newcommand{\proven}{\;$\square$\n}
\newcommand{\problem}[1]{\par\noindent{#1}\n}
\newcommand{\problempart}[2]{\par\settowidth{\hangindent}{\textbf{(#1)} \indent{}}\textbf{(#1)} #2\n}
\newcommand{\ptxt}[1]{\textrm{\textnormal{#1}}}
\newcommand{\inlineeq}[1]{\n\centerline{$\displaystyle #1$}}
\newcommand{\pageline}{\noindent\rule{\textwidth}{0.1pt}\n}

%%%%%%%%%%%%%%%%%%%%%%%%%%%%%
% Math commands
%%%%%%%%%%%%%%%%%%%%%%%%%%%%%
% Set Theory
\newcommand{\card}[1]{\left|#1\right|}
\newcommand{\set}[1]{\left\{#1\right\}}
\newcommand{\ps}[1]{\mathcal{P}(#1)}
\newcommand{\naturals}{\mathbb{N}}
\newcommand{\N}{\naturals}
\newcommand{\integers}{\mathbb{Z}}
\newcommand{\Z}{\integers}
\newcommand{\rationals}{\mathbb{Q}}
\newcommand{\Q}{\rationals}
\newcommand{\reals}{\mathbb{R}}
\newcommand{\R}{\reals}
\newcommand{\complex}{\mathbb{C}}
\newcommand{\C}{\complex}
\newcommand{\comp}{^{\complement}}

% Graph Theory
\renewcommand{\deg}[1]{\ptxt{deg}\left(#1\right)}

% Standard Math
\newcommand{\inv}{^{-1}}
\newcommand{\abs}[1]{\left|#1\right|}
\newcommand{\ceil}[1]{\left\lceil{}#1\right\rceil}
\newcommand{\floor}[1]{\left\lfloor{}#1\right\rfloor{}}
\newcommand{\conj}[1]{\overline{#1}}
\newcommand{\of}{\circ}
\newcommand{\tri}{\triangle}
\newcommand{\inj}{\hookrightarrow}
\newcommand{\surj}{\twoheadrightarrow}
\newcommand{\mapsfrom}{\mathrel{\reflectbox{\ensuremath{\mapsto}}}}

% Linear Algebra
\newcommand{\Id}{\textrm{\textnormal{Id}}}
\newcommand{\im}{\textrm{\textnormal{im}}}
\newcommand{\norm}[1]{\abs{\abs{#1}}}
\newcommand{\tpose}{^{T}}
\newcommand{\iprod}[1]{\left<#1\right>}
\newcommand{\trace}{\ptxt{tr}}
\newcommand{\chgBasMat}[3]{\!\!\tensor*[_{#1}]{\left[#2\right]}{_{#3}}}
\newcommand{\vecBas}[2]{\tensor*[]{\left[#1\right]}{_{#2}}}

% Topology
\newcommand{\closure}[1]{\bar{#1}}

% Proofs
\newcommand{\st}{s.t.}
\newcommand{\unique}{!}

%%%%%%%%%%%%%%%%%%%%%%%%%%%%%
% Other commands
%%%%%%%%%%%%%%%%%%%%%%%%%%%%%
\newcommand{\flag}[1]{\textbf{\textcolor{red}{#1}}}

%%%%%%%%%%%%%%%%%%%%%%%%%%%%%
% Make l's curvy in math environments
%%%%%%%%%%%%%%%%%%%%%%%%%%%%%
\mathcode`l="8000
\begingroup
\makeatletter
\lccode`\~=`\l
\DeclareMathSymbol{\lsb@l}{\mathalpha}{letters}{`l}
\lowercase{\gdef~{\ifnum\the\mathgroup=\m@ne \ell \else \lsb@l \fi}}%
\endgroup

\author{Professor David Barrett\\ \small\textit{Transcribed by Thomas Cohn}}
\title{A Little Linear Algebra}
\date{9/5/2018} % Can also use \today

\begin{document}
\maketitle
\setlength\RaggedRightParindent{\parindent}
\RaggedRight

\par\noindent Let $\vec{p},\vec{q},\vec{x}$ be vectors in the vector space $V$, with $\vec{p}\ne\vec{q}$.\n

\par\noindent $\vec{p},\vec{q},\vec{x}$ are collinear $\leftrightarrow$ $\vec{x}-\vec{p}=t(\vec{q}-\vec{p})$ for some scalar $t$.\n
\phantom{$\vec{p},\vec{q},\vec{x}$ are collinear }$\leftrightarrow$ $\vec{x}=(1-t)\vec{p}+t\vec{q}$ for some scalar $t$.\n

\par\noindent So we can say that the line through $\vec{p},\vec{q}$ is the set of points $(1-t)\vec{p}+t\vec{q}$.\n

\defn{$S\subset{}V$ is \underline{affine} if and only if $S$ contains all lines joining any two of its points if and only if $\vec{x},\vec{y}\in{}S$, scalar $t$ implies that $(1-t)\vec{x}+t\vec{y}\in{}S$.}

\ex{$S_{1}=\set{(t+1,t,2t):t\ptxt{ scalar}}$\n
$S_{1}$ is the line through $(1,0,0)$ and $(2,1,2)$.\n
$S_{1}$ is an affine set.}

\ex{$S_{2}=\set{(x,y,3):x,y\ptxt{ scalar}}$\n
$S_{2}$ is the plane through $(0,0,3)$ parallel to the $x$-$y$ plane.\n
$S_{2}$ is an affine set.}

\thm{If $\vec{0}\in{}S$ (and $1+1\ne{}0$, that is, our vector space is not on the field of characteristic $2$), then $S$ is affine if and only if $S$ is a vector subspace of $V$.\n
Proof: Assume $S$ is a linear subspace, $\vec{x},\vec{y}\in{}S$, $t$ scalar. Then $(1-t)\vec{x}+t\vec{y}\in{}S$, so $S$ is affine.\n
\phantom{Proof: }Assume $\vec{0}\in{}S$, $\vec{y}\in{}S$, $t$ scalar. Then $t\vec{y}+(1-t)\vec{0}=t\vec{y}\in{}S$, so $S$ is closed under scalar\n
\phantom{Proof: }multiplication. Let $\vec{x},\vec{y}\in{}S$. Then $2\vec{x},2\vec{y}\in{}S$. So $(1-t)(2\vec{x})+t(2\vec{y})\in{}S$. Let $t=\frac{1}{2}$, so\n
\phantom{Proof: }$\vec{x}+\vec{y}\in{}S$. So $S$ is closed under addition, and is therefore a vector space.\proven}

\par\noindent\textbf{Special Case:} $F$ is a field of characteristic $2$. For example, $F=\set{0,1}={}^{\Z}/{}_{2\Z}$. Then the line through $\vec{p},\vec{q}$ is just $\set{\vec{p},\vec{q}}$. So all $2$-point sets are lines. And therefore, all subsets of $V$ are affine -- the set of affine sets is just $\ps{V}$.\n

\par\noindent Henceforth, for convenience (and our collective sanity), we will assume that $1+1\ne{}0$. We're almost always working with $F=\R$ in 395. In 396, we'll deal a bit with $F=\C$.\n

\defn{$S-\vec{x}=\set{\vec{y}-\vec{x}\middle|\vec{y}\in{}S}$}

\par\noindent\textbf{Important Note!} For $A,B\subset{}V$, we say $A\setminus{}B=\set{\vec{a}\in{}A\middle|\vec{a}\not\in{}B}$, $A-B=\set{\vec{a}-\vec{b}\middle|\vec{a}\in{}A,\vec{b}\in{}B}$\n

\ex{If $S\subset{}V$, $\vec{x}\in{}V$, then $S$ is affine if and only if $S-\vec{x}$ is affine.}

\par\noindent Hence, if $\vec{x}\in{}S$, then $S$ is affine iff $S-\vec{x}$ is affine iff $S-\vec{x}$ is a linear subspace.\n

\par\noindent With $S\subset{}V$, let $\widetilde{S}=\set{\vec{a}-\vec{b}\middle|\vec{a},\vec{b}\in{}S}=S-S$.\n

\thm{If $S$ is affine, $\vec{x}\in{}S$, then $S-\vec{x}=\widetilde{S}$.\n
Proof: $\vec{y}\in{}S-\vec{x}\to{}\vec{y}=\vec{a}-\vec{x}$ for some $\vec{a}\in{}S$, so $\vec{y}\in{}S$, so $S-\vec{x}\subset\widetilde{S}$.\n
$\vec{y}\in\widetilde{S}\to\vec{y}=\vec{a}-\vec{b}$ for some $\vec{a},\vec{b}\in{}S$, so $\vec{y}=\vec{a}-\vec{b}+\vec{x}-\vec{x}=(\vec{a}-\vec{x})-(\vec{b}-\vec{x})$. $\vec{a}-\vec{x},\vec{b}-\vec{x}\in{}S-\vec{x}$, so $\vec{y}\in{}S-\vec{x}$, so $\widetilde{S}\subset{}S-\vec{x}$.\proven}

\par\noindent Corrolary: If $S$ is affine, $\vec{x_{1}},\vec{x_{2}}\in{}S$, then $S-\vec{x_{1}}=S-\vec{x_{2}}$. We say that $\widetilde{S}$ is the unique linear subspace associated to $S$.\n

\ex{$S_{1}=\set{(t+1,t,2t):t\in{}F}$\n
$\widetilde{S_{1}}=\set{(t,t,2t):t\in{}F}$\n\n
$S_{2}=\set{(x,y,3):x,y\in{}F}$\n
$\widetilde{S_{2}}=\set{(x,y,0):x,y\in{}F}$}

\defn{If $S$ is affine, the \underline{dimension} of $S$, $\dim(S)=\dim(\widetilde{S})$.}

\par\noindent Note: If $\vec{a_{1}},\ldots,\vec{a_{k}}$ basis for $\widetilde{S}$, $\widetilde{S}=\set{t_{1}\vec{a_{1}}+\cdots+t_{k}\vec{a_{k}}|t_{1},\ldots,t_{k}\in{}F}$, and for some $\vec{x}\in{}S$,\n
$S=\set{\vec{x}+t_{1}\vec{a_{1}}+\cdots{}+t_{k}\vec{a_{k}}|a_{1},\ldots,a_{k}\in{}F}$.\n

\defn{$S\subset{}V$ is \underline{convex} if it contains all line segments joining any two of its points. If $\vec{x},\vec{y}\in{}S$, then $(1-t)\vec{x}+t\vec{y}\in{}S$ for all $0\le{}t\le{}1$.}

\ex{$S_{3}=\set{(x,y,3)|x^{2}+y^{2}\le{}1}$ is convex.}

\ex{If $S\subset{}V$, $\dim{}V=1$, $S$ is convex $\leftrightarrow$ $S$ is connected.}

\ex{$S$ is convex $\leftrightarrow$ the intersection with each affine line is connected.}

\end{document}