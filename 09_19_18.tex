\documentclass[10pt,letterpaper]{article}
\usepackage[utf8]{inputenc}
\usepackage{amsmath}
\usepackage{amsfonts}
\usepackage{amssymb}
\usepackage{ragged2e}
\usepackage[letterpaper, margin=1in]{geometry}
\usepackage{graphicx}
\usepackage{cancel}
\usepackage{mathtools}
\usepackage{tabularx}
\usepackage{arydshln}
\usepackage{tensor}
\usepackage{array}
\usepackage{xcolor}
\usepackage[boxed]{algorithm}
\usepackage[noend]{algpseudocode}
\usepackage{listings}

%%%%%%%%%%%%%%%%%%%%%%%%%%%%%
% Formatting commands
%%%%%%%%%%%%%%%%%%%%%%%%%%%%%
\newcommand{\n}{\hfill\break}
\newcommand{\lemma}[1]{\par\noindent\settowidth{\hangindent}{\textbf{Lemma: }}\textbf{Lemma: }#1\n}
\newcommand{\defn}[1]{\par\noindent\settowidth{\hangindent}{\textbf{Defn: }}\textbf{Defn: }#1\n}
\newcommand{\thm}[1]{\par\noindent\settowidth{\hangindent}{\textbf{Thm: }}\textbf{Thm: }#1\n}
\newcommand{\prop}[1]{\par\noindent\settowidth{\hangindent}{\textbf{Prop: }}\textbf{Prop: }#1\n}
\newcommand{\cor}[1]{\par\noindent\settowidth{\hangindent}{\textbf{Cor: }}\textbf{Cor: }#1\n}
\newcommand{\ex}[1]{\par\noindent\settowidth{\hangindent}{\textbf{Ex: }}\textbf{Ex: }#1\n}
\newcommand{\proven}{\;$\square$\n}
\newcommand{\problem}[1]{\par\noindent{#1}\n}
\newcommand{\problempart}[2]{\par\settowidth{\hangindent}{\textbf{(#1)} \indent{}}\textbf{(#1)} #2\n}
\newcommand{\ptxt}[1]{\textrm{\textnormal{#1}}}
\newcommand{\inlineeq}[1]{\n\centerline{$\displaystyle #1$}}
\newcommand{\pageline}{\noindent\rule{\textwidth}{0.1pt}}

%%%%%%%%%%%%%%%%%%%%%%%%%%%%%
% Math commands
%%%%%%%%%%%%%%%%%%%%%%%%%%%%%
% Set Theory
\newcommand{\card}[1]{\left|#1\right|}
\newcommand{\set}[1]{\left\{#1\right\}}
\newcommand{\ps}[1]{\mathcal{P}\left(#1\right)}
\newcommand{\pfinite}[1]{\mathcal{P}^{\ptxt{finite}}\left(#1\right)}
\newcommand{\naturals}{\mathbb{N}}
\newcommand{\N}{\naturals}
\newcommand{\integers}{\mathbb{Z}}
\newcommand{\Z}{\integers}
\newcommand{\rationals}{\mathbb{Q}}
\newcommand{\Q}{\rationals}
\newcommand{\reals}{\mathbb{R}}
\newcommand{\R}{\reals}
\newcommand{\complex}{\mathbb{C}}
\newcommand{\C}{\complex}
\newcommand{\comp}{^{\complement}}
\newcommand{\Hom}{\ptxt{Hom}\>}

% Graph Theory
\renewcommand{\deg}[1]{\ptxt{deg}\left(#1\right)}
\newcommand{\degp}[1]{\ptxt{deg}^{+}\!\!\left(#1\right)}
\newcommand{\degn}[1]{\ptxt{deg}^{-}\!\!\left(#1\right)}

% Standard Math
\newcommand{\inv}{^{-1}}
\newcommand{\abs}[1]{\left|#1\right|}
\newcommand{\ceil}[1]{\left\lceil{}#1\right\rceil}
\newcommand{\floor}[1]{\left\lfloor{}#1\right\rfloor{}}
\newcommand{\conj}[1]{\overline{#1}}
\newcommand{\of}{\circ}
\newcommand{\tri}{\triangle}
\newcommand{\inj}{\hookrightarrow}
\newcommand{\surj}{\twoheadrightarrow}
\newcommand{\mapsfrom}{\mathrel{\reflectbox{\ensuremath{\mapsto}}}}
\newcommand{\Graph}{\ptxt{Graph}\>}

% Linear Algebra
\newcommand{\Id}{\textrm{\textnormal{Id}}}
\newcommand{\im}{\textrm{\textnormal{im}}}
\newcommand{\norm}[1]{\abs{\abs{#1}}}
\newcommand{\tpose}{^{T}}
\newcommand{\iprod}[1]{\left<#1\right>}
\newcommand{\trace}{\ptxt{tr}}
\newcommand{\chgBasMat}[3]{\!\!\tensor*[_{#1}]{\left[#2\right]}{_{#3}}}
\newcommand{\vecBas}[2]{\tensor*[]{\left[#1\right]}{_{#2}}}
\newcommand{\GL}{\ptxt{GL}\>}

% Topology
\newcommand{\closure}[1]{\bar{#1}}
\newcommand{\uball}{\mathcal{U}}
\newcommand{\Int}{\ptxt{Int}\>}
\newcommand{\Ext}{\ptxt{Ext}\>}
\newcommand{\Bd}{\ptxt{Bd}\>}

% Proofs
\newcommand{\st}{s.t.}
\newcommand{\unique}{!}

% Algorithms
\algrenewcommand{\algorithmiccomment}[1]{\hskip 1em \texttt{// #1}}
\algrenewcommand\algorithmicrequire{\textbf{Input:}}
\algrenewcommand\algorithmicensure{\textbf{Output:}}

%%%%%%%%%%%%%%%%%%%%%%%%%%%%%
% Other commands
%%%%%%%%%%%%%%%%%%%%%%%%%%%%%
\newcommand{\flag}[1]{\textbf{\textcolor{red}{#1}}}

%%%%%%%%%%%%%%%%%%%%%%%%%%%%%
% Make l's curvy in math environments
%%%%%%%%%%%%%%%%%%%%%%%%%%%%%
\mathcode`l="8000
\begingroup
\makeatletter
\lccode`\~=`\l
\DeclareMathSymbol{\lsb@l}{\mathalpha}{letters}{`l}
\lowercase{\gdef~{\ifnum\the\mathgroup=\m@ne \ell \else \lsb@l \fi}}%
\endgroup

\author{Thomas Cohn}
\title{Derivatives}
\date{9/19/18} % Can also use \today

\begin{document}
\maketitle
\setlength\RaggedRightParindent{\parindent}
\RaggedRight

\defn{$\lim_{t\to{}0}\frac{f(\vec{a}+t\vec{u})-f(\vec{a})}{t}=T(\vec{u})$\n
We call this $f'(\vec{a};\vec{u})$ a \underline{directional derivative}.}

\defn{$T\in\Hom(V,W)$ is said to be \underline{bounded} if $\exists{}M\in\R_{\ge{}0}$ \st{} $\norm{T(\vec{v})}\le{}M\vec{v}$, $\forall\vec{v}\in{}V$.}

\defn{$B(V,W)$ is the set of bounded linear maps $V\to{}W$.}

\par\noindent With normed vector space $V$, for $\vec{a},\vec{u}\in{}V$, define $g_{\vec{a},\vec{u}}:\R\to{}V$, $t\mapsto{}\vec{a}+t\vec{u}$. Note that $g'_{\vec{a},\vec{u}}(t)=\vec{u}$.\n

\par\noindent Return to $f:V\to{}W$. We have $(f\of{}g_{\vec{a},\vec{u}})'(0)=\frac{(f\of{}g)(t)-f(g(0))}{t}=\frac{f(\vec{a}+\vec{u}t)-f(\vec{a})}{t}$, the directional derivative. This requires $\vec{a}$ be an interior point of the domain of $f$.\n

\par\noindent If we $\lambda$-dilate $\Graph{}f$ centered at $(\vec{a},f(\vec{a}))$, and let $\lambda\to\infty$, then the graph converges pointwise to some affine graph (if the limit exists).\n

\par\noindent To get theory on this, we need\n
(1) $f'(\vec{a};\vec{u})=T(\vec{u})$ linear in $\vec{u}$.\n
(2) $f(\vec{a})+T(\vec{y}-\vec{a})\approx{}f(\vec{y})$.\n

\par\noindent Formally, for bounded $T$, $Df(\vec{a})=T\to\lim_{h\to{}0}\frac{\norm{f(a+h)-f(h)-T(h)}}{\norm{h}}=0$. \flag{I need to check with Nikhil to make sure I have this written down right.}\n

\prop{If $Df(vec{a})$ exists and $\vec{u}\in{}V$, then $Df(\vec{a})(\vec{u})$ is $f'(\vec{a};\vec{u})$.}

\cor{$Df(\vec{a})$ is unique.\n
Proof: Directional derivatives are unique because they are limits.\proven}

\prop{$Df(\vec{a}))$ exists implies that $f$ is continuous at $\vec{a}$.\n
Proof: It is enough to show that $\vec{x}\to\vec{a}$ implies that $f(\vec{x})\to{}f(\vec{a})$. So does $f(\vec{x})-f(\vec{a})\to{}0$? \flag{I need to check with Nikhil on this part too.}}

\par\noindent Special Cases:\n
(A) $W=\R^{n}$, $f=\left[\begin{array}{c}f_{1}\\ \vdots\\ f_{n}\end{array}\right]$, $A\subset\R^{n}$.\n

\prop{$f$ is differentiable at $\vec{a}$ $\leftrightarrow$ each $f_{j}$ is differentiable at $\vec{a}$.}

\par\noindent (B) $V=\R^{m}$. $\displaystyle{}Df(\vec{a})=\left[\begin{array}{c}u_{1}\\ \vdots\\ u_{m}\end{array}\right]=\sum_{j=1}^{m}u_{j}T(e_{j})=\sum_{j=1}^{m}u_{j}f'(\vec{a_{j}};\vec{e_{j}})$ for $T=Df(\vec{a})$, and $f'$ directional derivative. This is the partial derivative.

\end{document}