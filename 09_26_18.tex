\documentclass[10pt,letterpaper]{article}
\usepackage[utf8]{inputenc}
\usepackage{amsmath}
\usepackage{amsfonts}
\usepackage{amssymb}
\usepackage{ragged2e}
\usepackage[letterpaper, margin=1in]{geometry}
\usepackage{graphicx}
\usepackage{cancel}
\usepackage{mathtools}
\usepackage{tabularx}
\usepackage{arydshln}
\usepackage{tensor}
\usepackage{array}
\usepackage{xcolor}
\usepackage[boxed]{algorithm}
\usepackage[noend]{algpseudocode}
\usepackage{listings}
\usepackage{textcomp}
\usepackage[pdf,tmpdir,singlefile]{graphviz}

%%%%%%%%%%%%%%%%%%%%%%%%%%%%%
% Formatting commands
%%%%%%%%%%%%%%%%%%%%%%%%%%%%%
\newcommand{\n}{\hfill\break}
\newcommand{\lemma}[1]{\par\noindent\settowidth{\hangindent}{\textbf{Lemma: }}\textbf{Lemma: }#1\n}
\newcommand{\defn}[1]{\par\noindent\settowidth{\hangindent}{\textbf{Defn: }}\textbf{Defn: }#1\n}
\newcommand{\thm}[1]{\par\noindent\settowidth{\hangindent}{\textbf{Thm: }}\textbf{Thm: }#1\n}
\newcommand{\prop}[1]{\par\noindent\settowidth{\hangindent}{\textbf{Prop: }}\textbf{Prop: }#1\n}
\newcommand{\cor}[1]{\par\noindent\settowidth{\hangindent}{\textbf{Cor: }}\textbf{Cor: }#1\n}
\newcommand{\ex}[1]{\par\noindent\settowidth{\hangindent}{\textbf{Ex: }}\textbf{Ex: }#1\n}
\newcommand{\proven}{\;$\square$\n}
\newcommand{\problem}[1]{\par\noindent{#1}\n}
\newcommand{\problempart}[2]{\par\settowidth{\hangindent}{\textbf{(#1)} \indent{}}\textbf{(#1)} #2\n}
\newcommand{\ptxt}[1]{\textrm{\textnormal{#1}}}
\newcommand{\inlineeq}[1]{\centerline{$\displaystyle #1$}}
\newcommand{\pageline}{\noindent\rule{\textwidth}{0.1pt}}

%%%%%%%%%%%%%%%%%%%%%%%%%%%%%
% Math commands
%%%%%%%%%%%%%%%%%%%%%%%%%%%%%
% Set Theory
\newcommand{\card}[1]{\left|#1\right|}
\newcommand{\set}[1]{\left\{#1\right\}}
\newcommand{\ps}[1]{\mathcal{P}\left(#1\right)}
\newcommand{\pfinite}[1]{\mathcal{P}^{\ptxt{finite}}\left(#1\right)}
\newcommand{\naturals}{\mathbb{N}}
\newcommand{\N}{\naturals}
\newcommand{\integers}{\mathbb{Z}}
\newcommand{\Z}{\integers}
\newcommand{\rationals}{\mathbb{Q}}
\newcommand{\Q}{\rationals}
\newcommand{\reals}{\mathbb{R}}
\newcommand{\R}{\reals}
\newcommand{\complex}{\mathbb{C}}
\newcommand{\C}{\complex}
\newcommand{\comp}{^{\complement}}
\newcommand{\Hom}{\ptxt{Hom}\>}

% Graph Theory
\renewcommand{\deg}[1]{\ptxt{deg}\left(#1\right)}
\newcommand{\degp}[1]{\ptxt{deg}^{+}\!\!\left(#1\right)}
\newcommand{\degn}[1]{\ptxt{deg}^{-}\!\!\left(#1\right)}

% Standard Math
\newcommand{\inv}{^{-1}}
\newcommand{\abs}[1]{\left|#1\right|}
\newcommand{\ceil}[1]{\left\lceil{}#1\right\rceil}
\newcommand{\floor}[1]{\left\lfloor{}#1\right\rfloor{}}
\newcommand{\conj}[1]{\overline{#1}}
\newcommand{\of}{\circ}
\newcommand{\tri}{\triangle}
\newcommand{\inj}{\hookrightarrow}
\newcommand{\surj}{\twoheadrightarrow}
\newcommand{\mapsfrom}{\mathrel{\reflectbox{\ensuremath{\mapsto}}}}
\newcommand{\Graph}{\ptxt{Graph}\>}
\newcommand{\ndiv}{\nmid}
\renewcommand{\epsilon}{\varepsilon}

% Linear Algebra
\newcommand{\Id}{\textrm{\textnormal{Id}}}
\newcommand{\im}{\textrm{\textnormal{im}}}
\newcommand{\norm}[1]{\abs{\abs{#1}}}
\newcommand{\tpose}{^{T}}
\newcommand{\iprod}[1]{\left<#1\right>}
\newcommand{\trace}{\ptxt{tr}}
\newcommand{\chgBasMat}[3]{\!\!\tensor*[_{#1}]{\left[#2\right]}{_{#3}}}
\newcommand{\vecBas}[2]{\tensor*[]{\left[#1\right]}{_{#2}}}
\newcommand{\GL}{\ptxt{GL}\>}
\newcommand{\Mat}{\ptxt{Mat}\>}

% Topology
\newcommand{\closure}[1]{\overline{#1}}
\newcommand{\uball}{\mathcal{U}}
\newcommand{\Int}{\ptxt{Int}\>}
\newcommand{\Ext}{\ptxt{Ext}\>}
\newcommand{\Bd}{\ptxt{Bd}\>}

% Proofs
\newcommand{\st}{s.t.}
\newcommand{\unique}{!}

% Algorithms
\algrenewcommand{\algorithmiccomment}[1]{\hskip 1em \texttt{// #1}}
\algrenewcommand\algorithmicrequire{\textbf{Input:}}
\algrenewcommand\algorithmicensure{\textbf{Output:}}

%%%%%%%%%%%%%%%%%%%%%%%%%%%%%
% Other commands
%%%%%%%%%%%%%%%%%%%%%%%%%%%%%
\newcommand{\flag}[1]{\textbf{\textcolor{red}{#1}}}

%%%%%%%%%%%%%%%%%%%%%%%%%%%%%
% Make l's curvy in math environments
%%%%%%%%%%%%%%%%%%%%%%%%%%%%%
\mathcode`l="8000
\begingroup
\makeatletter
\lccode`\~=`\l
\DeclareMathSymbol{\lsb@l}{\mathalpha}{letters}{`l}
\lowercase{\gdef~{\ifnum\the\mathgroup=\m@ne \ell \else \lsb@l \fi}}%
\endgroup

\author{Professor David Barrett\\ \small\textit{Transcribed by Thomas Cohn}}
\title{More Notes on Derivatives}
\date{9/26/18} % Can also use \today

\begin{document}
\maketitle
\setlength\RaggedRightParindent{\parindent}
\RaggedRight

\par\noindent Refresher from Monday: Given $A^{\ptxt{open}}\subset{}V$ normed vector space, with $W$ normed vector space, and $f:A\to{}W$.\n
$f$ is $C^{1}$ $\leftrightarrow$ $f$ is continuously differentiable\n
\phantom{$f$ is $C^{1}$} $\leftrightarrow$ $f$ is differentiable at each $\vec{a}\in{}A$ and $Df:A\to{}B(V,W)$, $\vec{a}\mapsto{}Df(\vec{a})$.\n

\par\noindent $C^{1}(A,W)$ is the set of all $C^{1}$ $f:A\to{}W$.\n

\par\noindent Special Case: $V=\R^{m}$, $W=\R^{n}$, so we have $B(V,W)\leftrightarrow{}\Mat(n,m)$.\n
Then the $(j,k)$ entry of $Df(\vec{a})$ is $D_{k}f_{j}(\vec{a})=f_{j}'(\vec{a};\vec{e_{k}})$\n
$f$ is $C^{1}$ iff it is differentiable at each $\vec{a}\in{}A$ and each $D_{k}f_{j}:A\to\R$ is continuous.\n

\thm{Given $f:A^{\ptxt{osso}\R^{n}}\to\R^{n}$, and all $D_{k}f_{j}$ exist and are continuous on $A$, then $f\in{}C^{1}(A,\R^{n})$.\n
Proof: It is enough to show that $f$ is differentiable at each $\vec{a}\in{}A$. From last wednesday, it is enough to show that each component is differentiable.\n
\n
Some board work for $m=2$:\n
Fix $\vec{a}=(a_{1},a_{2})\in{}A$, and consider small $\vec{h}=(h_{1},h_{2})$.\n
$\displaystyle\begin{array}{ccrl}
 & & \cdot & (a_{1}+h_{1},a_{2}+h_{2})\\
 & & \uparrow & \vec{q}\\
 \cdot & \rightarrow & \cdot & \\
 (a_{1},a_{2}) & \vec{p} & (a_{1}+h_{1},a_{2}) & \\
\end{array}$}\n

\par\noindent $f(a_{1}+h_{1},a_{2})-f(a_{1},a_{2})=D_{1}f(\vec{p})h_{1}$ for some $\vec{p}$ by MVT.\n
$f(a_{1}+h_{1},a_{2}+h_{2})-f(a_{1}+h_{1},a_{2})=D_{2}f(\vec{q})h_{2}$ for some $\vec{q}$ by MVT.\n

\par\noindent If $D_{f}(\vec{a})$ exists, it must be $\left(\begin{array}{cc}D_{1}f(\vec{a}) & D_{2}f(\vec{a})\end{array}\right)\left(\begin{array}{c}h_{1}\\ h_{2}\end{array}\right)f(\vec{a}+\vec{h})-f(\vec{a})=D_{1}f(\vec{p})h_{1}+D_{2}f(\vec{q})h_{2}$.\n

\par\noindent \textbf{Goal}: ($\star$) $\displaystyle\frac{f(\vec{a}+\vec{h})-f(\vec{a})-(D_{1}f(\vec{a})h_{1}+D_{2}f(\vec{a})h_{2})}{\norm{\vec{h}}}\to{}0$ as $\vec{h}\to\vec{0}$.\n
Well, this is $=(D_{1}f(\vec{p})-D_{1}f(\vec{a}))\frac{h_{1}}{\norm{\vec{h}}}+(D_{2}f(\vec{q})-D_{2}f(\vec{a}))\frac{h_{2}}{\norm{\vec{h}}}$.\n

\par\noindent As $\vec{h}\to\vec{0}$, $\frac{h_{1}}{\norm{\vec{h}}}$ is bounded, since $\abs{\frac{h_{1}}{\norm{\vec{h}}}}\le{}1$. The same applies for $\frac{h_{2}}{\norm{\vec{h}}}$.\n

\par\noindent As $\vec{h}\to\vec{0}$, $\vec{p},\vec{q}\to\vec{a}$. So $D_{1}f(\vec{p})\to{}D_{1}f(\vec{a})$ and $D_{2}f(\vec{q})\to{}D_{2}f(\vec{a})$. So $D_{1}f(\vec{p})-D_{1}f(\vec{a})\to{}0$ and $D_{2}f(\vec{q})-D_{2}f(\vec{a})\to{}0$.\n

\par\noindent Therefore, $(D_{1}f(\vec{p})-D_{1}f(\vec{a}))\frac{h_{1}}{\norm{\vec{h}}}+(D_{2}f(\vec{q})-D_{2}f(\vec{a}))\frac{h_{2}}{\norm{\vec{h}}}\to{}0$.\proven

\ex{Generalize the above proof (Munkres 6.2).\n}

\thm{If $A\subset{}V$ is open ($V$ is a normed vector space), $f:A\to{}W$ (another normed vector space), and $D_{f}(\vec{a};\vec{u})$ exist and are continuous for $(\vec{a},\vec{u})\in{}A\times{}V$, then $f$ is differentiable, and $f\in{}C^{1}(A,W)$.\n}

\par\noindent $f\in{}C^{1}(A^{\ptxt{osso}V},W)$ leads to $D_{f}:A\to{}B(V,W)$ continuous. It might be differentiable.\n
If so, have $D^{2}f=D(Df):A\to{}B(V,B(V,W))$.\n

\par\noindent $f\in{}C^{2}(A,W)\leftrightarrow{}Df\in{}C^{1}(A,B(V,W))\leftrightarrow{}Df$ is differentiable at each $\vec{a}$ and $D^{2}f$ is continuous.\n
$\leftarrow{}V=\R^{m},W=\R^{m}$, and $f'$ is $C^{1}$ and each $D_{l}D_{k}f_{j}$ exists and is continuous on $A$.\n

\par\noindent $C^{r}(A,W)$ follows similarly. $C^{r}(A)=C^{r}(A,W)$.\n

\thm{If $f\in{}C^{2}(A^{\ptxt{osso}\R^{2}},\R)$, then $\displaystyle{}D_{2}D_{1}f(a,b)=\lim_{(h,k)\to(0,0)}\frac{f(a+h,b+k)-f(a+h,b)-f(a,b+k)+f(a,b)}{hk}$\n
Proof: Let $\varphi(s)=f(s,b+k)-f(s,b)$. It's differentiable by the chain rule.\n
$\varphi'(s)=D_{1}f(s,b+k)-D_{1}f(s,b)$.\n
\n
So the numerator is $\varphi(a+h)-\varphi(a)=\varphi'(s_{0})h$ for some $s_{0}$ by the MVT.\n
So this is equal to $(D_{1}f(s_{0},b+k)-D_{1}f(s_{0},b))h$.\n
Applying the MVT again gives us $(D_{2}D_{1}f(s_{0},t_{0})kh$ for some $s_{0},t_{0}$.\n
\n
So $\displaystyle\lim_{(h,k)\to(0,0)}\frac{D_{2}D_{1}f(s_{0},t_{0})hk}{kh}=D_{2}D_{1}f(a,b)$.\proven}

\cor{$f$ as above $\to$ $D_{2}D_{1}f=D_{1}D_{2}f$ (Clairaut's Theorem). Note that the existence of $D^{2}f$ is not enough for this result. See Munkres \S{}6 \#10.}

\par\noindent Notation: For differentiable $f:A^{\ptxt{osso}V}\to{}W$, $\vec{u}\in{}V$, set $\begin{array}{ll}D_{\vec{u}}f:A\to{}W\\ \vec{a}\mapsto{}f'(\vec{a};\vec{u})=Df(\vec{a})(\vec{u})\end{array}$.\n

\par\noindent If $f\in{}C^{2}(A,\R)$, then $D_{\vec{u_{1}}}D_{\vec{u_{2}}}f=D_{\vec{u_{2}}}D_{\vec{u_{1}}}f$.\n

\par\noindent Proof 1: Apply chain rule twice to $(x_{1},x_{2})\mapsto{}f(\vec{a}+x_{1}\vec{u_{1}}+x_{2}\vec{u_{2}})$.\n

\par\noindent Proof 2: Study $\displaystyle\lim_{(h,k)\to(0,0}\frac{f(\vec{a}+h\vec{u_{1}}+k\vec{u_{2}})-f(\vec{a}+h\vec{u_{1}})-f(\vec{a}+k\vec{u_{2}})+f(\vec{a})}{hk}$.\n

\par\noindent Spoiler! It equals both $D_{\vec{u_{1}}}D_{\vec{u_{2}}}f$ and $D_{\vec{u_{2}}}D_{\vec{u_{1}}}f$.\n

\cor{This also works for $f\in{}C^{2}(A,\R^{m})$.}

\end{document}