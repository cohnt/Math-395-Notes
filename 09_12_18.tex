\documentclass[10pt,letterpaper]{article}
\usepackage[utf8]{inputenc}
\usepackage{amsmath}
\usepackage{amsfonts}
\usepackage{amssymb}
\usepackage{ragged2e}
\usepackage[letterpaper, margin=1in]{geometry}
\usepackage{graphicx}
\usepackage{cancel}
\usepackage{mathtools}
\usepackage{tabularx}
\usepackage{arydshln}
\usepackage{tensor}
\usepackage{array}
\usepackage{xcolor}

%%%%%%%%%%%%%%%%%%%%%%%%%%%%%
% Formatting commands
%%%%%%%%%%%%%%%%%%%%%%%%%%%%%
\newcommand{\n}{\hfill\break}
\newcommand{\lemma}[1]{\par\noindent\settowidth{\hangindent}{\textbf{Lemma: }}\textbf{Lemma: }#1\n}
\newcommand{\defn}[1]{\par\noindent\settowidth{\hangindent}{\textbf{Defn: }}\textbf{Defn: }#1\n}
\newcommand{\thm}[1]{\par\noindent\settowidth{\hangindent}{\textbf{Thm: }}\textbf{Thm: }#1\n}
\newcommand{\prop}[1]{\par\noindent\settowidth{\hangindent}{\textbf{Prop: }}\textbf{Prop: }#1\n}
\newcommand{\ex}[1]{\par\noindent\settowidth{\hangindent}{\textbf{Ex: }}\textbf{Ex: }#1\n}
\newcommand{\proven}{\;$\square$\n}
\newcommand{\problem}[1]{\par\noindent{#1}\n}
\newcommand{\problempart}[2]{\par\settowidth{\hangindent}{\textbf{(#1)} \indent{}}\textbf{(#1)} #2\n}
\newcommand{\ptxt}[1]{\textrm{\textnormal{#1}}}
\newcommand{\inlineeq}[1]{\n\centerline{$\displaystyle #1$}}
\newcommand{\pageline}{\noindent\rule{\textwidth}{0.1pt}}

%%%%%%%%%%%%%%%%%%%%%%%%%%%%%
% Math commands
%%%%%%%%%%%%%%%%%%%%%%%%%%%%%
% Set Theory
\newcommand{\card}[1]{\left|#1\right|}
\newcommand{\set}[1]{\left\{#1\right\}}
\newcommand{\ps}[1]{\mathcal{P}\left(#1\right)}
\newcommand{\pfinite}[1]{\mathcal{P}^{\ptxt{finite}}\left(#1\right)}
\newcommand{\naturals}{\mathbb{N}}
\newcommand{\N}{\naturals}
\newcommand{\integers}{\mathbb{Z}}
\newcommand{\Z}{\integers}
\newcommand{\rationals}{\mathbb{Q}}
\newcommand{\Q}{\rationals}
\newcommand{\reals}{\mathbb{R}}
\newcommand{\R}{\reals}
\newcommand{\complex}{\mathbb{C}}
\newcommand{\C}{\complex}
\newcommand{\comp}{^{\complement}}

% Graph Theory
\renewcommand{\deg}[1]{\ptxt{deg}\left(#1\right)}
\newcommand{\degp}[1]{\ptxt{deg}^{+}\!\!\left(#1\right)}
\newcommand{\degn}[1]{\ptxt{deg}^{-}\!\!\left(#1\right)}

% Standard Math
\newcommand{\inv}{^{-1}}
\newcommand{\abs}[1]{\left|#1\right|}
\newcommand{\ceil}[1]{\left\lceil{}#1\right\rceil}
\newcommand{\floor}[1]{\left\lfloor{}#1\right\rfloor{}}
\newcommand{\conj}[1]{\overline{#1}}
\newcommand{\of}{\circ}
\newcommand{\tri}{\triangle}
\newcommand{\inj}{\hookrightarrow}
\newcommand{\surj}{\twoheadrightarrow}
\newcommand{\mapsfrom}{\mathrel{\reflectbox{\ensuremath{\mapsto}}}}

% Linear Algebra
\newcommand{\Id}{\textrm{\textnormal{Id}}}
\newcommand{\im}{\textrm{\textnormal{im}}}
\newcommand{\norm}[1]{\abs{\abs{#1}}}
\newcommand{\tpose}{^{T}}
\newcommand{\iprod}[1]{\left<#1\right>}
\newcommand{\trace}{\ptxt{tr}}
\newcommand{\chgBasMat}[3]{\!\!\tensor*[_{#1}]{\left[#2\right]}{_{#3}}}
\newcommand{\vecBas}[2]{\tensor*[]{\left[#1\right]}{_{#2}}}

% Topology
\newcommand{\closure}[1]{\bar{#1}}
\newcommand{\uball}{\mathcal{U}}
\newcommand{\Int}{\ptxt{Int}\>}
\newcommand{\Ext}{\ptxt{Ext}\>}
\newcommand{\Bd}{\ptxt{Bd}\>}

% Proofs
\newcommand{\st}{s.t.}
\newcommand{\unique}{!}

%%%%%%%%%%%%%%%%%%%%%%%%%%%%%
% Other commands
%%%%%%%%%%%%%%%%%%%%%%%%%%%%%
\newcommand{\flag}[1]{\textbf{\textcolor{red}{#1}}}

%%%%%%%%%%%%%%%%%%%%%%%%%%%%%
% Make l's curvy in math environments
%%%%%%%%%%%%%%%%%%%%%%%%%%%%%
\mathcode`l="8000
\begingroup
\makeatletter
\lccode`\~=`\l
\DeclareMathSymbol{\lsb@l}{\mathalpha}{letters}{`l}
\lowercase{\gdef~{\ifnum\the\mathgroup=\m@ne \ell \else \lsb@l \fi}}%
\endgroup

\author{Professor David Barrett\\ \small\textit{Transcribed by Thomas Cohn}}
\title{Bolzano-Weierstrass}
\date{9/12/18} % Can also use \today

\begin{document}
\maketitle
\setlength\RaggedRightParindent{\parindent}
\RaggedRight

\thm{(Bolzano-Weierstrass) $(X,d)$ compact $\leftrightarrow$ $X$ sequentially compact -- i.e. every sequence admits a convergent subsequence.\n
\n
Proof: $\Rightarrow$ Suppose to the contrary, we have $(v_{j})$ in $X$ with no convergent subsequence. Then for every $y\in{}X$, $\exists{}U_{y}$ open with $y\in{}U_{y}$ such that $U_{y}$ only contains finitely many $v_{j}$.\n
So $X=\bigcup_{y\in{}X}U_{y}=\bigcup_{i<k}U_{y_{i}}$. But that only contains finitely many $v_{j}$. Oops!\n
\n
$\Leftarrow$ Lemma: If $X$ is a sequentially compact metric space, then $\forall\epsilon>0$ we can cover $X$ by finitely many $\epsilon$-balls. We then call $X$ ``totally bounded''.\n
Proof: suppose not. Pick $v_{1}\in{}X$\n
\phantom{Proof: suppose not. Pick }$v_{2}\not\in{}B_{\epsilon}(v_{1})$\n
\phantom{Proof: suppose not. Pick $v_{2}$}$\;\;\vdots$\n
\phantom{Proof: suppose not. Pick }$v_{n}\not\in{}B_{\epsilon}(v_{n-1})\cup\cdots\cup{}B_{\epsilon}(v_{1})$. Oops.\proven
Suppose $X=\bigcup_{\alpha\in{}A}X_{\alpha}$ with each $X_{\alpha}$ open. Call $S\subset{}X$ ``good'' if $S$ contained in a finite union of $X_{\alpha}$. Otherwise, call $S$ ``bad''. Note that the finite union of ``good'' sets is ``good''. Our goal is to show that $X$ is good. Suppose, to the contrary, that $X$ is bad. Lemma + note $\to\forall{}m\in\N$, $\exists{}v_{m}\in{}X$ \st{} $B_{1/m}(v_{m})$ is bdd.\n
Choose $v_{m_{k}}\to{}v\in{}X$ by hypothesis. Choose $\alpha$ \st{} $v\in{}X_{\alpha}$. Then $\exists\epsilon>0$ \st{} $B_{\epsilon}(v)\subset{}X_{\alpha}$. Pick $k\in\N$ \st{} $m_{k}>2/\epsilon$ and $d(v,v_{m_{k}})<\epsilon/2$. But then $B_{1/m_{k}}(v_{m_{k}})$ is good. Oops!\proven
\n
This proves the Bolzano-Weierstrass Theorem.\proven}

\par\noindent It is also true that $(X,d)$ is compact if and only if $(X,d)$ is totally bounded and complete. We will receive a handout on this.\n

\par\noindent Let $f:X\to{}Y$ with $X$, $Y$ metric spaces with metrics $d_{X}$ and $d_{Y}$, respectively.\n

\defn{We say $f$ is \underline{Lipschitz} if $\exists{}C\in\R_{>0}$ \st{} $d_{Y}(f(x_{1}),f(x_{2}))\le{}Cd_{X}(x_{1},x_{2})$, implying continuity.}

\ex{The $\inf$ of Lipschitz constants of $f$ is a Lipschitz constant of $f$.}

\par\noindent $f$ is uniformly continuous means we can choose $\delta$ independently of $x_{0}$.\n

\prop{$f$ is Lipschitz $\to$ $f$ is uniformly continuous.}

\defn{$f$ is a \underline{contraction} if $f$ is a Lipschitz mapping with $C<1$.}

\defn{A \underline{bi-Lipschitz} map $f:X\to{}Y$ is a bijection \st{} $f$, $f\inv$ are Lipschitz.}

\ex{$f:\R\to\R$, $x\mapsto{}x+\frac{\sin{x}}{37}$ is bi-Lipschitz.}

\defn{$f$ is said to be an \underline{isometry} iff $d_{Y}(f(x_{1}),f(x_{2}))=d_{X}(x_{1},x_{2})$.}

\defn{Suppose $X$ is a vector space (over $\R$ or $\C$) and is also a metric space with metric $d$. $d$ is said to be \underline{translation invariant} $\leftrightarrow{}d(x_{1},x_{2})=d(x_{1}+y,x_{2}+y)$.}

\ex{Show that $d$ is translation-invariant iff $d(x_{1},x_{2})=d(0,x_{2}-x_{1})$.}

\defn{$d$ is \underline{homogenous} $\leftrightarrow{}d(tx_{1},tx_{2})=\abs{t}d(x_{1},x_{2})$.}

\defn{If $d$ is both translation-invariant and homogenous, then $d$ defines a \underline{norm}.}

\end{document}