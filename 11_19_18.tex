\documentclass[10pt,letterpaper]{article}
\usepackage[utf8]{inputenc}
\usepackage[intlimits]{amsmath}
\usepackage{amsfonts}
\usepackage{amssymb}
\usepackage{ragged2e}
\usepackage[letterpaper, margin=1in]{geometry}
\usepackage{graphicx}
\usepackage{cancel}
\usepackage{mathtools}
\usepackage{tabularx}
\usepackage{arydshln}
\usepackage{tensor}
\usepackage{array}
\usepackage{xcolor}
\usepackage[boxed]{algorithm}
\usepackage[noend]{algpseudocode}
\usepackage{listings}
\usepackage{textcomp}
\usepackage[pdf,tmpdir,singlefile]{graphviz}
\usepackage{mathrsfs}
\usepackage{bbm}
\usepackage{tikz}
\usepackage{enumitem}
\usepackage{arydshln}

%%%%%%%%%%%%%%%%%%%%%%%%%%%%%
% Formatting commands
%%%%%%%%%%%%%%%%%%%%%%%%%%%%%
\newcommand{\n}{\hfill\break}
\newcommand{\up}{\vspace{-\baselineskip}}
\newcommand{\lemma}[1]{\par\noindent\settowidth{\hangindent}{\textbf{Lemma: }}\textbf{Lemma: }#1}
\newcommand{\defn}[1]{\par\noindent\settowidth{\hangindent}{\textbf{Defn: }}\textbf{Defn: }#1\n}
\newcommand{\thm}[1]{\par\noindent\settowidth{\hangindent}{\textbf{Thm: }}\textbf{Thm: }#1\n}
\newcommand{\prop}[1]{\par\noindent\settowidth{\hangindent}{\textbf{Prop: }}\textbf{Prop: }#1\n}
\newcommand{\cor}[1]{\par\noindent\settowidth{\hangindent}{\textbf{Cor: }}\textbf{Cor: }#1\n}
\newcommand{\ex}[1]{\par\noindent\settowidth{\hangindent}{\textbf{Ex: }}\textbf{Ex: }#1\n}
\newcommand{\proven}{\;$\square$\n}
\newcommand{\problem}[1]{\par\noindent{#1}\n}
\newcommand{\problempart}[2]{\par\noindent\indent{}\settowidth{\hangindent}{\textbf{(#1)} \indent{}}\textbf{(#1)} #2\n}
\newcommand{\ptxt}[1]{\textrm{\textnormal{#1}}}
\newcommand{\inlineeq}[1]{\centerline{$\displaystyle #1$}}
\newcommand{\pageline}{\noindent\rule{\textwidth}{0.1pt}}

%%%%%%%%%%%%%%%%%%%%%%%%%%%%%
% Math commands
%%%%%%%%%%%%%%%%%%%%%%%%%%%%%
% Set Theory
\newcommand{\card}[1]{\left|#1\right|}
\newcommand{\set}[1]{\left\{#1\right\}}
\newcommand{\ps}[1]{\mathcal{P}\left(#1\right)}
\newcommand{\pfinite}[1]{\mathcal{P}^{\ptxt{finite}}\left(#1\right)}
\newcommand{\naturals}{\mathbb{N}}
\newcommand{\N}{\naturals}
\newcommand{\integers}{\mathbb{Z}}
\newcommand{\Z}{\integers}
\newcommand{\rationals}{\mathbb{Q}}
\newcommand{\Q}{\rationals}
\newcommand{\reals}{\mathbb{R}}
\newcommand{\R}{\reals}
\newcommand{\complex}{\mathbb{C}}
\newcommand{\C}{\complex}
\newcommand{\comp}{^{\complement}}
\DeclareMathOperator{\Hom}{Hom}
\newcommand{\Ind}{\mathbbm{1}}
\newcommand{\cut}{\setminus}

% Graph Theory
\let\deg\relax
\DeclareMathOperator{\deg}{deg}
\newcommand{\degp}{\ptxt{deg}^{+}}
\newcommand{\degn}{\ptxt{deg}^{-}}
\newcommand{\precdot}{\mathrel{\ooalign{$\prec$\cr\hidewidth\hbox{$\cdot\mkern0.5mu$}\cr}}}
\newcommand{\succdot}{\mathrel{\ooalign{$\cdot\mkern0.5mu$\cr\hidewidth\hbox{$\succ$}\cr\phantom{$\succ$}}}}
\DeclareMathOperator{\cl}{cl}
\DeclareMathOperator{\affdim}{affdim}

% Probability
\newcommand{\Prob}{\mathbb{P}}
\newcommand{\Avg}{\mathbb{E}}

% Standard Math
\newcommand{\inv}{^{-1}}
\newcommand{\abs}[1]{\left|#1\right|}
\newcommand{\ceil}[1]{\left\lceil{}#1\right\rceil{}}
\newcommand{\floor}[1]{\left\lfloor{}#1\right\rfloor{}}
\newcommand{\conj}[1]{\overline{#1}}
\newcommand{\of}{\circ}
\newcommand{\tri}{\triangle}
\newcommand{\inj}{\hookrightarrow}
\newcommand{\surj}{\twoheadrightarrow}
\newcommand{\mapsfrom}{\mathrel{\reflectbox{\ensuremath{\mapsto}}}}
\newcommand{\mapsdown}{\rotatebox[origin=c]{-90}{$\mapsto$}\mkern2mu}
\newcommand{\mapsup}{\rotatebox[origin=c]{90}{$\mapsto$}\mkern2mu}
\newcommand{\ndiv}{\nmid}
\renewcommand{\epsilon}{\varepsilon}
\newcommand{\divides}{\mid}
\newcommand{\ndivides}{\nmid}
\DeclareMathOperator{\lcm}{lcm}

% Linear Algebra
\newcommand{\Id}{\textrm{\textnormal{Id}}}
\newcommand{\im}{\textrm{\textnormal{im}}}
\newcommand{\norm}[1]{\abs{\abs{#1}}}
\newcommand{\tpose}{^{T}}
\newcommand{\iprod}[1]{\left<#1\right>}
\DeclareMathOperator{\trace}{tr}
\newcommand{\chgBasMat}[3]{\!\!\tensor*[_{#1}]{\left[#2\right]}{_{#3}}}
\newcommand{\vecBas}[2]{\tensor*[]{\left[#1\right]}{_{#2}}}
\DeclareMathOperator{\GL}{GL}
\DeclareMathOperator{\Mat}{Mat}
\DeclareMathOperator{\vspan}{span}
\DeclareMathOperator{\rank}{rank}
\newcommand{\V}[1]{\vec{#1}}

% Topology
\newcommand{\closure}[1]{\overline{#1}}
\newcommand{\uball}{\mathcal{U}}
\DeclareMathOperator{\Int}{Int}
\DeclareMathOperator{\Ext}{Ext}
\DeclareMathOperator{\Bd}{Bd}
\DeclareMathOperator{\rInt}{rInt}
\DeclareMathOperator{\ch}{ch}
\DeclareMathOperator{\ah}{ah}

% Analysis
\DeclareMathOperator{\Graph}{Graph}
\DeclareMathOperator{\epi}{epi}
\DeclareMathOperator{\hypo}{hypo}
\DeclareMathOperator{\supp}{supp}
\newcommand{\lint}[2]{\underset{#1}{\overset{#2}{{\color{black}\underline{{\color{white}\overline{{\color{black}\int}}\color{black}}}}}}}
\newcommand{\uint}[2]{\underset{#1}{\overset{#2}{{\color{white}\underline{{\color{black}\overline{{\color{black}\int}}\color{black}}}}}}}
\newcommand{\alignint}[2]{\underset{#1}{\overset{#2}{{\color{white}\underline{{\color{white}\overline{{\color{black}\int}}\color{black}}}}}}}
\newcommand{\extint}{\ptxt{ext}\int}
\newcommand{\extalignint}[2]{\ptxt{ext}\alignint{#1}{#2}}
\newcommand{\conv}{\ast}

% Proofs
\newcommand{\st}{s.t.}
\newcommand{\unique}{!}

% Brackets
\newcommand{\paren}[1]{\left(#1\right)}
\renewcommand{\brack}[1]{\left[#1\right]}
\renewcommand{\brace}[1]{\left\{#1\right\}}
\newcommand{\ang}[1]{\left<#1\right>}

% Algorithms
\algrenewcommand{\algorithmiccomment}[1]{\hskip 1em \texttt{// #1}}
\algrenewcommand\algorithmicrequire{\textbf{Input:}}
\algrenewcommand\algorithmicensure{\textbf{Output:}}
\newcommand{\parSymbol}{\P}
\renewcommand{\P}{\ptxt{\textbf{P}}}
\newcommand{\NP}{\ptxt{\textbf{NP}}}
\newcommand{\NPC}{\ptxt{\textbf{NP-Complete}}}
\newcommand{\NPH}{\ptxt{\textbf{NP-Hard}}}
\newcommand{\EXP}{\ptxt{\textbf{EXP}}}

%%%%%%%%%%%%%%%%%%%%%%%%%%%%%
% Other commands
%%%%%%%%%%%%%%%%%%%%%%%%%%%%%
\newcommand{\flag}[1]{\textbf{\textcolor{red}{#1}}}

%%%%%%%%%%%%%%%%%%%%%%%%%%%%%
% Make l's curvy in math environments
%%%%%%%%%%%%%%%%%%%%%%%%%%%%%
\mathcode`l="8000
\begingroup
\makeatletter
\lccode`\~=`\l
\DeclareMathSymbol{\lsb@l}{\mathalpha}{letters}{`l}
\lowercase{\gdef~{\ifnum\the\mathgroup=\m@ne \ell \else \lsb@l \fi}}%
\endgroup

\newcommand{\B}{
    \begin{tikzpicture}
    \filldraw [fill=red, draw=black] (0, 0) rectangle (0.37, 0.45);
    \draw [line width=0.5mm, white ] (0.1,0.08) -- (0.1,0.38);
    \draw[line width=0.5mm, white ] (0.1, 0.35) .. controls (0.2, 0.35) and (0.4, 0.2625) .. (0.1, 0.225);
    \draw[line width=0.5mm, white ] (0.1, 0.225) .. controls (0.2, 0.225) and (0.4, 0.1625) .. (0.1, 0.1);
    \end{tikzpicture}
}

\author{Thomas Cohn}
\title{Parallelopipeds and the Pythagorean Theorem}
\date{11/19/18} % Can also use \today

\begin{document}
\maketitle
\setlength\RaggedRightParindent{\parindent}
\RaggedRight

\par\noindent Let $A\in\Mat(n,k)$. Consider $A\tpose{}A\in\Mat(k,k)$.\n

\par\noindent Claim: $\ker{}A\tpose{}A=\ker{}A$\n
Proof: $\supset$ trivial\n
\phantom{Proof: }$\subset$ $A\tpose{}A\vec{x}=\vec{0}\Rightarrow\norm{A\vec{x}}^{2}=\iprod{A\vec{x},A\vec{x}}=(\vec{x}A)\tpose{}A\vec{x}=\vec{0}$.

\cor{$\rank{}A\tpose{}A=k-\dim(\ker(A\tpose{}A))=k-\dim(\ker(A))=\rank(A)$\n
\underline{or} $\det(A\tpose{}A)=0\Leftrightarrow\rank(A)<k$.}

\cor{$k>n\Rightarrow\det(A\tpose{}A)=0$}

\par\noindent Claim: All eigenvalues of $A\tpose{}A$ are non-negative.\n
Proof: If $A\tpose{}A\vec{x}=\lambda\vec{x}$ (with $\vec{x}\ne\vec{0}$), then $\iprod{A\tpose{}A\vec{x},\vec{x}}=\norm{A\vec{x}}^{2}$, and $\iprod{A\tpose{}A\vec{x},\vec{x}}=\iprod{\lambda\vec{x},\vec{x}}=\lambda\norm{\vec{x}}^{2}$.\n
So $\lambda=\frac{\norm{A\vec{x}}^{2}}{\norm{\vec{x}}^{2}}=\paren{\frac{\norm{A\vec{x}}}{\norm{\vec{x}}}}^{2}$. So $\lambda\ge{}0$.\proven

\cor{$\det{}A\tpose{}A\ge{}0$}

\par\noindent Recall: Thm 21.2 $A\in\Mat(n,k)\Rightarrow\exists{}B\in{}O_{n}(\R)$ (i.e. $B\tpose{}B=\Id$) with $BA=\paren{\begin{array}{c}M\\ 0\end{array}}\in\Mat(k,n)$.\n
Note: $M\tpose{}M=\paren{\begin{array}{cc}M\tpose & 0\end{array}}\paren{\begin{array}{c}M\\ 0\end{array}}=A\tpose{}B\tpose{}BA=A\tpose{}A$.\n
So $(\det{}M)^{2}=\det{}A\tpose{}A$, and $\abs{\det{}M}=\sqrt{\det{}A\tpose{}A}$.\n

\par\noindent Given $T:Q^{\ptxt{box in }\R^{k}}\to\R^{n}$ injective, affine (i.e. $T:\vec{x}\mapsto{}A\vec{x}+b$), then $T[Q]$ is a ``$k$-parallelopiped''.\n
We want $V_{k}:\set{k-\ptxt{p'pipeds}}\to(0,+\infty)$ unique \st{}
\begin{enumerate}[label=(\arabic*)]
	\item $A=\paren{\begin{array}{c}M\\ 0\end{array}}\Rightarrow{}v_{k}(T[Q])=\abs{\det{M}}v_{k}(Q)$.
	\item $h:\vec{x}\mapsto{}B^{\ptxt{orthogonal}}\vec{x}+\vec{p}\Rightarrow{}v_{k}((h\of{}T)[Q])=v_{k}(T[Q])$.
\end{enumerate}

\par\noindent Choose $B$ as in Thm 21.2, suitable $\vec{p}$. Then we get $(h\of{}T):\vec{x}\mapsto\paren{\begin{array}{c}M\\ 0\end{array}}\vec{x}=BA\vec{x}$.\n
Thus, $v_{k}(T[Q])=\sqrt{\det(A\tpose{}A)}v(Q)$.\n

\defn{$V_{k}(T[Q])=\sqrt{\det{}A\tpose{}A}v(Q)$}

\par\noindent Check (1) holds: $A=\paren{\begin{array}{c}M\\ 0\end{array}}\to\sqrt{\det(A\tpose{}a)}=\abs{\det{}M}\Rightarrow{}V_{k}(T[Q])=\abs{\det{}M}v(Q)$. \checkmark\n

\par\noindent Check (2) holds: $\vec{x}\overset{h}{\mapsto}B\vec{x}+\vec{p}$ ($\R^{n}\to\R^{n}$) $\Rightarrow\det((BA)\tpose(BA))=\det(A\tpose{}A)\Rightarrow{}V_{k}((h\of{}T)[Q])=V_{k}(T[Q])$.\n

\par\noindent Useful observation: $V(A)\overset{\ptxt{def}}{=}\sqrt{\det(A\tpose{}A)}$.\n

\thm{(Pythagorean Theorem) $(V(A))^{2}$ is the sum of the squares of all $k$-by-$k$ sub-determinants of $A$.\n
Proof: Theorem 21.4.\proven}

\defn{Given $U^{\ptxt{open}}\subset\R^{k}$, $\alpha\in{}C^{1}(U,\R^{n})$, $Y=\alpha[U]$, then $Y_{\alpha}$ is a \underline{parameterized manifold}.}

\par\noindent Think of $V(D_{\alpha})=\sqrt{\det(D_{\alpha}\tpose{}D_{\alpha})}$ as the ``volume magnification factor''.\n

\defn{$V_{k}(Y_{\alpha})\overset{\ptxt{def}}{=}\extint_{U}V(D_{\alpha})$}

\par\noindent Does this only depend on $Y$ and not on $\alpha$?\n
No, it depends on both.\n

\par\noindent But suppose...\n
$\begin{array}{rcc}U & \xrightarrow{\alpha} & \\ g\ptxt{ difffeo}\downarrow{} & & Y\ptxt{ Manifold,}\\ V & \xrightarrow{\beta} & \end{array}$ with $Y=\beta[V]=\alpha[U]$. Then
\begin{align*}
V(Y_{\alpha}) & =\int_{U}\sqrt{\det(D(\beta\of{}g))\tpose{}D(\beta\of{}g))}\\
 & =\int_{U}\sqrt{\det(Dg\tpose(D\beta\of{}g)\tpose(D\beta\of{}g)Dg)}\\
 & =\int_{U}\sqrt{\det(D\beta\tpose{}D\beta)}\of{}g\abs{\det{}Dg}\\
 & =\int_{V}V(D\beta)\of{}g\abs{\det{}Dg}\\
 & =\int_{V}V(D\beta)=V(Y_{\beta})
\end{align*}

\end{document}