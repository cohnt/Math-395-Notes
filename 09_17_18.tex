\documentclass[10pt,letterpaper]{article}
\usepackage[utf8]{inputenc}
\usepackage{amsmath}
\usepackage{amsfonts}
\usepackage{amssymb}
\usepackage{ragged2e}
\usepackage[letterpaper, margin=1in]{geometry}
\usepackage{graphicx}
\usepackage{cancel}
\usepackage{mathtools}
\usepackage{tabularx}
\usepackage{arydshln}
\usepackage{tensor}
\usepackage{array}
\usepackage{xcolor}
\usepackage[boxed]{algorithm}
\usepackage[noend]{algpseudocode}
\usepackage{listings}

%%%%%%%%%%%%%%%%%%%%%%%%%%%%%
% Formatting commands
%%%%%%%%%%%%%%%%%%%%%%%%%%%%%
\newcommand{\n}{\hfill\break}
\newcommand{\lemma}[1]{\par\noindent\settowidth{\hangindent}{\textbf{Lemma: }}\textbf{Lemma: }#1\n}
\newcommand{\defn}[1]{\par\noindent\settowidth{\hangindent}{\textbf{Defn: }}\textbf{Defn: }#1\n}
\newcommand{\thm}[1]{\par\noindent\settowidth{\hangindent}{\textbf{Thm: }}\textbf{Thm: }#1\n}
\newcommand{\prop}[1]{\par\noindent\settowidth{\hangindent}{\textbf{Prop: }}\textbf{Prop: }#1\n}
\newcommand{\ex}[1]{\par\noindent\settowidth{\hangindent}{\textbf{Ex: }}\textbf{Ex: }#1\n}
\newcommand{\proven}{\;$\square$\n}
\newcommand{\problem}[1]{\par\noindent{#1}\n}
\newcommand{\problempart}[2]{\par\settowidth{\hangindent}{\textbf{(#1)} \indent{}}\textbf{(#1)} #2\n}
\newcommand{\ptxt}[1]{\textrm{\textnormal{#1}}}
\newcommand{\inlineeq}[1]{\n\centerline{$\displaystyle #1$}}
\newcommand{\pageline}{\noindent\rule{\textwidth}{0.1pt}}

%%%%%%%%%%%%%%%%%%%%%%%%%%%%%
% Math commands
%%%%%%%%%%%%%%%%%%%%%%%%%%%%%
% Set Theory
\newcommand{\card}[1]{\left|#1\right|}
\newcommand{\set}[1]{\left\{#1\right\}}
\newcommand{\ps}[1]{\mathcal{P}\left(#1\right)}
\newcommand{\pfinite}[1]{\mathcal{P}^{\ptxt{finite}}\left(#1\right)}
\newcommand{\naturals}{\mathbb{N}}
\newcommand{\N}{\naturals}
\newcommand{\integers}{\mathbb{Z}}
\newcommand{\Z}{\integers}
\newcommand{\rationals}{\mathbb{Q}}
\newcommand{\Q}{\rationals}
\newcommand{\reals}{\mathbb{R}}
\newcommand{\R}{\reals}
\newcommand{\complex}{\mathbb{C}}
\newcommand{\C}{\complex}
\newcommand{\comp}{^{\complement}}

% Graph Theory
\renewcommand{\deg}[1]{\ptxt{deg}\left(#1\right)}
\newcommand{\degp}[1]{\ptxt{deg}^{+}\!\!\left(#1\right)}
\newcommand{\degn}[1]{\ptxt{deg}^{-}\!\!\left(#1\right)}

% Standard Math
\newcommand{\inv}{^{-1}}
\newcommand{\abs}[1]{\left|#1\right|}
\newcommand{\ceil}[1]{\left\lceil{}#1\right\rceil}
\newcommand{\floor}[1]{\left\lfloor{}#1\right\rfloor{}}
\newcommand{\conj}[1]{\overline{#1}}
\newcommand{\of}{\circ}
\newcommand{\tri}{\triangle}
\newcommand{\inj}{\hookrightarrow}
\newcommand{\surj}{\twoheadrightarrow}
\newcommand{\mapsfrom}{\mathrel{\reflectbox{\ensuremath{\mapsto}}}}
\newcommand{\Graph}{\ptxt{Graph}\>}

% Linear Algebra
\newcommand{\Id}{\textrm{\textnormal{Id}}}
\newcommand{\im}{\textrm{\textnormal{im}}}
\newcommand{\norm}[1]{\abs{\abs{#1}}}
\newcommand{\tpose}{^{T}}
\newcommand{\iprod}[1]{\left<#1\right>}
\newcommand{\trace}{\ptxt{tr}}
\newcommand{\chgBasMat}[3]{\!\!\tensor*[_{#1}]{\left[#2\right]}{_{#3}}}
\newcommand{\vecBas}[2]{\tensor*[]{\left[#1\right]}{_{#2}}}
\newcommand{\GL}{\ptxt{GL}\>}

% Topology
\newcommand{\closure}[1]{\bar{#1}}
\newcommand{\uball}{\mathcal{U}}
\newcommand{\Int}{\ptxt{Int}\>}
\newcommand{\Ext}{\ptxt{Ext}\>}
\newcommand{\Bd}{\ptxt{Bd}\>}

% Proofs
\newcommand{\st}{s.t.}
\newcommand{\unique}{!}

% Algorithms
\algrenewcommand{\algorithmiccomment}[1]{\hskip 1em \texttt{// #1}}
\algrenewcommand\algorithmicrequire{\textbf{Input:}}
\algrenewcommand\algorithmicensure{\textbf{Output:}}

%%%%%%%%%%%%%%%%%%%%%%%%%%%%%
% Other commands
%%%%%%%%%%%%%%%%%%%%%%%%%%%%%
\newcommand{\flag}[1]{\textbf{\textcolor{red}{#1}}}

%%%%%%%%%%%%%%%%%%%%%%%%%%%%%
% Make l's curvy in math environments
%%%%%%%%%%%%%%%%%%%%%%%%%%%%%
\mathcode`l="8000
\begingroup
\makeatletter
\lccode`\~=`\l
\DeclareMathSymbol{\lsb@l}{\mathalpha}{letters}{`l}
\lowercase{\gdef~{\ifnum\the\mathgroup=\m@ne \ell \else \lsb@l \fi}}%
\endgroup

\author{Thomas Cohn}
\title{Some Directional Derivatives}
\date{9/17/18} % Can also use \today

\begin{document}
\maketitle
\setlength\RaggedRightParindent{\parindent}
\RaggedRight

\par\noindent Recall that $X$ is path-connected if $\forall\alpha,\beta\in{}X$, $\exists\varphi:[0,1]\to{}X$ continuous with $\varphi(0)=\alpha$ and $\varphi(1)=\beta$.\n

\par\noindent\textbf{Some More Special Cases}
\par\noindent\settowidth{\hangindent}{(4) }(4) Is $\GL(n,\R)$, the set of invertible $n\times{}n$ real matrices, connected?\n
No! Consider $f:\GL(n\R)\to\set{0,1}$, with $\displaystyle{}M\mapsto\frac{\frac{\det{}M}{\abs{\det{}M}}+1}{2}$. $f$ is both continuous and surjective.\n

\par\noindent (5) Exercise: $\GL_{+}(n,\R)=\set{M\in\GL(n,\R):\det{}M>0}$. Show that $\GL_{+}(n,\R)$ is path-connected.\n

\par\noindent\settowidth{\hangindent}{(6) }(6) $X\subseteq\R\leftrightarrow{}X$ is an interval or $X=\emptyset$ or $X$ is a singleton. (Note that some people consider $\emptyset$ and singletons to be an interval.)\n

\prop{For $X$ a topological space, we have $f=(f_{1},\ldots,f_{n}):X\to\R^{n}$ continuous $\leftrightarrow$ each $f_{i}$ is continuous.\n
Proof: $\Rightarrow$ Let $f_{j}=p_{j}\of{}f$ where $p_{j}:\R^{n}\to\R$ $(y_{1},\ldots,y_{n})\mapsto{}y_{j}$. The composition of two continuous functions is continuous, and $p_{j}$ is continuous, so $f_{j}$ is continuous.\n
$\Leftarrow$ Assume $X$ is a metric space. Fix $x_{0}\in{}X$, $\varepsilon>0$. There is $\delta_{j}>0$ \st{} $\abs{f_{j}(x)-f_{j}(x_{0})}\le\frac{\varepsilon}{\sqrt{n}}$ for $1\le{}j\le{}n$, when $d(x,x_{0})<\delta_{j}$. Then let $\delta=\min\set{\delta_{1},\ldots,\delta_{n}}$.\n
$d(f(x),f(x_{0}))=\norm{f(x_{0})-f(x)}\le\sqrt{n}\abs{f(x)-f(x_{0})}_{\ptxt{sup norm}}\le\sqrt{n}\frac{\varepsilon}{\sqrt{n}}=\varepsilon$ when $d(x,x_{0})<\delta$.\n
\proven}

\par\noindent This is not true the other way!\n

\ex{$f:\R^{2}\to\R$\n
$(x_{1},x_{2})\mapsto\left\{\begin{array}{lll}\frac{x_{1}x_{2}}{x_{1}^{2}+x_{2}^{2}} & \quad & (x_{1},x_{2})\ne(0,0)\\ 0 & \quad & (x_{1},x_{2})=(0,0)\end{array}\right.$\n
This is a continuous function of $x_{1}$ if $x_{2}$ is fixed.\n
This is a continuous function of $x_{2}$ if $x_{1}$ is fixed.\n
But $f$ is not continuous on $\R^{2}$ since $f(\frac{1}{n},\frac{1}{n})=\frac{1}{2}\to\frac{1}{2}\ne{}f(0,0)$. So $f$ is not sequentially continuous, and we're in a metric space, so $f$ is not continuous.}

\par\noindent Now, we present a potential paradox. Consider $f:\R\to\R$.\n
We say $f$ is continuous at $a$ $\leftrightarrow$ the graph of $f$ is ``almost horizontal'' when magnified.\n
We say $f$ is differentiable at $a$ $\leftrightarrow$ the graph of $f$ is ``almost affine'' (and not vertical) when magnified.\n
But continuous at $a$ $\not\to$ differentiable at $a$.\n

\par\noindent Try this in a vector space $V$.\n
$f:V\to{}V$, $x\mapsto\lambda{}x$. This is the dilation centered at $\vec{0}$.\n
The dilation centered at $\vec{p}$:\n
$\displaystyle\begin{array}{ccccccc}V & \to & V & \to & V & \to & V\\
\vec{x} & \mapsto & \vec{x}-\vec{p} & \mapsto & \lambda(\vec{x}-\vec{p}) & \mapsto & \lambda(\vec{x}-\vec{p})+\vec{p}\end{array}$.\n
Put concisely, $\vec{x}\mapsto\lambda\vec{x}+(1-\lambda)\vec{p}$\n

\par\noindent Given $f:V\to{}W$ with $V,W$ vector spaces over $\R$ (or perhaps $\C$), we define $\Graph{}f=\set{(\vec{x},f(\vec{x})\in{}V\times{}W:\vec{x}\in{}V}$.\n

\par\noindent Dilation about $(\vec{a},f(\vec{a}))$ is $(\vec{x},f(\vec{x}))\mapsto(\lambda(\vec{x}-\vec{a})+\vec{a},\lambda(f(\vec{x})-f(\vec{a}))+f(\vec{a}))$.\n

\par\noindent Set $t=\frac{1}{\lambda}$, $\vec{u}=\frac{\vec{x}-\vec{a}}{t}$. So $\vec{x}=\vec{a}+t\vec{u}$. So the dilated grpah now looks like\n
$\displaystyle\set{\left(\vec{a}+\vec{u},f(\vec{a})\frac{f(\vec{a}+t\vec{u})-f(\vec{a})}{t}\right):\vec{u}\in{}V}$\n

\par\noindent With $t\to{}0$ (i.e. $\lambda\to\infty$), we want $f(\vec{a})\frac{f(\vec{a}+t\vec{u})-f(\vec{u})}{t}$ to be an affine function of $\vec{a}+\vec{u}$.\n
i.e. a linear function of $(\vec{a}+\vec{u})$ plus a constant.\n
i.e. a linear function of $\vec{u}$ plus some other constant.\n
i.e. $T(\vec{u})+\vec{b}$ with $\vec{b}=f(\vec{a})$.\n

\defn{This reduces to $\displaystyle\lim_{t\to{}0}\frac{f(\vec{a}+t\vec{u})-f(\vec{a})}{t}=f'(\vec{a};\vec{u})$, the \underline{directional derivative} of $f$ at $\vec{a}$ in direction $\vec{u}$.}

\par\noindent We could try to make this theorem the core definition for multivariable differential calculus, but we won't!\n
(1) Munkres \S{}5 EX 2: all $f'(\vec{a};\vec{u})$ with fixed $\vec{a}$ exist but not linear in $\vec{u}$.\n
(2) $f(x,y)=\left\{\begin{array}{lll}x^{3}/y & \quad & y\ne{}0\\ 0 & \quad & y=0\end{array}\right.$ $f'(\vec{0};\vec{v})=\vec{0}$ $\forall\vec{u}$. But $f(\frac{1}{n},\frac{1}{n}u)\not\to{}0$ as $n\to\infty$. So $f$ is not continuous. Does that mean differentiability $\not\to$ continuity? No, we just need a stronger definition of differentiable.\n
(3) The Chain Rule (Munkres \S{}7) will fail without a stronger assumption.\n

\par\noindent Something easier: vector-valued functions of a scalar.\n
$f:I\subset\R\to{}W$ where $I$ is an open interval and $W$ is a vector space.\n
$f'(x)=\lim_{t\to\infty}\frac{f(x+t)-f(x)}{t}$, but we need a topology on $W$.\n
Choose $W$ to be a normed vector space, thus giving us a topology.\n

\par\noindent Fact: $\dim{}W<\infty\to$ all norms on $W$ induce the same topology.

\end{document}