\documentclass[10pt,letterpaper]{article}
\usepackage[utf8]{inputenc}
\usepackage[intlimits]{amsmath}
\usepackage{amsfonts}
\usepackage{amssymb}
\usepackage{ragged2e}
\usepackage[letterpaper, margin=1in]{geometry}
\usepackage{graphicx}
\usepackage{cancel}
\usepackage{mathtools}
\usepackage{tabularx}
\usepackage{arydshln}
\usepackage{tensor}
\usepackage{array}
\usepackage{xcolor}
\usepackage[boxed]{algorithm}
\usepackage[noend]{algpseudocode}
\usepackage{listings}
\usepackage{textcomp}
\usepackage[pdf,tmpdir,singlefile]{graphviz}
\usepackage{mathrsfs}
\usepackage{bbm}
\usepackage{tikz}
\usepackage{enumitem}
\usepackage{arydshln}

%%%%%%%%%%%%%%%%%%%%%%%%%%%%%
% Formatting commands
%%%%%%%%%%%%%%%%%%%%%%%%%%%%%
\newcommand{\n}{\hfill\break}
\newcommand{\up}{\vspace{-\baselineskip}}
\newcommand{\lemma}[1]{\par\noindent\settowidth{\hangindent}{\textbf{Lemma: }}\textbf{Lemma: }#1}
\newcommand{\defn}[1]{\par\noindent\settowidth{\hangindent}{\textbf{Defn: }}\textbf{Defn: }#1\n}
\newcommand{\thm}[1]{\par\noindent\settowidth{\hangindent}{\textbf{Thm: }}\textbf{Thm: }#1\n}
\newcommand{\prop}[1]{\par\noindent\settowidth{\hangindent}{\textbf{Prop: }}\textbf{Prop: }#1\n}
\newcommand{\cor}[1]{\par\noindent\settowidth{\hangindent}{\textbf{Cor: }}\textbf{Cor: }#1\n}
\newcommand{\ex}[1]{\par\noindent\settowidth{\hangindent}{\textbf{Ex: }}\textbf{Ex: }#1\n}
\newcommand{\proven}{\;$\square$\n}
\newcommand{\problem}[1]{\par\noindent{#1}\n}
\newcommand{\problempart}[2]{\par\noindent\indent{}\settowidth{\hangindent}{\textbf{(#1)} \indent{}}\textbf{(#1)} #2\n}
\newcommand{\ptxt}[1]{\textrm{\textnormal{#1}}}
\newcommand{\inlineeq}[1]{\centerline{$\displaystyle #1$}}
\newcommand{\pageline}{\noindent\rule{\textwidth}{0.1pt}}

%%%%%%%%%%%%%%%%%%%%%%%%%%%%%
% Math commands
%%%%%%%%%%%%%%%%%%%%%%%%%%%%%
% Set Theory
\newcommand{\card}[1]{\left|#1\right|}
\newcommand{\set}[1]{\left\{#1\right\}}
\newcommand{\ps}[1]{\mathcal{P}\left(#1\right)}
\newcommand{\pfinite}[1]{\mathcal{P}^{\ptxt{finite}}\left(#1\right)}
\newcommand{\naturals}{\mathbb{N}}
\newcommand{\N}{\naturals}
\newcommand{\integers}{\mathbb{Z}}
\newcommand{\Z}{\integers}
\newcommand{\rationals}{\mathbb{Q}}
\newcommand{\Q}{\rationals}
\newcommand{\reals}{\mathbb{R}}
\newcommand{\R}{\reals}
\newcommand{\complex}{\mathbb{C}}
\newcommand{\C}{\complex}
\newcommand{\comp}{^{\complement}}
\DeclareMathOperator{\Hom}{Hom}
\newcommand{\Ind}{\mathbbm{1}}
\newcommand{\cut}{\setminus}

% Graph Theory
\let\deg\relax
\DeclareMathOperator{\deg}{deg}
\newcommand{\degp}{\ptxt{deg}^{+}}
\newcommand{\degn}{\ptxt{deg}^{-}}
\newcommand{\precdot}{\mathrel{\ooalign{$\prec$\cr\hidewidth\hbox{$\cdot\mkern0.5mu$}\cr}}}
\newcommand{\succdot}{\mathrel{\ooalign{$\cdot\mkern0.5mu$\cr\hidewidth\hbox{$\succ$}\cr\phantom{$\succ$}}}}
\DeclareMathOperator{\cl}{cl}
\DeclareMathOperator{\affdim}{affdim}

% Probability
\newcommand{\Prob}{\mathbb{P}}
\newcommand{\Avg}{\mathbb{E}}

% Standard Math
\newcommand{\inv}{^{-1}}
\newcommand{\abs}[1]{\left|#1\right|}
\newcommand{\ceil}[1]{\left\lceil{}#1\right\rceil{}}
\newcommand{\floor}[1]{\left\lfloor{}#1\right\rfloor{}}
\newcommand{\conj}[1]{\overline{#1}}
\newcommand{\of}{\circ}
\newcommand{\tri}{\triangle}
\newcommand{\inj}{\hookrightarrow}
\newcommand{\surj}{\twoheadrightarrow}
\newcommand{\mapsfrom}{\mathrel{\reflectbox{\ensuremath{\mapsto}}}}
\newcommand{\ndiv}{\nmid}
\renewcommand{\epsilon}{\varepsilon}
\newcommand{\divides}{\mid}
\newcommand{\ndivides}{\nmid}
\DeclareMathOperator{\lcm}{lcm}

% Linear Algebra
\newcommand{\Id}{\textrm{\textnormal{Id}}}
\newcommand{\im}{\textrm{\textnormal{im}}}
\newcommand{\norm}[1]{\abs{\abs{#1}}}
\newcommand{\tpose}{^{T}}
\newcommand{\iprod}[1]{\left<#1\right>}
\DeclareMathOperator{\trace}{tr}
\newcommand{\chgBasMat}[3]{\!\!\tensor*[_{#1}]{\left[#2\right]}{_{#3}}}
\newcommand{\vecBas}[2]{\tensor*[]{\left[#1\right]}{_{#2}}}
\DeclareMathOperator{\GL}{GL}
\DeclareMathOperator{\Mat}{Mat}
\DeclareMathOperator{\vspan}{span}
\DeclareMathOperator{\rank}{rank}
\newcommand{\V}[1]{\vec{#1}}

% Topology
\newcommand{\closure}[1]{\overline{#1}}
\newcommand{\uball}{\mathcal{U}}
\DeclareMathOperator{\Int}{Int}
\DeclareMathOperator{\Ext}{Ext}
\DeclareMathOperator{\Bd}{Bd}
\DeclareMathOperator{\rInt}{rInt}
\DeclareMathOperator{\ch}{ch}
\DeclareMathOperator{\ah}{ah}

% Analysis
\DeclareMathOperator{\Graph}{Graph}
\DeclareMathOperator{\epi}{epi}
\DeclareMathOperator{\hypo}{hypo}
\DeclareMathOperator{\supp}{supp}
\newcommand{\lint}[2]{\underset{#1}{\overset{#2}{{\color{black}\underline{{\color{white}\overline{{\color{black}\int}}\color{black}}}}}}}
\newcommand{\uint}[2]{\underset{#1}{\overset{#2}{{\color{white}\underline{{\color{black}\overline{{\color{black}\int}}\color{black}}}}}}}
\newcommand{\alignint}[2]{\underset{#1}{\overset{#2}{{\color{white}\underline{{\color{white}\overline{{\color{black}\int}}\color{black}}}}}}}
\newcommand{\extint}{\ptxt{ext}\int}
\newcommand{\extalignint}[2]{\ptxt{ext}\alignint{#1}{#2}}
\newcommand{\conv}{\ast}

% Proofs
\newcommand{\st}{s.t.}
\newcommand{\unique}{!}

% Brackets
\newcommand{\paren}[1]{\left(#1\right)}
\renewcommand{\brack}[1]{\left[#1\right]}
\renewcommand{\brace}[1]{\left\{#1\right\}}
\newcommand{\ang}[1]{\left<#1\right>}

% Algorithms
\algrenewcommand{\algorithmiccomment}[1]{\hskip 1em \texttt{// #1}}
\algrenewcommand\algorithmicrequire{\textbf{Input:}}
\algrenewcommand\algorithmicensure{\textbf{Output:}}
\newcommand{\parSymbol}{\P}
\renewcommand{\P}{\ptxt{\textbf{P}}}
\newcommand{\NP}{\ptxt{\textbf{NP}}}
\newcommand{\NPC}{\ptxt{\textbf{NP-Complete}}}
\newcommand{\NPH}{\ptxt{\textbf{NP-Hard}}}
\newcommand{\EXP}{\ptxt{\textbf{EXP}}}

%%%%%%%%%%%%%%%%%%%%%%%%%%%%%
% Other commands
%%%%%%%%%%%%%%%%%%%%%%%%%%%%%
\newcommand{\flag}[1]{\textbf{\textcolor{red}{#1}}}

%%%%%%%%%%%%%%%%%%%%%%%%%%%%%
% Make l's curvy in math environments
%%%%%%%%%%%%%%%%%%%%%%%%%%%%%
\mathcode`l="8000
\begingroup
\makeatletter
\lccode`\~=`\l
\DeclareMathSymbol{\lsb@l}{\mathalpha}{letters}{`l}
\lowercase{\gdef~{\ifnum\the\mathgroup=\m@ne \ell \else \lsb@l \fi}}%
\endgroup

\newcommand{\B}{
    \begin{tikzpicture}
    \filldraw [fill=red, draw=black] (0, 0) rectangle (0.37, 0.45);
    \draw [line width=0.5mm, white ] (0.1,0.08) -- (0.1,0.38);
    \draw[line width=0.5mm, white ] (0.1, 0.35) .. controls (0.2, 0.35) and (0.4, 0.2625) .. (0.1, 0.225);
    \draw[line width=0.5mm, white ] (0.1, 0.225) .. controls (0.2, 0.225) and (0.4, 0.1625) .. (0.1, 0.1);
    \end{tikzpicture}
}

\author{Thomas Cohn}
\title{Change of Variable Theorem}
\date{11/12/18} % Can also use \today

\begin{document}
\maketitle
\setlength\RaggedRightParindent{\parindent}
\RaggedRight

\thm{(Change of Variables Theorem) Given $A^{\ptxt{osso}\R^{n}}\xrightarrow[C^{1}\ptxt{-diffeo}]{g}B^{\ptxt{osso}\R^{n}}\xrightarrow[\ptxt{cts}]{f}\R$, then $(\ptxt{ext})\int_{B}f\overset{\ptxt{def}}{=}(\ptxt{ext})\int_{A}(f\of{}g)\abs{\det{}Dg}$.}

\par\noindent Special Case: $g:\vec{x}\mapsto{}r\vec{x}$ for $r>0$. Then $(\ptxt{ext})\int_{\vec{x}\in{}B}f(\vec{x})=(\ptxt{ext})\int_{\vec{x}\in{}A}f(r\vec{x})r^{n}$.\n

\ex{Let $B^{n}(r)=\set{\vec{x}\in\R^{n}:\norm{\vec{x}}<r}=U(\vec{0},r)$.\n
Then $v(B^{n}(r))=r^{n}v(B^{n}(1))=r^{n}\lambda_{n}$.}

\par\noindent Goal: compute $\lambda_{n}$. Guess the behavior of $\frac{v(B^{n}(1)}{v([-1,1]^{n})}$ for large $n$.
\[
\lambda_{n}=\int_{B^{n}(1)}1=\int_{\vec{x}\in{}B^{k}(1)}\paren{\;\int_{\vec{y}\in{}B^{n-k}(\sqrt{1-\norm{\vec{x}}^{2}})}1}=\int_{\vec{x}\in{}B^{k}(1)}\paren{1-\norm{\vec{x}}^{2}}^{\frac{n-k}{2}}\lambda_{n-k}
\]

\par\noindent This formula becomes the nicest for $k=2$. So
\[
\lambda_{n}=\lambda_{n-2}\int_{B^{2}(1)}\paren{1-\norm{\vec{x}}^{2}}^{\frac{n}{2}-1}=\lambda_{n-2}\int_{B^{2}(1)\cut((-1,0]\times\set{0})}\paren{1-\norm{\vec{x}}^{2}}^{\frac{n}{2}-1}+\lambda_{n-2}\underbrace{\int_{(-1,0]\times\set{0}}\paren{1-\norm{\vec{x}}^{2}}^{\frac{n}{2}-1}}_{=0\ptxt{, because its over a set of measure $0$}}
\]

\par\noindent Thus,
\begin{align*}
\lambda_{n}=\lambda_{n-2}\int_{B^{2}(1)\cut((-1,0]\times\set{0})}\paren{1-\norm{\vec{x}}^{2}}^{\frac{n}{2}-1} & =\lambda_{n-2}\int_{\begin{subarray}{c}0<r<1\\ -\pi<\theta<\pi\end{subarray}}\paren{1-r^{2}}^{\frac{n}{2}-1}r\\
 & =\lambda_{n-2}\int_{0}^{1}\int_{-\pi}^{\pi}\paren{1-r^{2}}^{\frac{n}{2}-1}rd\theta{}dr\\
 & =\lambda_{n-2}\int_{0}^{1}2\pi{}r\paren{1-r^{2}}^{\frac{n}{2}-1}dr\qquad\qquad\begin{array}{c}u=1-r^{2}\\ du=-2rdr\end{array}\\
 & =\lambda_{n-2}\int_{1}^{0}\pi{}u^{\frac{n}{2}-1}du\\
 & =\pi\lambda{n-2}\int_{0}^{1}u^{\frac{n}{2}-1}du\\
 & =\pi\lambda_{n-2}\brack{\frac{u^{\frac{n}{2}}}{\frac{n}{2}}}_{u=0}^{u=1}\\
 & =\pi\lambda_{n-2}\cdot\frac{1}{\frac{n}{2}}\\
 & =\frac{2\pi\lambda_{n-2}}{n}
\end{align*}

\par\noindent $\lambda_{2}\overset{\ptxt{def}}{=}\pi$: the area of the unit circle. So $\lambda_{4}=\frac{\pi^{2}}{2}$, $\lambda_{6}=\frac{\pi^{3}}{6}$, $\lambda_{2n}=\frac{\pi^{n}}{n!}$.\n
$\lambda_{1}\overset{\ptxt{def}}{=}2$: the length of the unit interval $(-1,1)$. So $\lambda_{3}=\frac{4\pi}{3}$, $\lambda_{5}=\frac{8\pi^{2}}{15}$, $\lambda_{2n+1}=\frac{2^{n+1}\pi^{n}}{(2n+1)(2n-1)\cdots{}3\cdot{}1}$

\par\noindent So $\displaystyle\lambda_{n}=\left\{\begin{array}{ll}\frac{\pi^{n/2}}{(n/2)!} & \quad{}2\divides{}n\\ \frac{2^{\frac{n+1}{2}}\pi^{\frac{n-1}{2}}}{n\cdot(n-2)\cdot\cdots\cdot{}3\cdot{}1} & \quad{}2\ndiv{}n\end{array}\right.$\n

\par\noindent $\lambda_{n}=v(B^{n}(1))$\n
$\mu_{n}=v([-1,1]^{n})=v\paren{\set{\vec{x}\in\R^{n}:\norm{\vec{x}}_{\sup}<1}}=2^{n}$\n
$K_{n}=v\paren{\set{\vec{x}\in\R^{n}:\abs{x_{1}}+\abs{x_{2}}+\cdots+\abs{x_{n}}<1}}=\frac{2^{n}}{n!}$. Note that $\abs{x_{1}}+\cdots+\abs{x_{n}}\overset{\ptxt{def}}{=}\norm{\vec{x}}_{1}$.\n

\par\noindent $K_{n}\le\lambda_{n}\le\mu_{n}$. Check that $\frac{\lambda_{n}}{\mu_{n}}\to{}0$, and that strict inequalities for $n>1$.\n

\par\noindent Proof the Change of Variables Theorem:\n
(With the temporarily added assumptions $A,B$ bounded and rectifiable, $g$ a diffeomorphism from a neighborhood of $\closure{A}$ to a neighborhood of $\closure{B}$)\n

\par\noindent Special case 1: $g:\paren{\begin{array}{c}x_{1}\\ \vdots\\ x_{j}\\ \vdots\\ x_{k}\\ \vdots\\ x_{n}\end{array}}\mapsto\paren{\begin{array}{c}x_{1}\\ \vdots\\ x_{k}\\ \vdots\\ x_{j}\\ \vdots\\ x_{n}\end{array}}$ coordinate transposition. $\deg{}Dg=-1$, so $\abs{\det{}Dg}=1$.\n
Thus, $\displaystyle\int_{B}f=\int_{A}f\of{}g$. This works because reorienting boxes doesn't change their volume.\n

\par\noindent Special case 2: $E^{\ptxt{bdd},\ptxt{open},\ptxt{rect}}\subset\R^{n+1}$, $\varphi,\psi\in{}C(\closure{E},\R)$, $\varphi<\psi$ on $E$,\n
$B=\set{\vec{x}\in\R^{n}:\paren{\begin{array}{c}x_{1}\\ \vdots\\ x_{n-1}\end{array}},\varphi\paren{\begin{array}{c}x_{1}\\ \vdots\\ x_{n-1}\end{array}}<x_{n}<\psi\paren{\begin{array}{c}x_{1}\\ \vdots\\ x_{n-1}\end{array}}}$, and $g^{\ptxt{diffeo}}:\paren{\begin{array}{c}x_{1}\\ \vdots\\ x_{n-1}\\ x_{n}\end{array}}\mapsto\paren{\begin{array}{c}x_{1}\\ \vdots\\ x_{n-1}\\ \alpha(\vec{x})\end{array}}$.\n
Then
\begin{enumerate}[label=(\alph*)]
	\item $A$ and $B$ are rectifiable (by study ex 5, HW9, or lemma 14.3)
	\item For fixed $\paren{\begin{array}{c}x_{1}\\ \vdots\\ x_{n}\end{array}}$, we have $\vec{x}\in{}A\Leftrightarrow{}x_{n}\in{}I$ where $I$ is an interval determined by $\paren{\begin{array}{c}x_{1}\\ \vdots\\ x_{n-1}\end{array}}$.
	\item $\det{}Dg=D_{n}\alpha$, so $\abs{\deg{}Dg}=\abs{\det{}D_{n}\alpha}$.
\end{enumerate}

\par\noindent Thus
\[
\int_{B}f=\int_{\paren{\begin{subarray}{c}x_{1}\\ \vdots\\ x_{n-1}\end{subarray}}\in{}E}\paren{\int_{x_{n}\in\paren{\varphi\paren{\begin{subarray}{c}x_{1}\\ \vdots\\ x_{n-1}\end{subarray}},\psi\paren{\begin{subarray}{c}x_{1}\\ \vdots\\ x_{n-1}\end{subarray}}}}f}=\int_{\paren{\begin{subarray}{c}x_{1}\\ \vdots\\ x_{n-1}\end{subarray}}\in{}E}\paren{\;\int_{\alpha(\vec{x})\in(\cdots,\cdots)}(f\of{}g)\cdot\abs{D_{n}\alpha}}=\int_{A}(f\of{}g)\det{}Dg
\]

\prop{Given $A\xrightarrow[\ptxt{diffeo}]{g}B\xrightarrow[\ptxt{diffeo}]{h}C\xrightarrow[\ptxt{cts}]{f}\R$, Then the COVT holds for $g$ and $h$.\n
Proof:\n
$\extint_{C}f=\extint_{B}f\of{}h\abs{\det{}Dh}=\extint_{A}((f\of{}h)\of{}g)\abs{\det{}Dh}\cdot{}g\abs{\det{}Dg}=\extint_{A}f\of{}h\of{}g\abs{\det{}D(h\of{}g)}$\n
\proven}

\par\noindent Strategy: Factor general diffeomorphic maps into composition of maps of types (1) and (2).\n
Good news: we can do this!\n
Bad news: good news is only local.

\end{document}