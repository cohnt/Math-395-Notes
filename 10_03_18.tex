\documentclass[10pt,letterpaper]{article}
\usepackage[utf8]{inputenc}
\usepackage{amsmath}
\usepackage{amsfonts}
\usepackage{amssymb}
\usepackage{ragged2e}
\usepackage[letterpaper, margin=1in]{geometry}
\usepackage{graphicx}
\usepackage{cancel}
\usepackage{mathtools}
\usepackage{tabularx}
\usepackage{arydshln}
\usepackage{tensor}
\usepackage{array}
\usepackage{xcolor}
\usepackage[boxed]{algorithm}
\usepackage[noend]{algpseudocode}
\usepackage{listings}
\usepackage{textcomp}
\usepackage[pdf,tmpdir,singlefile]{graphviz}

%%%%%%%%%%%%%%%%%%%%%%%%%%%%%
% Formatting commands
%%%%%%%%%%%%%%%%%%%%%%%%%%%%%
\newcommand{\n}{\hfill\break}
\newcommand{\lemma}[1]{\par\noindent\settowidth{\hangindent}{\textbf{Lemma: }}\textbf{Lemma: }#1\n}
\newcommand{\defn}[1]{\par\noindent\settowidth{\hangindent}{\textbf{Defn: }}\textbf{Defn: }#1\n}
\newcommand{\thm}[1]{\par\noindent\settowidth{\hangindent}{\textbf{Thm: }}\textbf{Thm: }#1\n}
\newcommand{\prop}[1]{\par\noindent\settowidth{\hangindent}{\textbf{Prop: }}\textbf{Prop: }#1\n}
\newcommand{\cor}[1]{\par\noindent\settowidth{\hangindent}{\textbf{Cor: }}\textbf{Cor: }#1\n}
\newcommand{\ex}[1]{\par\noindent\settowidth{\hangindent}{\textbf{Ex: }}\textbf{Ex: }#1\n}
\newcommand{\proven}{\;$\square$\n}
\newcommand{\problem}[1]{\par\noindent{#1}\n}
\newcommand{\problempart}[2]{\par\noindent\indent{}\settowidth{\hangindent}{\textbf{(#1)} \indent{}}\textbf{(#1)} #2\n}
\newcommand{\ptxt}[1]{\textrm{\textnormal{#1}}}
\newcommand{\inlineeq}[1]{\centerline{$\displaystyle #1$}}
\newcommand{\pageline}{\noindent\rule{\textwidth}{0.1pt}}

%%%%%%%%%%%%%%%%%%%%%%%%%%%%%
% Math commands
%%%%%%%%%%%%%%%%%%%%%%%%%%%%%
% Set Theory
\newcommand{\card}[1]{\left|#1\right|}
\newcommand{\set}[1]{\left\{#1\right\}}
\newcommand{\ps}[1]{\mathcal{P}\left(#1\right)}
\newcommand{\pfinite}[1]{\mathcal{P}^{\ptxt{finite}}\left(#1\right)}
\newcommand{\naturals}{\mathbb{N}}
\newcommand{\N}{\naturals}
\newcommand{\integers}{\mathbb{Z}}
\newcommand{\Z}{\integers}
\newcommand{\rationals}{\mathbb{Q}}
\newcommand{\Q}{\rationals}
\newcommand{\reals}{\mathbb{R}}
\newcommand{\R}{\reals}
\newcommand{\complex}{\mathbb{C}}
\newcommand{\C}{\complex}
\newcommand{\comp}{^{\complement}}
\newcommand{\Hom}{\ptxt{Hom}\>}

% Graph Theory
\renewcommand{\deg}[1]{\ptxt{deg}\left(#1\right)}
\newcommand{\degp}[1]{\ptxt{deg}^{+}\!\!\left(#1\right)}
\newcommand{\degn}[1]{\ptxt{deg}^{-}\!\!\left(#1\right)}

% Standard Math
\newcommand{\inv}{^{-1}}
\newcommand{\abs}[1]{\left|#1\right|}
\newcommand{\ceil}[1]{\left\lceil{}#1\right\rceil}
\newcommand{\floor}[1]{\left\lfloor{}#1\right\rfloor{}}
\newcommand{\conj}[1]{\overline{#1}}
\newcommand{\of}{\circ}
\newcommand{\tri}{\triangle}
\newcommand{\inj}{\hookrightarrow}
\newcommand{\surj}{\twoheadrightarrow}
\newcommand{\mapsfrom}{\mathrel{\reflectbox{\ensuremath{\mapsto}}}}
\newcommand{\Graph}{\ptxt{Graph}\>}
\newcommand{\ndiv}{\nmid}
\renewcommand{\epsilon}{\varepsilon}

% Linear Algebra
\newcommand{\Id}{\textrm{\textnormal{Id}}}
\newcommand{\im}{\textrm{\textnormal{im}}}
\newcommand{\norm}[1]{\abs{\abs{#1}}}
\newcommand{\tpose}{^{T}}
\newcommand{\iprod}[1]{\left<#1\right>}
\newcommand{\trace}{\ptxt{tr}}
\newcommand{\chgBasMat}[3]{\!\!\tensor*[_{#1}]{\left[#2\right]}{_{#3}}}
\newcommand{\vecBas}[2]{\tensor*[]{\left[#1\right]}{_{#2}}}
\newcommand{\GL}{\ptxt{GL}\>}
\newcommand{\Mat}{\ptxt{Mat}\>}

% Topology
\newcommand{\closure}[1]{\overline{#1}}
\newcommand{\uball}{\mathcal{U}}
\newcommand{\Int}{\ptxt{Int}\>}
\newcommand{\Ext}{\ptxt{Ext}\>}
\newcommand{\Bd}{\ptxt{Bd}\>}

% Proofs
\newcommand{\st}{s.t.}
\newcommand{\unique}{!}

% Algorithms
\algrenewcommand{\algorithmiccomment}[1]{\hskip 1em \texttt{// #1}}
\algrenewcommand\algorithmicrequire{\textbf{Input:}}
\algrenewcommand\algorithmicensure{\textbf{Output:}}

%%%%%%%%%%%%%%%%%%%%%%%%%%%%%
% Other commands
%%%%%%%%%%%%%%%%%%%%%%%%%%%%%
\newcommand{\flag}[1]{\textbf{\textcolor{red}{#1}}}

%%%%%%%%%%%%%%%%%%%%%%%%%%%%%
% Make l's curvy in math environments
%%%%%%%%%%%%%%%%%%%%%%%%%%%%%
\mathcode`l="8000
\begingroup
\makeatletter
\lccode`\~=`\l
\DeclareMathSymbol{\lsb@l}{\mathalpha}{letters}{`l}
\lowercase{\gdef~{\ifnum\the\mathgroup=\m@ne \ell \else \lsb@l \fi}}%
\endgroup

\author{Professor David Barrett\\ \small\textit{Transcribed by Thomas Cohn}}
\title{Contraction Mapping Theorem}
\date{10/3/18} % Can also use \today

\begin{document}
\maketitle
\setlength\RaggedRightParindent{\parindent}
\RaggedRight

\par\noindent $f:X^{\ptxt{complete metric space}}\to{}X$ $\Rightarrow$ $f$ has a unique fixed point.\n

\par\noindent We define $T_{\vec{y}}:\vec{x}\mapsto\vec{x}+\vec{y}$.\n

\par\noindent From Monday:
\lemma{Given $\vec{0}\in\mathcal{U}^{\ptxt{open}}\subset\R^{n}$\n
\phantom{Given }$g:\mathcal{U}\to\R^{n}$ is at least $C^{1}$\n
\phantom{Given }$g(\vec{0})=\vec{0}$\n
\phantom{Given }$Dg(\vec{0})=\Id$\n
\phantom{Given }$0<\epsilon<1$\n
Then $\exists\delta>0$ \st{} $h=g-\Id$ satisfies (1) $\norm{h(\vec{y})-h(\vec{a})}\le\epsilon\norm{\vec{y}-\vec{x}}$ for $\vec{x},\vec{y}\in\uball(\vec{0},\delta)$\n
\phantom{Then $\exists\delta>0$ \st{} $h=g-\Id$ satisfies} (2) $(1-\epsilon)\norm{\vec{y}-\vec{x}}\le\norm{g(\vec{y})-g(\vec{x})}\le(1+\epsilon)\norm{\vec{y}-\vec{x}}$\n
\phantom{Then $\exists\delta>0$ \st{} $h=g-\Id$ satisfies} (3) $g$ is injective on $\uball(\vec{0},\delta)$.}

\par\noindent Let $f:A^{\ptxt{t.s.}}\to{}B^{\ptxt{t.s.}}$.
\defn{$f$ is \underline{continuous} $\leftrightarrow$ $f\inv(\uball)$ open in $A$ when $\uball$ open in $B$.\n
$f$ is \underline{open} $\leftrightarrow$ $f(\uball)$ open in $B$ when $\uball$ open in $A$.}

\par\noindent Suppose $f$ is a bijection.\n
Then $f$ is continuous $\leftrightarrow$ $f\inv$ is open\n
\phantom{Then }$f$ is open $\leftrightarrow$ $f\inv$ is continuous\n
\phantom{Then }$f$ is a homeomorphism $\leftrightarrow$ $f$ is open and continuous.\n

\par\noindent Let $\psi_{\vec{y}}:\vec{x}\mapsto\vec{y}-h(\vec{x})$ ($\vec{x}\in\uball$). Pick $0<\widetilde{\delta}<\delta$.\n
Then $\vec{y}\in\uball(\vec{0},(1-\epsilon)\widetilde{\delta})$, $\vec{x}\in\closure{\uball(\vec{0},\widetilde{\delta})}$ $\Rightarrow$ $\norm{\psi_{\vec{y}}(\vec{x})}\le\norm{\vec{y}}+\norm{h(\vec{x})}\le(1-\epsilon)\widetilde{\delta}+\epsilon\widetilde{\delta}=\widetilde{\delta}$.\n
So $\psi_{\vec{y}}:\closure{\uball(\vec{0},\widetilde{\delta})}\to\closure{\uball(\vec{0},\widetilde{\delta})}$ with $\norm{\psi_{\vec{y}}(\vec{x_{1}})-\psi_{\vec{y}}(\vec{x_{2}})}=\norm{h(\vec{x_{2}})-h(\vec{x_{1}})}\le\epsilon\norm{\vec{x_{2}}-\vec{x_{1}}}$.\n

\par\noindent Therefore, $\psi_{\vec{y}}:\closure{\uball(\vec{0},\widetilde{\delta})}\to\closure{\uball(\vec{0},\widetilde{\delta})}$ is a contraction. (We use the closure because $\uball(\vec{0},\widetilde{\delta})$ is not a complete metric space, but $\closure{\uball(\vec{0},\widetilde{\delta})}$ is.)\n
And so, by the contraction mapping theorem, $\exists\vec{x}\in\closure{\uball(\vec{0},\widetilde{\delta})}$ \st{} $\psi_{\vec{y}}(\vec{x})=\vec{x}$. But $\psi_{\vec{y}}(\vec{x})=\vec{y}-h(\vec{x})=\vec{y}-g(\vec{x})+\vec{x}$. So $\vec{y}-g(\vec{x})+\vec{x}=\vec{x}$. So $\vec{y}=g(\vec{x})$. And thus, $g(\uball)\supset{}g(\closure{\uball(\vec{0},\widetilde{\delta})})\supset\uball(\vec{0},(1-\epsilon)\widetilde{\delta})$.\n

\ex{Upgrade to $g(\uball(\vec{0},\delta))\supset\uball(\vec{0},(1-\epsilon)\delta)$.}

\par\noindent Conclude: Add $g(\uball)]\supset{}g(\vec{0},(1-\epsilon)\delta)$ to the lemma. In particular, $\vec{0}\in\Int{}g(\vec{u})$.\n

\newpage
\thm{(Cousin of Inverse Function Theorem) Given $E^{\ptxt{open}}\subset\R^{n}$, $f\in{}C^{1}(E,\R^{n})$, and $\det{}Df\ne{}0$ on $E$, then the following are true:
\begin{enumerate}
	\item $\vec{a}\in{}E\to{}f(\vec{a})\in\Int{}f[E]$.
	\item $f[E]$ is open in $\R^{n}$.
	\item $f:E\to{}f[E]$ is an open map.
\end{enumerate}}

\cor{$f$ as above and injective $\to$ $f:E\to{}f[E]$, is a homeomorphism.}

\par\noindent Proof:\n
(1) Apply lemma to $g=Df(\vec{a})\inv\of{}T_{-f(\vec{a})}\of{}f\of{}T_{\vec{a}}$. We get $\vec{0}\in\Int(\im(g))$. $\vec{0}=Df(\vec{a})(\vec{0})\in\im(T_{-\vec{a}}\of{}f\cancel{\of{}T_{\vec{a}}})$. We ignore $T_{\vec{a}}$ because it's bijective.\n
Apply $T_{f(\vec{a})}$. Then $f(\vec{a})\in\Int(\im(f))$, so $f(\vec{a})\in\Int(f[E])$.\n
(2) Since all $f(\vec{a})\in\Int(f[E])$, $f[E]$ is open.\n
(3) For $\uball^{\ptxt{open}}\subset{}E$, apply (2) to $f|_{\uball}$.\n
\proven

\prop{$f$ as in cor and $f\in{}C^{r}$ $\Rightarrow$ $f\inv\in{}C^{r}$.}

\par\noindent The inverse function theorem follows from the lemma, the corollary, and the proposition.\n

\par\noindent Proof: For $r=1$, $g=f\inv$, $\vec{b}=f(\vec{a})$, $M=Df(\vec{a})$, we need $\displaystyle\frac{g(\vec{b}+\vec{k})-g(\vec{b})-M\inv\vec{k}}{\norm{\vec{k}}}\to\vec{0}$ as $\vec{k}\to\vec{0}$.\n

\par\noindent $\displaystyle\frac{g(\vec{b}+\vec{k})-g(\vec{b})-M\inv\vec{k}}{\norm{\vec{k}}}-\frac{\vec{h}-M\inv\vec{k}}{\norm{\vec{k}}}=-M\inv\left(\frac{\vec{k}-M\vec{h}}{\norm{\vec{k}}}\right)=-M\inv\left(\frac{f(\vec{a}+\vec{h})-f(\vec{a})-M\vec{h}}{\norm{\vec{h}}}\right)\left(\frac{\norm{\vec{h}}}{\norm{\vec{k}}}\right)$.\n

\par\noindent $\left(\frac{\norm{\vec{h}}}{\norm{\vec{k}}}\right)$ is bounded for $\norm{\vec{k}}<\delta$ by lemma (2). So it follows that $\vec{k}\to\vec{0}$ as $\vec{h}\to\vec{0}$. So $\frac{f(\vec{a}+\vec{h})-f(\vec{a})-M\vec{h}}{\norm{\vec{h}}}\to\vec{0}$. So $M\inv\vec{0}=\vec{0}$. Thus, as $\vec{k}\to\vec{0}$, the other thing goes to $\vec{0}$.\n

\par\noindent We've now shown that $g$ is differentiable. $Dg(\vec{b})=Df(g(\vec{b}))\inv$. Still need $Dg$ continuous.\n

\[
f[\uball]\overset{g}{\underset{\ptxt{cts}}{\to}}\uball\overset{Df}{\underset{\ptxt{cts}}{\to}}\GL(n,\R)\overset{\ptxt{inversion}}{\underset{\ptxt{Thm 2.14}}{\to}}\GL(n,\R)
\]

\par\noindent So $Dg$ is the composition of continuous maps, and is therefore continuous. So $Dg$ is continuous, and we're done for $r=1$. For $r>1$, recall that $g\in{}C^{r}\Leftrightarrow{}Dg\in{}C{r-1}$.\n

\lemma{$C^{r-1}$ mapping closed under composition.\n
Proof: Will be done on Friday.}

\par\noindent Induction on $r$. Assume that prop holds for $C^{r-1}$. Then

\[
f(\uball)\overset{g}{\underset{C^{r-1}}{\to}}\uball\overset{Df}{\underset{C^{r-1}}{\to}}\GL(n,\R)\overset{\ptxt{inversion}}{\underset{C^{\infty}}{\to}}\GL(n,\R)
\]

\par\noindent So $Dg\in{}C^{r-1}$.\proven

\end{document}