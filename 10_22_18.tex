\documentclass[10pt,letterpaper]{article}
\usepackage[utf8]{inputenc}
\usepackage{amsmath}
\usepackage{amsfonts}
\usepackage{amssymb}
\usepackage{ragged2e}
\usepackage[letterpaper, margin=1in]{geometry}
\usepackage{graphicx}
\usepackage{cancel}
\usepackage{mathtools}
\usepackage{tabularx}
\usepackage{arydshln}
\usepackage{tensor}
\usepackage{array}
\usepackage{xcolor}
\usepackage[boxed]{algorithm}
\usepackage[noend]{algpseudocode}
\usepackage{listings}
\usepackage{textcomp}
\usepackage[pdf,tmpdir,singlefile]{graphviz}
\usepackage{mathrsfs}

%%%%%%%%%%%%%%%%%%%%%%%%%%%%%
% Formatting commands
%%%%%%%%%%%%%%%%%%%%%%%%%%%%%
\newcommand{\n}{\hfill\break}
\newcommand{\lemma}[1]{\par\noindent\settowidth{\hangindent}{\textbf{Lemma: }}\textbf{Lemma: }#1}
\newcommand{\defn}[1]{\par\noindent\settowidth{\hangindent}{\textbf{Defn: }}\textbf{Defn: }#1\n}
\newcommand{\thm}[1]{\par\noindent\settowidth{\hangindent}{\textbf{Thm: }}\textbf{Thm: }#1\n}
\newcommand{\prop}[1]{\par\noindent\settowidth{\hangindent}{\textbf{Prop: }}\textbf{Prop: }#1\n}
\newcommand{\cor}[1]{\par\noindent\settowidth{\hangindent}{\textbf{Cor: }}\textbf{Cor: }#1\n}
\newcommand{\ex}[1]{\par\noindent\settowidth{\hangindent}{\textbf{Ex: }}\textbf{Ex: }#1\n}
\newcommand{\proven}{\;$\square$\n}
\newcommand{\problem}[1]{\par\noindent{#1}\n}
\newcommand{\problempart}[2]{\par\noindent\indent{}\settowidth{\hangindent}{\textbf{(#1)} \indent{}}\textbf{(#1)} #2\n}
\newcommand{\ptxt}[1]{\textrm{\textnormal{#1}}}
\newcommand{\inlineeq}[1]{\centerline{$\displaystyle #1$}}
\newcommand{\pageline}{\noindent\rule{\textwidth}{0.1pt}}

%%%%%%%%%%%%%%%%%%%%%%%%%%%%%
% Math commands
%%%%%%%%%%%%%%%%%%%%%%%%%%%%%
% Set Theory
\newcommand{\card}[1]{\left|#1\right|}
\newcommand{\set}[1]{\left\{#1\right\}}
\newcommand{\ps}[1]{\mathcal{P}\left(#1\right)}
\newcommand{\pfinite}[1]{\mathcal{P}^{\ptxt{finite}}\left(#1\right)}
\newcommand{\naturals}{\mathbb{N}}
\newcommand{\N}{\naturals}
\newcommand{\integers}{\mathbb{Z}}
\newcommand{\Z}{\integers}
\newcommand{\rationals}{\mathbb{Q}}
\newcommand{\Q}{\rationals}
\newcommand{\reals}{\mathbb{R}}
\newcommand{\R}{\reals}
\newcommand{\complex}{\mathbb{C}}
\newcommand{\C}{\complex}
\newcommand{\comp}{^{\complement}}
\newcommand{\Hom}{\ptxt{Hom}\>}

% Graph Theory
\renewcommand{\deg}[1]{\ptxt{deg}}
\newcommand{\degp}[1]{\ptxt{deg}^{+}\!\!}
\newcommand{\degn}[1]{\ptxt{deg}^{-}\!\!}
\newcommand{\Prob}{\mathbb{P}}
\newcommand{\Avg}{\mathbb{E}}

% Standard Math
\newcommand{\inv}{^{-1}}
\newcommand{\abs}[1]{\left|#1\right|}
\newcommand{\ceil}[1]{\left\lceil{}#1\right\rceil}
\newcommand{\floor}[1]{\left\lfloor{}#1\right\rfloor{}}
\newcommand{\conj}[1]{\overline{#1}}
\newcommand{\of}{\circ}
\newcommand{\tri}{\triangle}
\newcommand{\inj}{\hookrightarrow}
\newcommand{\surj}{\twoheadrightarrow}
\newcommand{\mapsfrom}{\mathrel{\reflectbox{\ensuremath{\mapsto}}}}
\newcommand{\Graph}{\ptxt{Graph}\>}
\newcommand{\ndiv}{\nmid}
\renewcommand{\epsilon}{\varepsilon}

% Linear Algebra
\newcommand{\Id}{\textrm{\textnormal{Id}}}
\newcommand{\im}{\textrm{\textnormal{im}}}
\newcommand{\norm}[1]{\abs{\abs{#1}}}
\newcommand{\tpose}{^{T}}
\newcommand{\iprod}[1]{\left<#1\right>}
\newcommand{\trace}{\ptxt{tr}}
\newcommand{\chgBasMat}[3]{\!\!\tensor*[_{#1}]{\left[#2\right]}{_{#3}}}
\newcommand{\vecBas}[2]{\tensor*[]{\left[#1\right]}{_{#2}}}
\newcommand{\GL}{\ptxt{GL}\>}
\newcommand{\Mat}{\ptxt{Mat}\>}
\newcommand{\Span}{\ptxt{Span}}
\newcommand{\rank}{\ptxt{rank}\>}

% Topology
\newcommand{\closure}[1]{\overline{#1}}
\newcommand{\uball}{\mathcal{U}}
\newcommand{\Int}{\ptxt{Int}\>}
\newcommand{\Ext}{\ptxt{Ext}\>}
\newcommand{\Bd}{\ptxt{Bd}\>}
\newcommand{\rInt}{\ptxt{rInt}\>}

% Analysis
\newcommand{\graph}{\ptxt{graph}}
\newcommand{\epi}{\ptxt{epi}}
\newcommand{\epis}{\ptxt{epi}_{S}}
\newcommand{\hypo}{\ptxt{hypo}}
\newcommand{\hypos}{\ptxt{hypo}_{S}}

% Proofs
\newcommand{\st}{s.t.}
\newcommand{\unique}{!}

% Algorithms
\algrenewcommand{\algorithmiccomment}[1]{\hskip 1em \texttt{// #1}}
\algrenewcommand\algorithmicrequire{\textbf{Input:}}
\algrenewcommand\algorithmicensure{\textbf{Output:}}
\newcommand{\parSymbol}{\P}
\renewcommand{\P}{\ptxt{\textbf{P}}}
\newcommand{\NP}{\ptxt{\textbf{NP}}}

%%%%%%%%%%%%%%%%%%%%%%%%%%%%%
% Other commands
%%%%%%%%%%%%%%%%%%%%%%%%%%%%%
\newcommand{\flag}[1]{\textbf{\textcolor{red}{#1}}}

%%%%%%%%%%%%%%%%%%%%%%%%%%%%%
% Make l's curvy in math environments
%%%%%%%%%%%%%%%%%%%%%%%%%%%%%
\mathcode`l="8000
\begingroup
\makeatletter
\lccode`\~=`\l
\DeclareMathSymbol{\lsb@l}{\mathalpha}{letters}{`l}
\lowercase{\gdef~{\ifnum\the\mathgroup=\m@ne \ell \else \lsb@l \fi}}%
\endgroup

\author{Professor David Barrett\\ \small\textit{Transcribed by Thomas Cohn}}
\title{Beginning Integration}
\date{10/22/18} % Can also use \today

\begin{document}
\maketitle
\setlength\RaggedRightParindent{\parindent}
\RaggedRight

\par\noindent On Friday, we proved that given $f\in{}C^{2}(\Omega^{\ptxt{convex osso}\R^{n}},\R)$, $Hf(\vec{x})\ge{}0$ $\forall\vec{x}\in\Omega$, and $Df(\vec{x_{0}})=0$, then $f(\vec{x})\ge{}f(\vec{x_{0}})$ for all $\vec{x}\in\Omega$.\n

\cor{Given $f\in{}C^{2}(\Omega^{\ptxt{convex osso}\R^{n}},\R)$ and $Hf\ge{}0$ on $\Omega$, then $f(\vec{x})\ge{}f(\vec{x_{0}})-Df(\vec{x_{0}})(\vec{x}-\vec{x_{0}})-f(\vec{x_{0}})$.\n
This is a strict inequality for $\vec{x}\ne\vec{x_{0}}$ if $Hf>0$ on $\Omega$.\n
Proof: Let $g(\vec{x})=f(\vec{x})-Df(\vec{x_{0}})(\vec{x}-\vec{x_{0}})-f(\vec{x_{0}})$.\n
Then $Dg(\vec{x})=Df(\vec{x})-Df(\vec{x_{0}})$ and $Hg(\vec{x})=Hf(\vec{x})$.\n
Notice that $Dg(\vec{x_{0}})=\vec{0}$. So $g(\vec{x})\ge{}g(\vec{x_{0}})$.\proven}

\defn{For $\psi:\Omega\to\R$, the \underline{epigraph} of $\psi$ is $\set{(\vec{x},y)\in\Omega\times\R:y\ge\psi(\vec{x})}$.}

\defn{The opposite of the epigraph is the \underline{hypograph}.}

\cor{Given $f\in{}C^{2}(\Omega^{\ptxt{convex osso}\R^{n}},\R)$ and $Hf\ge{}0$, then\n
$\epi(f)=\bigcap_{\vec{x_{0}}\in\Omega}\set{(\vec{x},y)\in\Omega\times{}\R:y\ge{}f(\vec{x_{0}})+Df(\vec{x_{0}})(\vec{x}-\vec{x_{0}})}$\n
Proof: $(\vec{x},y)\in\epi(f)\Rightarrow(\vec{x},y)\in\ptxt{RHS}$ by previous result. So assume $(\vec{x},y)\in\ptxt{RHS}$ Then let $\vec{x_{0}}=\vec{x}$; then $y\ge{}f(\vec{x})\Rightarrow(\vec{x},y)\in\epi(f)$.\proven}

\cor{Same hypothesis as above $\Rightarrow$ the epigraph is convex.}

\defn{For $\Omega^{\ptxt{convex}}\subset\R^{n}$, $f:\Omega\to\R$, $f$ is convex $\overset{\ptxt{def}}{\Leftrightarrow}\epi(f)$ is convex.\n
\phantom{For $\Omega^{\ptxt{convex}}\subset\R^{n}$, $f:\Omega\to\R$, $f$ is convex}$\overset{\ptxt{HW8}}{\Leftrightarrow}f((1-t)\vec{x_{0}}+t\vec{x_{1}})\le{}(1-t)f(\vec{x_{0}})+tf(\vec{x_{1}})$\n
\phantom{For $\Omega^{\ptxt{convex}}\subset\R^{n}$, $f:\Omega\to\R$, $f$ is convex$\overset{\ptxt{HW8}}{\Leftrightarrow}$} for $\vec{x_{0}},\vec{x_{1}}\in\Omega$ and $0\le{}t\le{}1$.}

\par\noindent Assume $f\in{}C^{2}(\Omega,\R)$.\n
\begin{tabular}{rcl}
$Hf(\vec{x_{0}})\ne{}0$ & $\Leftrightarrow$ & $\vec{a}\tpose{}Hf(\vec{x_{0}})\vec{a}<0$ for some $\vec{a}$.\\
 (Friday) & $\Rightarrow$ & $(f\of\varphi)''(0)<0$ for $\varphi(t)=\vec{x_{0}}+t\vec{a}$\\
 & $\Rightarrow$ & $\epi(f)\cap\set{(\vec{x_{0}}+t\vec{a},y):t,y\in\R}$ is affine, and hence convex.\\
 & $\Rightarrow$ & $\epi(f)$ is \textit{not} convex.\\
\end{tabular}\n\n

\cor{Given $f\in{}C^{2}(\Omega^{\ptxt{convex osso}\R^{n}},\R)$, then $f$ is convex if and only if $Hf(\vec{x})\ge{}0$ for all $\vec{x}\in\Omega$.}

\begin{tabular}{c|c|c|}
& $\R$ & $\R^{n}$\\ \hline
Reimann/Darboux & 295/297 & Munkres/Lecture\\ \hline
Lebesgue & IBL & Later\\ \hline
\end{tabular}

\par\noindent Lebesgue integration is more robust and coherent.\n

\par\noindent $Q=[a_{1},b_{1}]\times\cdots\times[a_{n},b_{n}]$. Munkres calls this a rectangle. We'll call it a box.\n

\defn{$\displaystyle{}V(Q)\overset{\ptxt{def}}{=}(b_{1}-a_{1})\cdots(b_{n}-a_{n})=\prod_{i=1}^{n}(b_{i}-a_{i})$. In IBL, we may say $m(Q)$.}

\par\noindent We want to define $\int_{Q}f$ for $f:Q\to\R$ (assume $Q$ is bounded).\n

\par\noindent We hope to have $\int_{Q}c=c\cdot{}V(Q)$, and $f\le{}g\to\int_{Q}f\le\int_{Q}g$.\n

\par\noindent Subdivide each $[a_{j},b_{j}]$ with finitely many partition points. We want $\int_{Q}f=\sum\int_{R}f$ for $R$ subbox of $Q$. So...\n

\par\noindent Set $m_{R}(f)=\underset{R}{\inf{}}f=\inf\set{f(\vec{x}):\vec{x}\in{}R}$\n
$M_{R}(f)=\underset{R}{\sup{}}f$\n
$L(f,P)\overset{\ptxt{def}}{=}\sum_{R}m_{R}(f)\cdot{}V(R)$\n
$U(f,P)\overset{\ptxt{def}}{=}\sum_{R}M_{R}(f)\cdot{}V(R)$\n

\par\noindent Then we have $L(f,P)\le{}U(f,P)$.\n

\defn{$P'$ \underline{refines} $P$ if and only if $P'$ is obtained from $P$ by adding more partition points.}

\par\noindent Then $L(f,P)\le{}L(f,P')\le{}U(f,P')\le{}U(f,P)$.\n

\lemma{$P,P'$ arbitrary partitions of $Q$. Then $L(f,P)\le{}U(f,P')$.\n
Proof: Let $P''$ use all partition points in $P$ and $P'$. Then it refines both, so\n
$L(f,P)\le{}L(f,P'')\le{}U(f,P'')\le{}U(f,P')$.\proven}

\defn{$\displaystyle\underline{\int_{Q}}f\overset{\ptxt{def}}{=}\underset{P}{\sup{}}L(f,P)$\n}

\defn{$\displaystyle\overline{\int_{Q}}f\overset{\ptxt{def}}{=}\underset{P}{\inf{}}U(f,P)$\n}

\defn{Lemma + ${}^{295\#11}/{}_{297\#12}$ $\displaystyle\Rightarrow\underline{\int_{Q}}f=\overline{\int_{Q}}f\overset{\ptxt{def}}{=}\int_{Q}f$ (if they match).\n}

\thm{(``Riemann Criterion'' or ``Cauchy Criterion'' for Integrability)\n
$f$ is (Riemann)-integrable on $Q$ $\Leftrightarrow$ $\forall\epsilon>0$, $\exists{}P$ partition \st{} $U(f,P)-L(f,P)<\epsilon$.}

\end{document}