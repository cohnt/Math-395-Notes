\documentclass[10pt,letterpaper]{article}
\usepackage[utf8]{inputenc}
\usepackage{amsmath}
\usepackage{amsfonts}
\usepackage{amssymb}
\usepackage{ragged2e}
\usepackage[letterpaper, margin=1in]{geometry}
\usepackage{graphicx}
\usepackage{cancel}
\usepackage{mathtools}
\usepackage{tabularx}
\usepackage{arydshln}
\usepackage{tensor}
\usepackage{array}
\usepackage{xcolor}

%%%%%%%%%%%%%%%%%%%%%%%%%%%%%
% Formatting commands
%%%%%%%%%%%%%%%%%%%%%%%%%%%%%
\newcommand{\n}{\hfill\break}
\newcommand{\lemma}[1]{\par\noindent\settowidth{\hangindent}{\textbf{Lemma: }}\textbf{Lemma: }#1\n}
\newcommand{\defn}[1]{\par\noindent\settowidth{\hangindent}{\textbf{Defn: }}\textbf{Defn: }#1\n}
\newcommand{\thm}[1]{\par\noindent\settowidth{\hangindent}{\textbf{Thm: }}\textbf{Thm: }#1\n}
\newcommand{\ex}[1]{\par\noindent\settowidth{\hangindent}{\textbf{Ex: }}\textbf{Ex: }#1\n}
\newcommand{\proven}{\;$\square$\n}
\newcommand{\problem}[1]{\par\noindent{#1}\n}
\newcommand{\problempart}[2]{\par\settowidth{\hangindent}{\textbf{(#1)} \indent{}}\textbf{(#1)} #2\n}
\newcommand{\ptxt}[1]{\textrm{\textnormal{#1}}}
\newcommand{\inlineeq}[1]{\n\centerline{$\displaystyle #1$}}
\newcommand{\pageline}{\noindent\rule{\textwidth}{0.1pt}}

%%%%%%%%%%%%%%%%%%%%%%%%%%%%%
% Math commands
%%%%%%%%%%%%%%%%%%%%%%%%%%%%%
% Set Theory
\newcommand{\card}[1]{\left|#1\right|}
\newcommand{\set}[1]{\left\{#1\right\}}
\newcommand{\ps}[1]{\mathcal{P}(#1)}
\newcommand{\naturals}{\mathbb{N}}
\newcommand{\N}{\naturals}
\newcommand{\integers}{\mathbb{Z}}
\newcommand{\Z}{\integers}
\newcommand{\rationals}{\mathbb{Q}}
\newcommand{\Q}{\rationals}
\newcommand{\reals}{\mathbb{R}}
\newcommand{\R}{\reals}
\newcommand{\complex}{\mathbb{C}}
\newcommand{\C}{\complex}
\newcommand{\comp}{^{\complement}}

% Graph Theory
\renewcommand{\deg}[1]{\ptxt{deg}\left(#1\right)}

% Standard Math
\newcommand{\inv}{^{-1}}
\newcommand{\abs}[1]{\left|#1\right|}
\newcommand{\ceil}[1]{\left\lceil{}#1\right\rceil}
\newcommand{\floor}[1]{\left\lfloor{}#1\right\rfloor{}}
\newcommand{\conj}[1]{\overline{#1}}
\newcommand{\of}{\circ}
\newcommand{\tri}{\triangle}
\newcommand{\inj}{\hookrightarrow}
\newcommand{\surj}{\twoheadrightarrow}
\newcommand{\mapsfrom}{\mathrel{\reflectbox{\ensuremath{\mapsto}}}}

% Linear Algebra
\newcommand{\Id}{\textrm{\textnormal{Id}}}
\newcommand{\im}{\textrm{\textnormal{im}}}
\newcommand{\norm}[1]{\abs{\abs{#1}}}
\newcommand{\tpose}{^{T}}
\newcommand{\iprod}[1]{\left<#1\right>}
\newcommand{\trace}{\ptxt{tr}}
\newcommand{\chgBasMat}[3]{\!\!\tensor*[_{#1}]{\left[#2\right]}{_{#3}}}
\newcommand{\vecBas}[2]{\tensor*[]{\left[#1\right]}{_{#2}}}

% Topology
\newcommand{\closure}[1]{\bar{#1}}

% Proofs
\newcommand{\st}{s.t.}
\newcommand{\unique}{!}

%%%%%%%%%%%%%%%%%%%%%%%%%%%%%
% Other commands
%%%%%%%%%%%%%%%%%%%%%%%%%%%%%
\newcommand{\flag}[1]{\textbf{\textcolor{red}{#1}}}

%%%%%%%%%%%%%%%%%%%%%%%%%%%%%
% Make l's curvy in math environments
%%%%%%%%%%%%%%%%%%%%%%%%%%%%%
\mathcode`l="8000
\begingroup
\makeatletter
\lccode`\~=`\l
\DeclareMathSymbol{\lsb@l}{\mathalpha}{letters}{`l}
\lowercase{\gdef~{\ifnum\the\mathgroup=\m@ne \ell \else \lsb@l \fi}}%
\endgroup

\author{Thomas Cohn}
\title{Moore ``Affine" Notions}
\date{9/7/18} % Can also use \today

\begin{document}
\maketitle
\setlength\RaggedRightParindent{\parindent}
\RaggedRight

\par\noindent The extended reals are denoted as $[-\infty,+\infty]=\R\cup\set{\pm\infty}$.\n

\par\noindent Notation: Suppose for $\alpha\in{}A$ we are given $S_{\alpha}\subset{}X$, i.e., we have a function $f:A\to\ps{X}$ where $\alpha\mapsto{}S_{\alpha}$.\n
$\displaystyle\bigcup_{\alpha\in{}A}S_{\alpha}=\set{x\in{}X:x\in{}S_{\alpha}\ptxt{ for at least one }\alpha\in{}A}$\n
$\displaystyle\bigcap_{\alpha\in{}A}S_{\alpha}=\set{x\in{}X:x\in{}S_{\alpha}\ptxt{ for all }\alpha\in{}A}$\n
If $A\ne\emptyset$, then $\displaystyle\bigcup_{\alpha\in{}A}S_{\alpha}\ne\emptyset$ and $\displaystyle\bigcap_{\alpha\in{}A}S_{\alpha}=X$. \flag{Is that right?}\n\n

\par\noindent Let $V,W$ be vector spaces over $F$, a field where $1+1\ne{}0$. We will study functions $T:V\to{}W$.\n

\par\noindent Graph $T=\set{(\vec{v},\vec{w})\in{}V\times{}W:\vec{w}=T(\vec{v})}=\set{(\vec{v},T(\vec{v}))\in{}V\times{}W:\vec{v}\in{}V}$. Note that $V\times{}W$ is a vector space: $(\vec{v_{1}},\vec{w_{1}})+(\vec{v_{2}},\vec{w_{2}})=(\vec{v_{1}}+\vec{v_{2}},\vec{w_{1}}+\vec{w_{2}})$ and $t(\vec{v},\vec{w})=(t\vec{v},t\vec{w})$.\n

\par\noindent The ``simplest'' $T$'s are those with flat graphs, i.e., a graph that is an affine subset of $V\times{}W$.\n

\defn{A function $T$ is \underline{affine} if and only if its graph is affine.}

\par\noindent Special Case: $T(\vec{0})=\vec{0}$, or equivalently, $(\vec{0},\vec{0})\in\ptxt{Graph }T$.\n
Then $T$ is affine $\leftrightarrow$ Graph $T$ is a linear subspace of $V\times{}W$.\n
\phantom{Then $T$ is affine }$\leftrightarrow$ $\vec{v_{1}},\vec{v_{2}}\in{}V$, $t\in{}F$ implies that $T(\vec{v_{1}}+\vec{v_{2}})=T(\vec{v_{1}})+T(\vec{v_{2}})$, and $tT(\vec{v_{1}})=T(t\vec{v_{1}})$.\n
\phantom{Then $T$ is affine }$\leftrightarrow$ $T$ is linear.\n

\par\noindent General Case: $T:V\to{}W$ is affine $\leftrightarrow$ Graph $T$ is affine.\n
\phantom{General Case: $T:V\to{}W$ is affine }$\leftrightarrow$ Graph $T-(\vec{0},T(\vec{0}))$ is a linear subspace.\n
\phantom{\phantom{General Case: $T:V\to{}W$ is affine }$\leftrightarrow$ }Note that Graph $T-(\vec{0},T(\vec{0}))=\set{(\vec{v},T(\vec{v})-T(\vec{0})):\vec{v}\in{}V}$.\n
\phantom{General Case: $T:V\to{}W$ is affine }$\leftrightarrow$ $L$ is linear. \flag{What was $L$?}\n
\phantom{General Case: $T:V\to{}W$ is affine }$\leftrightarrow$ $T$ is of the form $T(\vec{v})=\widetilde{T}(\vec{v})+\vec{b}$ with $\widetilde{T}$ linear.\n

\par\noindent As an exercise, prove that $\widetilde{T}$ is uniquely determined by $T$.

\end{document}