\documentclass[10pt,letterpaper]{article}
\usepackage[utf8]{inputenc}
\usepackage{amsmath}
\usepackage{amsfonts}
\usepackage{amssymb}
\usepackage{ragged2e}
\usepackage[letterpaper, margin=1in]{geometry}
\usepackage{graphicx}
\usepackage{cancel}
\usepackage{mathtools}
\usepackage{tabularx}
\usepackage{arydshln}
\usepackage{tensor}
\usepackage{array}
\usepackage{xcolor}
\usepackage[boxed]{algorithm}
\usepackage[noend]{algpseudocode}
\usepackage{listings}
\usepackage{textcomp}
\usepackage[pdf,tmpdir,singlefile]{graphviz}
\usepackage{mathrsfs}

%%%%%%%%%%%%%%%%%%%%%%%%%%%%%
% Formatting commands
%%%%%%%%%%%%%%%%%%%%%%%%%%%%%
\newcommand{\n}{\hfill\break}
\newcommand{\lemma}[1]{\par\noindent\settowidth{\hangindent}{\textbf{Lemma: }}\textbf{Lemma: }#1}
\newcommand{\defn}[1]{\par\noindent\settowidth{\hangindent}{\textbf{Defn: }}\textbf{Defn: }#1\n}
\newcommand{\thm}[1]{\par\noindent\settowidth{\hangindent}{\textbf{Thm: }}\textbf{Thm: }#1\n}
\newcommand{\prop}[1]{\par\noindent\settowidth{\hangindent}{\textbf{Prop: }}\textbf{Prop: }#1\n}
\newcommand{\cor}[1]{\par\noindent\settowidth{\hangindent}{\textbf{Cor: }}\textbf{Cor: }#1\n}
\newcommand{\ex}[1]{\par\noindent\settowidth{\hangindent}{\textbf{Ex: }}\textbf{Ex: }#1\n}
\newcommand{\proven}{\;$\square$\n}
\newcommand{\problem}[1]{\par\noindent{#1}\n}
\newcommand{\problempart}[2]{\par\noindent\indent{}\settowidth{\hangindent}{\textbf{(#1)} \indent{}}\textbf{(#1)} #2\n}
\newcommand{\ptxt}[1]{\textrm{\textnormal{#1}}}
\newcommand{\inlineeq}[1]{\centerline{$\displaystyle #1$}}
\newcommand{\pageline}{\noindent\rule{\textwidth}{0.1pt}}

%%%%%%%%%%%%%%%%%%%%%%%%%%%%%
% Math commands
%%%%%%%%%%%%%%%%%%%%%%%%%%%%%
% Set Theory
\newcommand{\card}[1]{\left|#1\right|}
\newcommand{\set}[1]{\left\{#1\right\}}
\newcommand{\ps}[1]{\mathcal{P}\left(#1\right)}
\newcommand{\pfinite}[1]{\mathcal{P}^{\ptxt{finite}}\left(#1\right)}
\newcommand{\naturals}{\mathbb{N}}
\newcommand{\N}{\naturals}
\newcommand{\integers}{\mathbb{Z}}
\newcommand{\Z}{\integers}
\newcommand{\rationals}{\mathbb{Q}}
\newcommand{\Q}{\rationals}
\newcommand{\reals}{\mathbb{R}}
\newcommand{\R}{\reals}
\newcommand{\complex}{\mathbb{C}}
\newcommand{\C}{\complex}
\newcommand{\comp}{^{\complement}}
\newcommand{\Hom}{\ptxt{Hom}\>}

% Graph Theory
\renewcommand{\deg}[1]{\ptxt{deg}}
\newcommand{\degp}[1]{\ptxt{deg}^{+}\!\!}
\newcommand{\degn}[1]{\ptxt{deg}^{-}\!\!}
\newcommand{\Prob}{\mathbb{P}}
\newcommand{\Avg}{\mathbb{E}}

% Standard Math
\newcommand{\inv}{^{-1}}
\newcommand{\abs}[1]{\left|#1\right|}
\newcommand{\ceil}[1]{\left\lceil{}#1\right\rceil}
\newcommand{\floor}[1]{\left\lfloor{}#1\right\rfloor{}}
\newcommand{\conj}[1]{\overline{#1}}
\newcommand{\of}{\circ}
\newcommand{\tri}{\triangle}
\newcommand{\inj}{\hookrightarrow}
\newcommand{\surj}{\twoheadrightarrow}
\newcommand{\mapsfrom}{\mathrel{\reflectbox{\ensuremath{\mapsto}}}}
\newcommand{\Graph}{\ptxt{Graph}\>}
\newcommand{\ndiv}{\nmid}
\renewcommand{\epsilon}{\varepsilon}

% Linear Algebra
\newcommand{\Id}{\textrm{\textnormal{Id}}}
\newcommand{\im}{\textrm{\textnormal{im}}}
\newcommand{\norm}[1]{\abs{\abs{#1}}}
\newcommand{\tpose}{^{T}}
\newcommand{\iprod}[1]{\left<#1\right>}
\newcommand{\trace}{\ptxt{tr}}
\newcommand{\chgBasMat}[3]{\!\!\tensor*[_{#1}]{\left[#2\right]}{_{#3}}}
\newcommand{\vecBas}[2]{\tensor*[]{\left[#1\right]}{_{#2}}}
\newcommand{\GL}{\ptxt{GL}\>}
\newcommand{\Mat}{\ptxt{Mat}\>}
\newcommand{\Span}{\ptxt{Span}}
\newcommand{\rank}{\ptxt{rank}\>}

% Topology
\newcommand{\closure}[1]{\overline{#1}}
\newcommand{\uball}{\mathcal{U}}
\newcommand{\Int}{\ptxt{Int}\>}
\newcommand{\Ext}{\ptxt{Ext}\>}
\newcommand{\Bd}{\ptxt{Bd}\>}
\newcommand{\rInt}{\ptxt{rInt}\>}

% Proofs
\newcommand{\st}{s.t.}
\newcommand{\unique}{!}

% Algorithms
\algrenewcommand{\algorithmiccomment}[1]{\hskip 1em \texttt{// #1}}
\algrenewcommand\algorithmicrequire{\textbf{Input:}}
\algrenewcommand\algorithmicensure{\textbf{Output:}}
\newcommand{\parSymbol}{\P}
\renewcommand{\P}{\ptxt{\textbf{P}}}
\newcommand{\NP}{\ptxt{\textbf{NP}}}

%%%%%%%%%%%%%%%%%%%%%%%%%%%%%
% Other commands
%%%%%%%%%%%%%%%%%%%%%%%%%%%%%
\newcommand{\flag}[1]{\textbf{\textcolor{red}{#1}}}

%%%%%%%%%%%%%%%%%%%%%%%%%%%%%
% Make l's curvy in math environments
%%%%%%%%%%%%%%%%%%%%%%%%%%%%%
\mathcode`l="8000
\begingroup
\makeatletter
\lccode`\~=`\l
\DeclareMathSymbol{\lsb@l}{\mathalpha}{letters}{`l}
\lowercase{\gdef~{\ifnum\the\mathgroup=\m@ne \ell \else \lsb@l \fi}}%
\endgroup

\author{Professor David Barrett\\ \small\textit{Transcribed by Thomas Cohn}}
\title{The Local Second Derivative Test}
\date{10/19/18} % Can also use \today

\begin{document}
\maketitle
\setlength\RaggedRightParindent{\parindent}
\RaggedRight

\par\noindent For $n=1$: Given $f\in{}C^{2}(I^{\ptxt{interval}\subset\R},\R)$, $f''\ge{}0$ on $I$, and $f'(x_{0})=0$, then $f(x)\ge{}f(x_{0})$ for all $x\in{}I$.\n

\par\noindent For any $n$: Given $f\in{}C^{2}(\Omega^{\ptxt{convex osso}\R^{n}},\R)$, $Hf(\vec{x})\ge{}0$ $\forall\vec{x}\in\Omega$ (that is, $\vec{a}\tpose\cdot{}Hf(\vec{x})\cdot\vec{a}\ge{}0$ for all $\vec{a}$), and $Df(\vec{x_{0}})=\vec{0}$, then $f(\vec{x})\ge{}f(\vec{x_{0}})$ for all $\vec{x}\in\Omega$.\n

\par\noindent Proof: Let $\varphi(t)=(1-t)\vec{x_{0}}+t\vec{x}$. Then $\varphi'(t)=\vec{x}-\vec{x_{0}}\overset{\ptxt{def}}{=}\vec{a}$.\n
So $(f\of\varphi)'(t)=f'(\varphi(t))\cdot\vec{a}=\sum_{j}D_{j}f(\varphi(t))\cdot\varphi'(t)\cdot{}a_{j}$\n
$(f\of\varphi)''(t)=\sum_{j}DD_{j}f(\varphi(t))\cdot\varphi'(t)\cdot{}a_{j}=\sum_{j,k}D_{k}D_{j}f(\varphi(t))\cdot{}a_{k}\cdot{}a_{j}=\vec{a}\tpose{}Hf(\varphi(t))\vec{a}\ge{}0$\n

\par\noindent The one dimensional result implies that $f(\vec{x})=(f\of\varphi)(1)\ge(f\of\varphi)(0)=f(\vec{x_{0}})$.\n

\par\noindent If we also have $Hf(\vec{x})>0$, then $\vec{a}\tpose{}Hf(\vec{x})\vec{a}>0$ for $\vec{a}\in\R^{n}\setminus\set{\vec{0}}$.\n

\pageline

\par\noindent Consider $f\in{}C^{2}(\ptxt{osso}\R^{n},\R)$ with $\vec{x_{0}}$ in the domain of $f$ and $Df(\vec{x_{0}})=\vec{0}$ (that is, $\vec{x_{0}}$ is a critical point for $f$).\n

\par\noindent $Hf(\vec{x_{0}})>0\Rightarrow{}Hf(\vec{x})>0$ for $\vec{x}\in\uball(\vec{x_{0}},\delta)\Rightarrow{}f$ has a strict local minimum at $\vec{x_{0}}$.\n
$Hf(\vec{x_{0}})\not\ge{}0\Rightarrow{}f$ has a strict local max along some line through $\vec{x_{0}}$. $f$ does \underline{not} have a local max at $\vec{x_{0}}$.\n
$Hf(\vec{x_{0}})<0\Rightarrow{}f$ has a strict local max at $\vec{x_{0}}$.\n
$Hf(\vec{x_{0}})\not\le{}0\Rightarrow{}f$ does not have a local max at $\vec{x_{0}}$.\n
$Hf(\vec{x_{0}})\begin{array}{c}\not\ge\\ \not\le\end{array}0\Rightarrow{}f$ does not have a local max or min at $\vec{x_{0}}$.\n

\ex{Consider $f(x)=x^{3}$, $f_{1}(x)=x^{3}+\frac{x}{10}$, and $f_{2}(x)=x^{3}-\frac{x}{10}$. They're all different -- $f$ is delicate.}

\ex{$f\left(\begin{array}{c}x_{1}\\ x_{2}\end{array}\right)=\left(\begin{array}{cc}x_{1} & x_{2}\end{array}\right)$\n
$Df\left[\begin{array}{c}0\\ 0\end{array}\right]=\left[\begin{array}{cc}0 & 0\end{array}\right]$ critical point\n
$Hf\left[\begin{array}{c}0\\ 0\end{array}\right]=\left[\begin{array}{cc}0 & 1\\ 1 & 0\end{array}\right]$\n
\n
This gives us a ``saddle point''.}

\end{document}