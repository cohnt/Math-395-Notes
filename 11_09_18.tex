\documentclass[10pt,letterpaper]{article}
\usepackage[utf8]{inputenc}
\usepackage[intlimits]{amsmath}
\usepackage{amsfonts}
\usepackage{amssymb}
\usepackage{ragged2e}
\usepackage[letterpaper, margin=1in]{geometry}
\usepackage{graphicx}
\usepackage{cancel}
\usepackage{mathtools}
\usepackage{tabularx}
\usepackage{arydshln}
\usepackage{tensor}
\usepackage{array}
\usepackage{xcolor}
\usepackage[boxed]{algorithm}
\usepackage[noend]{algpseudocode}
\usepackage{listings}
\usepackage{textcomp}
\usepackage[pdf,tmpdir,singlefile]{graphviz}
\usepackage{mathrsfs}
\usepackage{bbm}
\usepackage{tikz}
\usepackage{enumitem}
\usepackage{arydshln}

%%%%%%%%%%%%%%%%%%%%%%%%%%%%%
% Formatting commands
%%%%%%%%%%%%%%%%%%%%%%%%%%%%%
\newcommand{\n}{\hfill\break}
\newcommand{\up}{\vspace{-\baselineskip}}
\newcommand{\lemma}[1]{\par\noindent\settowidth{\hangindent}{\textbf{Lemma: }}\textbf{Lemma: }#1}
\newcommand{\defn}[1]{\par\noindent\settowidth{\hangindent}{\textbf{Defn: }}\textbf{Defn: }#1\n}
\newcommand{\thm}[1]{\par\noindent\settowidth{\hangindent}{\textbf{Thm: }}\textbf{Thm: }#1\n}
\newcommand{\prop}[1]{\par\noindent\settowidth{\hangindent}{\textbf{Prop: }}\textbf{Prop: }#1\n}
\newcommand{\cor}[1]{\par\noindent\settowidth{\hangindent}{\textbf{Cor: }}\textbf{Cor: }#1\n}
\newcommand{\ex}[1]{\par\noindent\settowidth{\hangindent}{\textbf{Ex: }}\textbf{Ex: }#1\n}
\newcommand{\proven}{\;$\square$\n}
\newcommand{\problem}[1]{\par\noindent{#1}\n}
\newcommand{\problempart}[2]{\par\noindent\indent{}\settowidth{\hangindent}{\textbf{(#1)} \indent{}}\textbf{(#1)} #2\n}
\newcommand{\ptxt}[1]{\textrm{\textnormal{#1}}}
\newcommand{\inlineeq}[1]{\centerline{$\displaystyle #1$}}
\newcommand{\pageline}{\noindent\rule{\textwidth}{0.1pt}}

%%%%%%%%%%%%%%%%%%%%%%%%%%%%%
% Math commands
%%%%%%%%%%%%%%%%%%%%%%%%%%%%%
% Set Theory
\newcommand{\card}[1]{\left|#1\right|}
\newcommand{\set}[1]{\left\{#1\right\}}
\newcommand{\ps}[1]{\mathcal{P}\left(#1\right)}
\newcommand{\pfinite}[1]{\mathcal{P}^{\ptxt{finite}}\left(#1\right)}
\newcommand{\naturals}{\mathbb{N}}
\newcommand{\N}{\naturals}
\newcommand{\integers}{\mathbb{Z}}
\newcommand{\Z}{\integers}
\newcommand{\rationals}{\mathbb{Q}}
\newcommand{\Q}{\rationals}
\newcommand{\reals}{\mathbb{R}}
\newcommand{\R}{\reals}
\newcommand{\complex}{\mathbb{C}}
\newcommand{\C}{\complex}
\newcommand{\comp}{^{\complement}}
\DeclareMathOperator{\Hom}{Hom}
\newcommand{\Ind}{\mathbbm{1}}
\newcommand{\cut}{\setminus}

% Graph Theory
\let\deg\relax
\DeclareMathOperator{\deg}{deg}
\newcommand{\degp}{\ptxt{deg}^{+}}
\newcommand{\degn}{\ptxt{deg}^{-}}
\newcommand{\precdot}{\mathrel{\ooalign{$\prec$\cr\hidewidth\hbox{$\cdot\mkern0.5mu$}\cr}}}
\newcommand{\succdot}{\mathrel{\ooalign{$\cdot\mkern0.5mu$\cr\hidewidth\hbox{$\succ$}\cr\phantom{$\succ$}}}}
\DeclareMathOperator{\cl}{cl}
\DeclareMathOperator{\affdim}{affdim}

% Probability
\newcommand{\Prob}{\mathbb{P}}
\newcommand{\Avg}{\mathbb{E}}

% Standard Math
\newcommand{\inv}{^{-1}}
\newcommand{\abs}[1]{\left|#1\right|}
\newcommand{\ceil}[1]{\left\lceil{}#1\right\rceil{}}
\newcommand{\floor}[1]{\left\lfloor{}#1\right\rfloor{}}
\newcommand{\conj}[1]{\overline{#1}}
\newcommand{\of}{\circ}
\newcommand{\tri}{\triangle}
\newcommand{\inj}{\hookrightarrow}
\newcommand{\surj}{\twoheadrightarrow}
\newcommand{\mapsfrom}{\mathrel{\reflectbox{\ensuremath{\mapsto}}}}
\newcommand{\ndiv}{\nmid}
\renewcommand{\epsilon}{\varepsilon}
\newcommand{\divides}{\mid}
\newcommand{\ndivdies}{\nmid}
\DeclareMathOperator{\lcm}{lcm}

% Linear Algebra
\newcommand{\Id}{\textrm{\textnormal{Id}}}
\newcommand{\im}{\textrm{\textnormal{im}}}
\newcommand{\norm}[1]{\abs{\abs{#1}}}
\newcommand{\tpose}{^{T}}
\newcommand{\iprod}[1]{\left<#1\right>}
\DeclareMathOperator{\trace}{tr}
\newcommand{\chgBasMat}[3]{\!\!\tensor*[_{#1}]{\left[#2\right]}{_{#3}}}
\newcommand{\vecBas}[2]{\tensor*[]{\left[#1\right]}{_{#2}}}
\DeclareMathOperator{\GL}{GL}
\DeclareMathOperator{\Mat}{Mat}
\DeclareMathOperator{\vspan}{span}
\DeclareMathOperator{\rank}{rank}
\newcommand{\V}[1]{\vec{#1}}

% Topology
\newcommand{\closure}[1]{\overline{#1}}
\newcommand{\uball}{\mathcal{U}}
\DeclareMathOperator{\Int}{Int}
\DeclareMathOperator{\Ext}{Ext}
\DeclareMathOperator{\Bd}{Bd}
\DeclareMathOperator{\rInt}{rInt}
\DeclareMathOperator{\ch}{ch}
\DeclareMathOperator{\ah}{ah}

% Analysis
\DeclareMathOperator{\Graph}{Graph}
\DeclareMathOperator{\epi}{epi}
\DeclareMathOperator{\hypo}{hypo}
\DeclareMathOperator{\supp}{supp}
\newcommand{\lint}[2]{\underset{#1}{\overset{#2}{{\color{black}\underline{{\color{white}\overline{{\color{black}\int}}\color{black}}}}}}}
\newcommand{\uint}[2]{\underset{#1}{\overset{#2}{{\color{white}\underline{{\color{black}\overline{{\color{black}\int}}\color{black}}}}}}}
\newcommand{\alignint}[2]{\underset{#1}{\overset{#2}{{\color{white}\underline{{\color{white}\overline{{\color{black}\int}}\color{black}}}}}}}
\newcommand{\extint}{\ptxt{ext}\int}
\newcommand{\extalignint}[2]{\ptxt{ext}\alignint{#1}{#2}}
\newcommand{\conv}{\ast}

% Proofs
\newcommand{\st}{s.t.}
\newcommand{\unique}{!}

% Brackets
\newcommand{\paren}[1]{\left(#1\right)}
\renewcommand{\brack}[1]{\left[#1\right]}
\renewcommand{\brace}[1]{\left\{#1\right\}}
\newcommand{\ang}[1]{\left<#1\right>}

% Algorithms
\algrenewcommand{\algorithmiccomment}[1]{\hskip 1em \texttt{// #1}}
\algrenewcommand\algorithmicrequire{\textbf{Input:}}
\algrenewcommand\algorithmicensure{\textbf{Output:}}
\newcommand{\parSymbol}{\P}
\renewcommand{\P}{\ptxt{\textbf{P}}}
\newcommand{\NP}{\ptxt{\textbf{NP}}}
\newcommand{\NPC}{\ptxt{\textbf{NP-Complete}}}
\newcommand{\NPH}{\ptxt{\textbf{NP-Hard}}}
\newcommand{\EXP}{\ptxt{\textbf{EXP}}}

%%%%%%%%%%%%%%%%%%%%%%%%%%%%%
% Other commands
%%%%%%%%%%%%%%%%%%%%%%%%%%%%%
\newcommand{\flag}[1]{\textbf{\textcolor{red}{#1}}}

%%%%%%%%%%%%%%%%%%%%%%%%%%%%%
% Make l's curvy in math environments
%%%%%%%%%%%%%%%%%%%%%%%%%%%%%
\mathcode`l="8000
\begingroup
\makeatletter
\lccode`\~=`\l
\DeclareMathSymbol{\lsb@l}{\mathalpha}{letters}{`l}
\lowercase{\gdef~{\ifnum\the\mathgroup=\m@ne \ell \else \lsb@l \fi}}%
\endgroup

\newcommand{\B}{
    \begin{tikzpicture}
    \filldraw [fill=red, draw=black] (0, 0) rectangle (0.37, 0.45);
    \draw [line width=0.5mm, white ] (0.1,0.08) -- (0.1,0.38);
    \draw[line width=0.5mm, white ] (0.1, 0.35) .. controls (0.2, 0.35) and (0.4, 0.2625) .. (0.1, 0.225);
    \draw[line width=0.5mm, white ] (0.1, 0.225) .. controls (0.2, 0.225) and (0.4, 0.1625) .. (0.1, 0.1);
    \end{tikzpicture}
}

\author{Professor David Barrett\\ \small\textit{Transcribed by Thomas Cohn}}
\title{Extended Riemann Integrals}
\date{11/9/18} % Can also use \today

\begin{document}
\maketitle
\setlength\RaggedRightParindent{\parindent}
\RaggedRight

\par\noindent Recall: $f\in{}C(A^{\ptxt{osso}\R^{n}},\R)$, $f\ge{}0$\n
$\extint_{A}f\overset{\ptxt{def}}{=}\sup\set{\int_{E}f:E^{\ptxt{cpt},\ptxt{rect}}\subset{}A}$\n
$\extint_{A}f=\ptxt{``ordinary''}\int_{A}f$ if $\int_{A}f$ exists\n
$\extint_{A}f=\lim_{j\to\infty}\int_{E_{j}}f$ if $E_{j}^{\ptxt{cpt},\ptxt{rect}}\subset{}A$, $E_{1}\subset{}E_{2}\subset\cdots$, and $\bigcup_{j=1}^{\infty}\Int{}E_{j}=A$.\n
$\extint_{A}f=\lim_{j\to\infty}\extint_{U_{j}}f$ if $U_{j}^{\ptxt{open}}\subset{}A$, $U_{1}\subset{}U_{2}\subset\cdots$, and $\bigcup_{j=1}^{\infty}U_{j}=A$.\n

\par\noindent Proof of the last one: $\extint_{U_{j}}f\le\extint_{A}f$, so $\lim_{j\to\infty}\extint_{U_{j}}f=\sup\set{\extint_{U_{j}}f}\le\extint_{A}f$.\n
Each compact rectifiable $E\subset{}A$ lies in some $U_{j}$. So $\int_{E}f\le\extint_{U_{j}}f\le\lim_{j\to\infty}\extint_{U_{j}}f$.\n
Then, take the supremum over the $E_{j}$. So $\extint_{A}f\le\lim_{j\to\infty}\extint_{U_{j}}f$.\n

\defn{For $x\in[-\infty,+\infty]$, $x_{+}\overset{\ptxt{def}}{=}\max\set{x,0}=\frac{\abs{x}+x}{2}$ and $x_{-}\overset{\ptxt{def}}{=}\max{\set{-x,0}}=\frac{\abs{x}-x}{2}$.}

\par\noindent Then $x_{+},x_{-}\ge{}0$, $x_{+}\cdot{}x_{-}=0$, $x=x_{+}-x_{-}$, and $\abs{x}=x_{+}+x{-}$.\n

\defn{For $f:X\to[-\infty,\infty]$, $f_{+}(x)\overset{\ptxt{def}}{=}(f(x))_{+}$ is the \underline{positive part of $f$}, and $f_{-}(x)\overset{\ptxt{def}}{=}(f(x))_{-}$ is the \underline{negative part of $f$}.}

\par\noindent $f_{+},f_{-}\ge{}0$, $f_{+}\cdot{}f_{-}=0$, $f=f_{+}-f_{-}$, and $\abs{f}=f_{+}+f_{-}$.\n

\par\noindent Consider $f\in{}C(A^{\ptxt{osso}\R^{n}},\R)$ (with $f$ not necessarily non-negative). Then we say $f$ is ``extended integrable on $A$'' or ``integrable in the extended sense'' if $\extint_{A}f_{+},\extint_{A}f_{-}<+\infty$.\n

\par\noindent $\extint_{A}f$ exists if at least one of $\extint_{A}f_{+}$ and $\extint_{A}f_{-}$ is finite. Set $\extint_{A}f=\extint_{A}f_{+}-\extint_{A}f_{-}$.\n

\par\noindent $\extint_{A}af+bg=a\extint_{A}f+b\extint_{A}g$\n
$f\ge{}g$ on $A$ $\Rightarrow$ $\extint_{A}f\le\extint_{A}g$ if they exist.\n
For compact, rectifiable $E_{1}\subset{}E_{2}\subset\cdots\subset{}A$ with $\bigcup_{j=1}^{\infty}\Int{}E_{j}=A$, $\extint_{A}f=\lim_{j\to\infty}\int_{E_{j}}f$.\n
For open $U_{1}\subset{}U_{2}\subset\cdots\subset{}A$, with $\bigcup_{j=1}^{\infty}{U_{j}}=A$, $\extint_{A}f=\lim_{j\to\infty}\extint_{U_{j}}f$\n

\par\noindent Consider $Q$ box $\overset{\vec{x}\mapsto{}M\vec{x}+\vec{b}}{\to}$ $P$ parallelopiped,\n
$A^{\ptxt{open}}\subset\R^{n}\overset{g\ptxt{ diffeo}}{\to}B^{\ptxt{open}}\subset\R^{n}\overset{f\ptxt{ cts}}{\to}\R$.\n
Then we want to prove $P$ is rectifiable, $v(P)=\abs{\det{}M}\cdot{}v(Q)$, and $\extint_{B}f=\extint_{A}f$.\n

\thm{(Change of Variable Thm) Given $f,g$ as above, then either $\displaystyle\extint_{B}f=\extint_{A=g\inv[B]}f\of{}g\abs{\det{}Dg}$, or the integral on neither side exists.}

\par\noindent Special case: $n=1$, $A$ connected (i.e. an interval), $A=(\alpha,\beta)$ for $\alpha<\beta\in[-\infty,\infty]$. Then $g$ monotonic.
\begin{enumerate}[label=Case \arabic*:,leftmargin=1.5cm]
	\item $B=(g(\alpha),g(\beta))$. Then $\extint_{B}f=\extint_{A}(f\of{}g)g'$
	\item $B=(g(\beta),g(\alpha))$. Then $\extint_{B}f=-\extint_{A}(f\of{}g)g'\overset{\ptxt{calc 1/2}}{=}-\extint_{g(\beta)}^{g(\alpha)}f$
\end{enumerate}

\end{document}