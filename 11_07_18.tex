\documentclass[10pt,letterpaper]{article}
\usepackage[utf8]{inputenc}
\usepackage[intlimits]{amsmath}
\usepackage{amsfonts}
\usepackage{amssymb}
\usepackage{ragged2e}
\usepackage[letterpaper, margin=1in]{geometry}
\usepackage{graphicx}
\usepackage{cancel}
\usepackage{mathtools}
\usepackage{tabularx}
\usepackage{arydshln}
\usepackage{tensor}
\usepackage{array}
\usepackage{xcolor}
\usepackage[boxed]{algorithm}
\usepackage[noend]{algpseudocode}
\usepackage{listings}
\usepackage{textcomp}
\usepackage[pdf,tmpdir,singlefile]{graphviz}
\usepackage{mathrsfs}
\usepackage{bbm}
\usepackage{tikz}
\usepackage{enumitem}
\usepackage{arydshln}

%%%%%%%%%%%%%%%%%%%%%%%%%%%%%
% Formatting commands
%%%%%%%%%%%%%%%%%%%%%%%%%%%%%
\newcommand{\n}{\hfill\break}
\newcommand{\up}{\vspace{-\baselineskip}}
\newcommand{\lemma}[1]{\par\noindent\settowidth{\hangindent}{\textbf{Lemma: }}\textbf{Lemma: }#1}
\newcommand{\defn}[1]{\par\noindent\settowidth{\hangindent}{\textbf{Defn: }}\textbf{Defn: }#1\n}
\newcommand{\thm}[1]{\par\noindent\settowidth{\hangindent}{\textbf{Thm: }}\textbf{Thm: }#1\n}
\newcommand{\prop}[1]{\par\noindent\settowidth{\hangindent}{\textbf{Prop: }}\textbf{Prop: }#1\n}
\newcommand{\cor}[1]{\par\noindent\settowidth{\hangindent}{\textbf{Cor: }}\textbf{Cor: }#1\n}
\newcommand{\ex}[1]{\par\noindent\settowidth{\hangindent}{\textbf{Ex: }}\textbf{Ex: }#1\n}
\newcommand{\proven}{\;$\square$\n}
\newcommand{\problem}[1]{\par\noindent{#1}\n}
\newcommand{\problempart}[2]{\par\noindent\indent{}\settowidth{\hangindent}{\textbf{(#1)} \indent{}}\textbf{(#1)} #2\n}
\newcommand{\ptxt}[1]{\textrm{\textnormal{#1}}}
\newcommand{\inlineeq}[1]{\centerline{$\displaystyle #1$}}
\newcommand{\pageline}{\noindent\rule{\textwidth}{0.1pt}}

%%%%%%%%%%%%%%%%%%%%%%%%%%%%%
% Math commands
%%%%%%%%%%%%%%%%%%%%%%%%%%%%%
% Set Theory
\newcommand{\card}[1]{\left|#1\right|}
\newcommand{\set}[1]{\left\{#1\right\}}
\newcommand{\ps}[1]{\mathcal{P}\left(#1\right)}
\newcommand{\pfinite}[1]{\mathcal{P}^{\ptxt{finite}}\left(#1\right)}
\newcommand{\naturals}{\mathbb{N}}
\newcommand{\N}{\naturals}
\newcommand{\integers}{\mathbb{Z}}
\newcommand{\Z}{\integers}
\newcommand{\rationals}{\mathbb{Q}}
\newcommand{\Q}{\rationals}
\newcommand{\reals}{\mathbb{R}}
\newcommand{\R}{\reals}
\newcommand{\complex}{\mathbb{C}}
\newcommand{\C}{\complex}
\newcommand{\comp}{^{\complement}}
\DeclareMathOperator{\Hom}{Hom}
\newcommand{\Ind}{\mathbbm{1}}
\newcommand{\cut}{\setminus}

% Graph Theory
\let\deg\relax
\DeclareMathOperator{\deg}{deg}
\newcommand{\degp}{\ptxt{deg}^{+}}
\newcommand{\degn}{\ptxt{deg}^{-}}
\newcommand{\precdot}{\mathrel{\ooalign{$\prec$\cr\hidewidth\hbox{$\cdot\mkern0.5mu$}\cr}}}
\newcommand{\succdot}{\mathrel{\ooalign{$\cdot\mkern0.5mu$\cr\hidewidth\hbox{$\succ$}\cr\phantom{$\succ$}}}}
\DeclareMathOperator{\cl}{cl}
\DeclareMathOperator{\affdim}{affdim}

% Probability
\newcommand{\Prob}{\mathbb{P}}
\newcommand{\Avg}{\mathbb{E}}

% Standard Math
\newcommand{\inv}{^{-1}}
\newcommand{\abs}[1]{\left|#1\right|}
\newcommand{\ceil}[1]{\left\lceil{}#1\right\rceil{}}
\newcommand{\floor}[1]{\left\lfloor{}#1\right\rfloor{}}
\newcommand{\conj}[1]{\overline{#1}}
\newcommand{\of}{\circ}
\newcommand{\tri}{\triangle}
\newcommand{\inj}{\hookrightarrow}
\newcommand{\surj}{\twoheadrightarrow}
\newcommand{\mapsfrom}{\mathrel{\reflectbox{\ensuremath{\mapsto}}}}
\newcommand{\ndiv}{\nmid}
\renewcommand{\epsilon}{\varepsilon}
\newcommand{\divides}{\mid}
\newcommand{\ndivdies}{\nmid}
\DeclareMathOperator{\lcm}{lcm}

% Linear Algebra
\newcommand{\Id}{\textrm{\textnormal{Id}}}
\newcommand{\im}{\textrm{\textnormal{im}}}
\newcommand{\norm}[1]{\abs{\abs{#1}}}
\newcommand{\tpose}{^{T}}
\newcommand{\iprod}[1]{\left<#1\right>}
\DeclareMathOperator{\trace}{tr}
\newcommand{\chgBasMat}[3]{\!\!\tensor*[_{#1}]{\left[#2\right]}{_{#3}}}
\newcommand{\vecBas}[2]{\tensor*[]{\left[#1\right]}{_{#2}}}
\DeclareMathOperator{\GL}{GL}
\DeclareMathOperator{\Mat}{Mat}
\DeclareMathOperator{\vspan}{span}
\DeclareMathOperator{\rank}{rank}
\newcommand{\V}[1]{\vec{#1}}

% Topology
\newcommand{\closure}[1]{\overline{#1}}
\newcommand{\uball}{\mathcal{U}}
\DeclareMathOperator{\Int}{Int}
\DeclareMathOperator{\Ext}{Ext}
\DeclareMathOperator{\Bd}{Bd}
\DeclareMathOperator{\rInt}{rInt}
\DeclareMathOperator{\ch}{ch}
\DeclareMathOperator{\ah}{ah}

% Analysis
\DeclareMathOperator{\Graph}{Graph}
\DeclareMathOperator{\epi}{epi}
\DeclareMathOperator{\hypo}{hypo}
\DeclareMathOperator{\supp}{supp}
\newcommand{\lint}[2]{\underset{#1}{\overset{#2}{{\color{black}\underline{{\color{white}\overline{{\color{black}\int}}\color{black}}}}}}}
\newcommand{\uint}[2]{\underset{#1}{\overset{#2}{{\color{white}\underline{{\color{black}\overline{{\color{black}\int}}\color{black}}}}}}}
\newcommand{\alignint}[2]{\underset{#1}{\overset{#2}{{\color{white}\underline{{\color{white}\overline{{\color{black}\int}}\color{black}}}}}}}
\newcommand{\extint}{\ptxt{ext}\int}
\newcommand{\extalignint}[2]{\ptxt{ext}\alignint{#1}{#2}}
\newcommand{\conv}{\ast}

% Proofs
\newcommand{\st}{s.t.}
\newcommand{\unique}{!}

% Brackets
\newcommand{\paren}[1]{\left(#1\right)}
\renewcommand{\brack}[1]{\left[#1\right]}
\renewcommand{\brace}[1]{\left\{#1\right\}}
\newcommand{\ang}[1]{\left<#1\right>}

% Algorithms
\algrenewcommand{\algorithmiccomment}[1]{\hskip 1em \texttt{// #1}}
\algrenewcommand\algorithmicrequire{\textbf{Input:}}
\algrenewcommand\algorithmicensure{\textbf{Output:}}
\newcommand{\parSymbol}{\P}
\renewcommand{\P}{\ptxt{\textbf{P}}}
\newcommand{\NP}{\ptxt{\textbf{NP}}}
\newcommand{\NPC}{\ptxt{\textbf{NP-Complete}}}
\newcommand{\NPH}{\ptxt{\textbf{NP-Hard}}}
\newcommand{\EXP}{\ptxt{\textbf{EXP}}}

%%%%%%%%%%%%%%%%%%%%%%%%%%%%%
% Other commands
%%%%%%%%%%%%%%%%%%%%%%%%%%%%%
\newcommand{\flag}[1]{\textbf{\textcolor{red}{#1}}}

%%%%%%%%%%%%%%%%%%%%%%%%%%%%%
% Make l's curvy in math environments
%%%%%%%%%%%%%%%%%%%%%%%%%%%%%
\mathcode`l="8000
\begingroup
\makeatletter
\lccode`\~=`\l
\DeclareMathSymbol{\lsb@l}{\mathalpha}{letters}{`l}
\lowercase{\gdef~{\ifnum\the\mathgroup=\m@ne \ell \else \lsb@l \fi}}%
\endgroup

\newcommand{\B}{
    \begin{tikzpicture}
    \filldraw [fill=red, draw=black] (0, 0) rectangle (0.37, 0.45);
    \draw [line width=0.5mm, white ] (0.1,0.08) -- (0.1,0.38);
    \draw[line width=0.5mm, white ] (0.1, 0.35) .. controls (0.2, 0.35) and (0.4, 0.2625) .. (0.1, 0.225);
    \draw[line width=0.5mm, white ] (0.1, 0.225) .. controls (0.2, 0.225) and (0.4, 0.1625) .. (0.1, 0.1);
    \end{tikzpicture}
}

\DeclareMathOperator{\arcsinh}{arcsinh}

\author{Professor David Barrett\\ \small\textit{Transcribed by Thomas Cohn}}
\title{The Extended Reimann Integral}
\date{11/7/18} % Can also use \today

\begin{document}
\maketitle
\setlength\RaggedRightParindent{\parindent}
\RaggedRight

\par\noindent Consider $f\in{}C(A^{\ptxt{osso}\R^{n}},\R)$ (with $f$ and/or $A$ possibly unbounded). For now, assume $f\ge{}0$.\n

\defn{We define the \underline{Extended Reimann Integral} $\displaystyle\extint_{A}f\overset{\ptxt{def}}{=}\sup\set{\int_{E}f:E^{\ptxt{cpt},\ptxt{rect}}\subset{}A}$.}

\lemma{$\displaystyle{}B^{\ptxt{open}}\subset{}A^{\ptxt{open}}\Rightarrow\extint_{B}f\le\extint_{A}f$}

\par\noindent What if $A$ and $f$ are bounded?
\begin{enumerate}[label=(\roman*)]
	\item The old $\displaystyle\int_{A}{}f$ may not exist. (It exists if $A$ is rectifiable.)
	\item $\displaystyle\alignint{E}{}f=\lint{E}{}f\le\alignint{A}{}f\overset{\ptxt{def}}{=}\lint{Q}{}f_{A}$. So $\displaystyle\extalignint{A}{}f\le\lint{A}{}f$
	\item Let $P$ be a partition of $Q^{\ptxt{box}}\supset{}A$. Then $\displaystyle{}L(f_{A},P)\le\int_{\ptxt{union of $P$-boxes}}f\le\extint_{A}f$
	\item Thus, $\displaystyle\extalignint{A}{}f=\lint{A}{}f$. So $\displaystyle\lint{A}{}f\le\extalignint{A}{}f$. So $\displaystyle\extint_{A}f=\ptxt{old}\int_{A}f$ if $\displaystyle\ptxt{old}\int_{A}f$ exists.
\end{enumerate}

\par\noindent How do we compute?\n

\par\noindent Suppose we have an infinite sequence of compact rectifiable sets $E_{1}\subset{}E_{2}\subset{}E_{3}\subset\cdots\subset{}A$, and\n
$\displaystyle\bigcup_{j=1}^{\infty}\Int{}E_{j}=A$. Then we claim $\displaystyle\extint_{A}f=\lim_{j\to\infty}\int_{E_{j}}f$\n

\ex{$E_{j}=[-j,0]\cup[\frac{1}{j},j]$. Then $\displaystyle\bigcup_{j=1}^{\infty}E_{j}=\R$, and $\displaystyle\bigcup_{j=1}^{\infty}\Int{}E_{j}=\R\cut\set{0}$.}

\par\noindent Proof of claim: $\displaystyle\int_{E_{j}}\le\extint_{A}f$, so $\displaystyle\lim_{j\to\infty}\int_{E_{j}}f\le\extint_{A}f$.\n
If $E\subset{}A$ is compact and rectifiable, then $E\subset{}E_{j}$ for some $j$. Thus, $\displaystyle\int_{E}f\le\int_{E_{j}}f\le\lim_{j\to\infty}\int_{E_{j}}f$.\n
Therefore, $\displaystyle\extint_{E}f\le\lim_{j\to\infty}\int_{E_{j}}f$.\proven

\par\noindent Alternate proof (outline): Let $E_{j}$ be the union of all closed (hyper) cubes with side length $\frac{1}{2^{j}}$, subsets of $A$, with each vertex having coords in $\Z/2^{j}\cap[-j,j]$.\n

\ex{$\displaystyle\int_{\R}\frac{1}{1+x^{2}}$\n
Let $E_{j}=[-j,j]$. Then $\displaystyle\int_{\R}\frac{1}{1+x^{2}}=\lim_{j\to\infty}\int_{-j}^{j}\frac{1}{1+x^{2}}=\lim_{j\to\infty}\brack{\arctan{}x}_{x=-j}^{x=j}=\lim_{j\to\infty}2\arctan{j}=\pi$}

\ex{$\displaystyle\int_{\R^{2}}\frac{1}{1+x^{2}+y^{2}}=\lim_{j\to\infty}\int_{-j}^{j}\int_{-j}^{j}\frac{1}{1+x^{2}+y^{2}}dxdy=\cdots=\lim_{j\to\infty}\int_{-j}^{j}\frac{2\arctan\frac{j}{\sqrt{1+y^{2}}}}{\sqrt{1+y^{2}}}dy$. Ew.\n
$\displaystyle\int_{\R^{2}}\frac{1}{1+x^{2}+y^{2}}=\lim_{j\to\infty}\lim_{k\to\infty}\int_{\begin{subarray}{c}-k\le{}x\le{}k\\ -j\le{}y\le{}j\end{subarray}}\frac{1}{1+x^{2}+y^{2}}=\lim_{j\to\infty}\lim_{k\to\infty}\int_{-j}^{j}\int_{-k}^{k}\frac{1}{1+x^{2}+y^{2}}dxdy=$\n
$\displaystyle=\lim_{j\to\infty}\lim_{k\to\infty}\int_{-j}^{j}\frac{2\arctan\frac{k}{\sqrt{1+y^{2}}}}{\sqrt{1+y^{2}}}=\lim_{j\to\infty}\int_{-j}^{j}\frac{\pi}{\sqrt{1+y^{2}}}dy=\pi\lim_{j\to\infty}\int_{-j}^{j}\frac{1}{\sqrt{1+y^{2}}}dy=\overset{\ptxt{use }y=\sinh{}u}{\cdots}=$\n
$\displaystyle=\pi\lim_{j\to\infty}\brack{\arcsinh{}y}_{y=-j}^{y=j}=+\infty-(-\infty)=+\infty$}

\par\noindent We ``computed'' this extended integral.

\end{document}