\documentclass[10pt,letterpaper]{article}
\usepackage[utf8]{inputenc}
\usepackage{amsmath}
\usepackage{amsfonts}
\usepackage{amssymb}
\usepackage{ragged2e}
\usepackage[letterpaper, margin=1in]{geometry}
\usepackage{graphicx}
\usepackage{cancel}
\usepackage{mathtools}
\usepackage{tabularx}
\usepackage{arydshln}
\usepackage{tensor}
\usepackage{array}
\usepackage{xcolor}

%%%%%%%%%%%%%%%%%%%%%%%%%%%%%
% Formatting commands
%%%%%%%%%%%%%%%%%%%%%%%%%%%%%
\newcommand{\n}{\hfill\break}
\newcommand{\lemma}[1]{\par\noindent\settowidth{\hangindent}{\textbf{Lemma: }}\textbf{Lemma: }#1\n}
\newcommand{\defn}[1]{\par\noindent\settowidth{\hangindent}{\textbf{Defn: }}\textbf{Defn: }#1\n}
\newcommand{\thm}[1]{\par\noindent\settowidth{\hangindent}{\textbf{Thm: }}\textbf{Thm: }#1\n}
\newcommand{\prop}[1]{\par\noindent\settowidth{\hangindent}{\textbf{Prop: }}\textbf{Prop: }#1\n}
\newcommand{\ex}[1]{\par\noindent\settowidth{\hangindent}{\textbf{Ex: }}\textbf{Ex: }#1\n}
\newcommand{\proven}{\;$\square$\n}
\newcommand{\problem}[1]{\par\noindent{#1}\n}
\newcommand{\problempart}[2]{\par\settowidth{\hangindent}{\textbf{(#1)} \indent{}}\textbf{(#1)} #2\n}
\newcommand{\ptxt}[1]{\textrm{\textnormal{#1}}}
\newcommand{\inlineeq}[1]{\n\centerline{$\displaystyle #1$}}
\newcommand{\pageline}{\noindent\rule{\textwidth}{0.1pt}}

%%%%%%%%%%%%%%%%%%%%%%%%%%%%%
% Math commands
%%%%%%%%%%%%%%%%%%%%%%%%%%%%%
% Set Theory
\newcommand{\card}[1]{\left|#1\right|}
\newcommand{\set}[1]{\left\{#1\right\}}
\newcommand{\ps}[1]{\mathcal{P}\left(#1\right)}
\newcommand{\pfinite}[1]{\mathcal{P}^{\ptxt{finite}}\left(#1\right)}
\newcommand{\naturals}{\mathbb{N}}
\newcommand{\N}{\naturals}
\newcommand{\integers}{\mathbb{Z}}
\newcommand{\Z}{\integers}
\newcommand{\rationals}{\mathbb{Q}}
\newcommand{\Q}{\rationals}
\newcommand{\reals}{\mathbb{R}}
\newcommand{\R}{\reals}
\newcommand{\complex}{\mathbb{C}}
\newcommand{\C}{\complex}
\newcommand{\comp}{^{\complement}}

% Graph Theory
\renewcommand{\deg}[1]{\ptxt{deg}\left(#1\right)}
\newcommand{\degp}[1]{\ptxt{deg}^{+}\!\!\left(#1\right)}
\newcommand{\degn}[1]{\ptxt{deg}^{-}\!\!\left(#1\right)}

% Standard Math
\newcommand{\inv}{^{-1}}
\newcommand{\abs}[1]{\left|#1\right|}
\newcommand{\ceil}[1]{\left\lceil{}#1\right\rceil}
\newcommand{\floor}[1]{\left\lfloor{}#1\right\rfloor{}}
\newcommand{\conj}[1]{\overline{#1}}
\newcommand{\of}{\circ}
\newcommand{\tri}{\triangle}
\newcommand{\inj}{\hookrightarrow}
\newcommand{\surj}{\twoheadrightarrow}
\newcommand{\mapsfrom}{\mathrel{\reflectbox{\ensuremath{\mapsto}}}}

% Linear Algebra
\newcommand{\Id}{\textrm{\textnormal{Id}}}
\newcommand{\im}{\textrm{\textnormal{im}}}
\newcommand{\norm}[1]{\abs{\abs{#1}}}
\newcommand{\tpose}{^{T}}
\newcommand{\iprod}[1]{\left<#1\right>}
\newcommand{\trace}{\ptxt{tr}}
\newcommand{\chgBasMat}[3]{\!\!\tensor*[_{#1}]{\left[#2\right]}{_{#3}}}
\newcommand{\vecBas}[2]{\tensor*[]{\left[#1\right]}{_{#2}}}

% Topology
\newcommand{\closure}[1]{\bar{#1}}
\newcommand{\uball}{\mathcal{U}}
\newcommand{\Int}{\ptxt{Int}\>}
\newcommand{\Ext}{\ptxt{Ext}\>}
\newcommand{\Bd}{\ptxt{Bd}\>}

% Proofs
\newcommand{\st}{s.t.}
\newcommand{\unique}{!}

%%%%%%%%%%%%%%%%%%%%%%%%%%%%%
% Other commands
%%%%%%%%%%%%%%%%%%%%%%%%%%%%%
\newcommand{\flag}[1]{\textbf{\textcolor{red}{#1}}}

%%%%%%%%%%%%%%%%%%%%%%%%%%%%%
% Make l's curvy in math environments
%%%%%%%%%%%%%%%%%%%%%%%%%%%%%
\mathcode`l="8000
\begingroup
\makeatletter
\lccode`\~=`\l
\DeclareMathSymbol{\lsb@l}{\mathalpha}{letters}{`l}
\lowercase{\gdef~{\ifnum\the\mathgroup=\m@ne \ell \else \lsb@l \fi}}%
\endgroup

\author{Professor David Barrett\\ \small\textit{Transcribed by Thomas Cohn}}
\title{Connectedness}
\date{9/14/18} % Can also use \today

\begin{document}
\maketitle
\setlength\RaggedRightParindent{\parindent}
\RaggedRight

\prop{The Following are Equivalent (TFAE):\n
(1) There exists $f:X\to\set{0,1}$ continuous and surjective.\n
(2) There exists $A\subset{}X$ open and closed in $X$ with $\emptyset\ne{}A\ne{}X$.\n
\n
Proof:\n
(1)${}\to{}$(2) $A=f\inv{}[\set{1}]$. Clearly, $\emptyset\ne{}A\ne{}X$. $A$ is closed because $\set{1}$ is closed and $f$ is continuous. $A$ is open because $A\comp=X\setminus{}A=f\inv{}[\set{0}]$ is closed.\n
(2)${}\to{}$(1) Let $f=\mathbb{I}_{A}$ (the indicator function for $A$). $A\ne\emptyset$, and $A\ne{}X$, so $f$ is surjective. $f\inv{}[\set{1}]$ is open, $f\inv{}[\set{0}]$ is open, so $f$ is continuous.\n
\proven}

\defn{If this holds, we say that $X$ is \underline{disconnected}. We say $X$ is \underline{connected} if it is not disconnected.}

\prop{$[0,1]$ is connected. Proof: Let $f:[0,1]\to\set{0,1}$ be continuous. It's enough to show that $f$ is not surjective. Use the intermediate value theorem.\proven}

\defn{A topological space $X$ is said to be \underline{path-connected} $\leftrightarrow\forall\alpha,\beta\in{}X$, there is a continuous map $\varphi:[0,1]\to{}X$ with $\varphi(0)=\alpha$ and $\varphi(1)=\beta$.}

\prop{$X$ is path connected $\to$ $X$ is connected.\n
Proof: Suppose to the contrary $X$ is path-connected, but $\exists{}f:X\to\set{0,1}$ continuous and surjective. Pick $\alpha,\beta\in{}X$ with $f(\alpha)=0$, $f(\beta)=1$, and $\varphi$ as above. Then $(f\of\varphi):[0,1]\to\set{0,1}$ is also continuous and surjective. So then $[0,1]$ is disconnected. Oops!\proven}

\par\noindent Examples and Special Cases:\n
\begin{enumerate}
	\item $X\subset\R^{n}$ is convex $\to$ $X$ is path-connected $\to$ $X$ is connected.
	\item $X=\set{(x,y)\in\R^{2}:x>0,y=\sin\left(\frac{1}{x}\right)}\cup\set{(x,y)\in\R^{2}:x=0,-1\le{}y\le{}1}$ is connected, but \textit{not} path-connected (proven in a supplement, to be given later).
	\item $X\subset\R^{n}$ is open, connected $\to$ $X$ is path-connected (proved in HW 3).
\end{enumerate}

\par\noindent Given $x\in{}X$ metric space, $B\subset{}X$.\n
Set $d(x,B)=\inf\set{d(x,b):b\in{}B}$.\n
$d(x,B)>0\leftrightarrow{}x\in\Ext{}B$.\n
$d(x,B)=0\leftrightarrow{}x\not\in\Ext{}B\leftrightarrow{}x\in\Int{}B\cup\Bd{}B\leftrightarrow{}x\in\closure{B}$.\n

\par\noindent Fact: $X\to\R$, $x\mapsto{}d(x,B)$ is Lipschitz, and hence continuous (proved in HW 3).\n

\defn{Given $A,B\subset{}X$, $d(A,B)=\inf\set{d(a,B):a\in{}A}=\inf\set{d(a,b):a\in{}A,b\in{}B}$.}

\par\noindent $d(B,A)=d(A,B)$ and $d(A,B)\ge{}0$. But there's no triangle inequality!

\end{document}