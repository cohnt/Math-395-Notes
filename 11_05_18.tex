\documentclass[10pt,letterpaper]{article}
\usepackage[utf8]{inputenc}
\usepackage[intlimits]{amsmath}
\usepackage{amsfonts}
\usepackage{amssymb}
\usepackage{ragged2e}
\usepackage[letterpaper, margin=1in]{geometry}
\usepackage{graphicx}
\usepackage{cancel}
\usepackage{mathtools}
\usepackage{tabularx}
\usepackage{arydshln}
\usepackage{tensor}
\usepackage{array}
\usepackage{xcolor}
\usepackage[boxed]{algorithm}
\usepackage[noend]{algpseudocode}
\usepackage{listings}
\usepackage{textcomp}
\usepackage[pdf,tmpdir,singlefile]{graphviz}
\usepackage{mathrsfs}
\usepackage{bbm}
\usepackage{tikz}
\usepackage{enumitem}
\usepackage{arydshln}

%%%%%%%%%%%%%%%%%%%%%%%%%%%%%
% Formatting commands
%%%%%%%%%%%%%%%%%%%%%%%%%%%%%
\newcommand{\n}{\hfill\break}
\newcommand{\up}{\vspace{-\baselineskip}}
\newcommand{\lemma}[1]{\par\noindent\settowidth{\hangindent}{\textbf{Lemma: }}\textbf{Lemma: }#1}
\newcommand{\defn}[1]{\par\noindent\settowidth{\hangindent}{\textbf{Defn: }}\textbf{Defn: }#1\n}
\newcommand{\thm}[1]{\par\noindent\settowidth{\hangindent}{\textbf{Thm: }}\textbf{Thm: }#1\n}
\newcommand{\prop}[1]{\par\noindent\settowidth{\hangindent}{\textbf{Prop: }}\textbf{Prop: }#1\n}
\newcommand{\cor}[1]{\par\noindent\settowidth{\hangindent}{\textbf{Cor: }}\textbf{Cor: }#1\n}
\newcommand{\ex}[1]{\par\noindent\settowidth{\hangindent}{\textbf{Ex: }}\textbf{Ex: }#1\n}
\newcommand{\proven}{\;$\square$\n}
\newcommand{\problem}[1]{\par\noindent{#1}\n}
\newcommand{\problempart}[2]{\par\noindent\indent{}\settowidth{\hangindent}{\textbf{(#1)} \indent{}}\textbf{(#1)} #2\n}
\newcommand{\ptxt}[1]{\textrm{\textnormal{#1}}}
\newcommand{\inlineeq}[1]{\centerline{$\displaystyle #1$}}
\newcommand{\pageline}{\noindent\rule{\textwidth}{0.1pt}}

%%%%%%%%%%%%%%%%%%%%%%%%%%%%%
% Math commands
%%%%%%%%%%%%%%%%%%%%%%%%%%%%%
% Set Theory
\newcommand{\card}[1]{\left|#1\right|}
\newcommand{\set}[1]{\left\{#1\right\}}
\newcommand{\ps}[1]{\mathcal{P}\left(#1\right)}
\newcommand{\pfinite}[1]{\mathcal{P}^{\ptxt{finite}}\left(#1\right)}
\newcommand{\naturals}{\mathbb{N}}
\newcommand{\N}{\naturals}
\newcommand{\integers}{\mathbb{Z}}
\newcommand{\Z}{\integers}
\newcommand{\rationals}{\mathbb{Q}}
\newcommand{\Q}{\rationals}
\newcommand{\reals}{\mathbb{R}}
\newcommand{\R}{\reals}
\newcommand{\complex}{\mathbb{C}}
\newcommand{\C}{\complex}
\newcommand{\comp}{^{\complement}}
\DeclareMathOperator{\Hom}{Hom}
\newcommand{\Ind}{\mathbbm{1}}
\newcommand{\cut}{\setminus}

% Graph Theory
\let\deg\relax
\DeclareMathOperator{\deg}{deg}
\newcommand{\degp}{\ptxt{deg}^{+}}
\newcommand{\degn}{\ptxt{deg}^{-}}
\newcommand{\precdot}{\mathrel{\ooalign{$\prec$\cr\hidewidth\hbox{$\cdot\mkern0.5mu$}\cr}}}
\newcommand{\succdot}{\mathrel{\ooalign{$\cdot\mkern0.5mu$\cr\hidewidth\hbox{$\succ$}\cr\phantom{$\succ$}}}}
\DeclareMathOperator{\cl}{cl}
\DeclareMathOperator{\affdim}{affdim}

% Probability
\newcommand{\Prob}{\mathbb{P}}
\newcommand{\Avg}{\mathbb{E}}

% Standard Math
\newcommand{\inv}{^{-1}}
\newcommand{\abs}[1]{\left|#1\right|}
\newcommand{\ceil}[1]{\left\lceil{}#1\right\rceil{}}
\newcommand{\floor}[1]{\left\lfloor{}#1\right\rfloor{}}
\newcommand{\conj}[1]{\overline{#1}}
\newcommand{\of}{\circ}
\newcommand{\tri}{\triangle}
\newcommand{\inj}{\hookrightarrow}
\newcommand{\surj}{\twoheadrightarrow}
\newcommand{\mapsfrom}{\mathrel{\reflectbox{\ensuremath{\mapsto}}}}
\newcommand{\ndiv}{\nmid}
\renewcommand{\epsilon}{\varepsilon}
\newcommand{\divides}{\mid}
\newcommand{\ndivdies}{\nmid}
\DeclareMathOperator{\lcm}{lcm}

% Linear Algebra
\newcommand{\Id}{\textrm{\textnormal{Id}}}
\newcommand{\im}{\textrm{\textnormal{im}}}
\newcommand{\norm}[1]{\abs{\abs{#1}}}
\newcommand{\tpose}{^{T}}
\newcommand{\iprod}[1]{\left<#1\right>}
\DeclareMathOperator{\trace}{tr}
\newcommand{\chgBasMat}[3]{\!\!\tensor*[_{#1}]{\left[#2\right]}{_{#3}}}
\newcommand{\vecBas}[2]{\tensor*[]{\left[#1\right]}{_{#2}}}
\DeclareMathOperator{\GL}{GL}
\DeclareMathOperator{\Mat}{Mat}
\DeclareMathOperator{\vspan}{span}
\DeclareMathOperator{\rank}{rank}
\newcommand{\V}[1]{\vec{#1}}

% Topology
\newcommand{\closure}[1]{\overline{#1}}
\newcommand{\uball}{\mathcal{U}}
\DeclareMathOperator{\Int}{Int}
\DeclareMathOperator{\Ext}{Ext}
\DeclareMathOperator{\Bd}{Bd}
\DeclareMathOperator{\rInt}{rInt}
\DeclareMathOperator{\ch}{ch}
\DeclareMathOperator{\ah}{ah}

% Analysis
\DeclareMathOperator{\Graph}{Graph}
\DeclareMathOperator{\epi}{epi}
\DeclareMathOperator{\hypo}{hypo}
\DeclareMathOperator{\supp}{supp}
\newcommand{\lint}[2]{\underset{#1}{\overset{#2}{{\color{black}\underline{{\color{white}\overline{{\color{black}\int}}\color{black}}}}}}}
\newcommand{\uint}[2]{\underset{#1}{\overset{#2}{{\color{white}\underline{{\color{black}\overline{{\color{black}\int}}\color{black}}}}}}}
\newcommand{\alignint}[2]{\underset{#1}{\overset{#2}{{\color{white}\underline{{\color{white}\overline{{\color{black}\int}}\color{black}}}}}}}
\newcommand{\extint}{\ptxt{ext}\int}
\newcommand{\alignextint}[2]{\ptxt{ext}\alignint{#1}{#2}}
\newcommand{\conv}{\ast}

% Proofs
\newcommand{\st}{s.t.}
\newcommand{\unique}{!}

% Brackets
\newcommand{\paren}[1]{\left(#1\right)}
\renewcommand{\brack}[1]{\left[#1\right]}
\renewcommand{\brace}[1]{\left\{#1\right\}}
\newcommand{\ang}[1]{\left<#1\right>}

% Algorithms
\algrenewcommand{\algorithmiccomment}[1]{\hskip 1em \texttt{// #1}}
\algrenewcommand\algorithmicrequire{\textbf{Input:}}
\algrenewcommand\algorithmicensure{\textbf{Output:}}
\newcommand{\parSymbol}{\P}
\renewcommand{\P}{\ptxt{\textbf{P}}}
\newcommand{\NP}{\ptxt{\textbf{NP}}}
\newcommand{\NPC}{\ptxt{\textbf{NP-Complete}}}
\newcommand{\NPH}{\ptxt{\textbf{NP-Hard}}}
\newcommand{\EXP}{\ptxt{\textbf{EXP}}}

%%%%%%%%%%%%%%%%%%%%%%%%%%%%%
% Other commands
%%%%%%%%%%%%%%%%%%%%%%%%%%%%%
\newcommand{\flag}[1]{\textbf{\textcolor{red}{#1}}}

%%%%%%%%%%%%%%%%%%%%%%%%%%%%%
% Make l's curvy in math environments
%%%%%%%%%%%%%%%%%%%%%%%%%%%%%
\mathcode`l="8000
\begingroup
\makeatletter
\lccode`\~=`\l
\DeclareMathSymbol{\lsb@l}{\mathalpha}{letters}{`l}
\lowercase{\gdef~{\ifnum\the\mathgroup=\m@ne \ell \else \lsb@l \fi}}%
\endgroup

\newcommand{\B}{
    \begin{tikzpicture}
    \filldraw [fill=red, draw=black] (0, 0) rectangle (0.37, 0.45);
    \draw [line width=0.5mm, white ] (0.1,0.08) -- (0.1,0.38);
    \draw[line width=0.5mm, white ] (0.1, 0.35) .. controls (0.2, 0.35) and (0.4, 0.2625) .. (0.1, 0.225);
    \draw[line width=0.5mm, white ] (0.1, 0.225) .. controls (0.2, 0.225) and (0.4, 0.1625) .. (0.1, 0.1);
    \end{tikzpicture}
}

\author{Thomas Cohn}
\title{Rectifiable Sets}
\date{11/5/18} % Can also use \today

\begin{document}
\maketitle
\setlength\RaggedRightParindent{\parindent}
\RaggedRight

\par\noindent Recall from Friday that if $S^{\ptxt{bdd}}\subset\R^{n}$ and $f:S\to\R$ is a bounded function, then we say $f$ is integrable over $S$ if and only if $\int_{S}f=\int_{Q}f_{S}$ is defined for some/all $Q^{\ptxt{box}}\supset{}S$.\n

\par\noindent Some rules:
\begin{enumerate}[label=(\alph*)]
	\item $f,g$ integrable over $S$ implies that $\displaystyle\int_{S}af+bg=a\int_{S}f+b\int_{S}g$
	\item $f,g$ integrable over $S$, and $f\le{}g$ on $S$ implies that $\displaystyle\int_{S}f\le\int_{S}g$
	\begin{enumerate}[label=(\alph*$'$),start=2]
		\item $f$ integrable over $S$ implies that $\abs{f}$ is integrable over $S$.\n
		Also, $\displaystyle\abs{\int_{S}f}=\max\set{\int_{S}f,-\int_{S}f}\le\int_{S}\abs{f}$
	\end{enumerate}
	\item $T\subseteq{}S$, $f\ge{}0$ integrable on $T,S$ implies that $\displaystyle\int_{T}f\le\int_{S}f$
	\item $f$ integrable over $S_{1}$ and $S_{2}$ implies that $f$ is integrable over $S_{1}\cup{}S_{2}$ and $S_{1}\cap{}S_{2}$, and\n
	$\displaystyle\int_{S_{1}\cup{}S_{2}}f=\int_{S_{1}}f+\int_{S_{2}}f-\int_{S_{1}\cap{}S_{2}}f$
\end{enumerate}

\par\noindent Proof (a): Let $A=\set{(x,y)\in\R^{2}:x=0\lor{}y=0\lor{}x=y}$. Define $\varphi:A\to\R$ by $(x,0)\mapsto{}x$, $(0,y)\mapsto{}y$, and $(x,x)\mapsto{}x$.\n

\par\noindent Exercise 1: Show that $\varphi$ is continuous.\n
Exercise 2: Show that $\varphi\of(f_{S_{1}},f_{S_{2}})=f_{S_{1}\cup{}S_{2}}$. This tells us that $f_{S_{1}\cup{}S_{2}}$ is continuous at points where $f_{S_{1}}$ and $f_{S_{2}}$ are continuous. Hence, $f_{S_{1}\cup{}S_{2}}$ is cotntinuous.\n

\par\noindent Now, use $f_{S_{1}\cup{}S_{2}}+f_{S_{1}\cap{}S_{2}}=f_{S_{1}}+f_{S_{2}}$.\n

\newpage
\ex{$S=\set{(x_{1},x_{2},x_{3}):x_{1}\ge{}0,x_{2}\ge{}0,x_{3}\ge{}0,x_{1}+x_{2}+x_{3}\le{}1}$. Let $Q=[0,1]\times[0,1]\times[0,1]$.\n
\n
Check that $\int_{S}1$ exists (use study exercise)
\begin{align*}
\int_{S}1=\int_{Q}\Ind_{S} & =\int_{0\le{}x_{1}\le{}1}\paren{\;\int_{0\le{}x_{2}\le{}1}\paren{\;\int_{0\le{}x_{3}\le{}1}\Ind_{S}}}\\
                           & =\int_{0\le{}x_{1}\le{}1}\paren{\;\int_{0\le{}x_{2}\le{}1}\max\set{1-x_{1}-x_{2},0}}\\
                           & =\int_{0\le{}x_{1}\le{}1}\paren{\;\int_{0\le{}x_{2}\le{}1-x_{1}}1-x_{1}-x_{2}}\\
                           & =\int_{0\le{}x_{1}\le{}1}\brack{(1-x_{1})x_{2}-\frac{x_{2}^{2}}{2}}_{x_{2}=0}^{x_{2}=1-x_{1}}\\
                           & =\int_{0\le{}x_{1}\le{}1}\frac{(1-x_{1})^{2}}{2}\\
                           & =\brack{-\frac{(1-x_{1})^{3}}{6}}_{x_{1}=0}^{x_{1}=1}\\
                           & =\frac{1}{6}
\end{align*}}

\thm{Given $S^{\ptxt{bdd}}\subset\R^{n}$, $f:S\to\R$ bounded and continuous, $\displaystyle{}E\overset{\ptxt{def}}{=}\set{\vec{x_{0}}\in\Bd{}S:\ptxt{it is false that }\lim_{\vec{x}\to\vec{x_{0}}\;(\vec{x}\in{}S)}f(\vec{x})=0}$, and $m^{*}(E)=0$, then $f$ is integrable on $S$.\n
Proof: $\mathscr{D}f_{S}\subset{}E$.\proven}

\cor{Given $S^{\ptxt{bdd}}$, $f:S\to\R$ bounded and continuous, and $m^{*}(\Bd{}S)=0$, then $f$ is integrable over $S$.}

\par\noindent Let's further study the condition that $m^{*}(\Bd{}S)=0$:

\defn{$S$ is \underline{rectifiable}\n
$\overset{\ptxt{def}}{\Leftrightarrow}1$ integrable over $S$\n
$\Leftrightarrow\Ind_{S}$ integrable on $Q^{\ptxt{box}}\supset{}S$\n
$\Leftrightarrow{}m^{*}(\Bd{}S)=0\overset{\ptxt{Cor}}{\to}$ all bounded $f\in{}C(S,\R)$ are integrable over $S$\n
$\Leftrightarrow{}m^{*,J}(\Bd{}S)=0$}

\par\noindent For $S$ rectifiable, we define $v(S)=\int_{S}1$\n

\par\noindent $S$ rectifiable, $A=\Int{}S\Rightarrow\Bd{}A\subset\Bd{}S$, $m^{*}(\Bd{}A)\le{}m^{*}(\Bd{}S)=0$. This implies that $\Ind_{S},\Ind_{A}$ are integrable. So $\Ind_{S\cut{}A}=\Ind_{S}-\Ind_{A}$ is integrable (on $Q$). Thus, $S\cut{}A$ is rectifiable, and $\Int(S\cut{}A)=\emptyset$.\n
All $L(\Ind_{S\cut{}A},P)=0$, so $\displaystyle\lint{Q}{}\Ind_{S\cut{}A}=0$. Thus, $\displaystyle\int\Ind_{A}=\int\Ind_{S}\Rightarrow{}v(A)=v(S)$.\n

\par\noindent But what if $S$ and or $f$ are not bounded? \textbf{Improper Integrals}\n

\par\noindent Munkres starts to focus on integrals over open sets. Start with $f\ge{}0$.\n

\defn{The \underline{extended integral} of $f$ over set $A$, $\displaystyle\extint_{A}f\overset{\ptxt{def}}{=}\sup\set{\int_{E}f:E\ptxt{ cpt, rect}}$}

\end{document}