\documentclass[10pt,letterpaper]{article}
\usepackage[utf8]{inputenc}
\usepackage[intlimits]{amsmath}
\usepackage{amsfonts}
\usepackage{amssymb}
\usepackage{ragged2e}
\usepackage[letterpaper, margin=1in]{geometry}
\usepackage{graphicx}
\usepackage{cancel}
\usepackage{mathtools}
\usepackage{tabularx}
\usepackage{arydshln}
\usepackage{tensor}
\usepackage{array}
\usepackage{xcolor}
\usepackage[boxed]{algorithm}
\usepackage[noend]{algpseudocode}
\usepackage{listings}
\usepackage{textcomp}
\usepackage[pdf,tmpdir,singlefile]{graphviz}
\usepackage{mathrsfs}
\usepackage{bbm}
\usepackage{tikz}
\usepackage{enumitem}
\usepackage{arydshln}

%%%%%%%%%%%%%%%%%%%%%%%%%%%%%
% Formatting commands
%%%%%%%%%%%%%%%%%%%%%%%%%%%%%
\newcommand{\n}{\hfill\break}
\newcommand{\up}{\vspace{-\baselineskip}}
\newcommand{\lemma}[1]{\par\noindent\settowidth{\hangindent}{\textbf{Lemma: }}\textbf{Lemma: }#1}
\newcommand{\defn}[1]{\par\noindent\settowidth{\hangindent}{\textbf{Defn: }}\textbf{Defn: }#1\n}
\newcommand{\thm}[1]{\par\noindent\settowidth{\hangindent}{\textbf{Thm: }}\textbf{Thm: }#1\n}
\newcommand{\prop}[1]{\par\noindent\settowidth{\hangindent}{\textbf{Prop: }}\textbf{Prop: }#1\n}
\newcommand{\cor}[1]{\par\noindent\settowidth{\hangindent}{\textbf{Cor: }}\textbf{Cor: }#1\n}
\newcommand{\ex}[1]{\par\noindent\settowidth{\hangindent}{\textbf{Ex: }}\textbf{Ex: }#1\n}
\newcommand{\proven}{\;$\square$\n}
\newcommand{\problem}[1]{\par\noindent{#1}\n}
\newcommand{\problempart}[2]{\par\noindent\indent{}\settowidth{\hangindent}{\textbf{(#1)} \indent{}}\textbf{(#1)} #2\n}
\newcommand{\ptxt}[1]{\textrm{\textnormal{#1}}}
\newcommand{\inlineeq}[1]{\centerline{$\displaystyle #1$}}
\newcommand{\pageline}{\noindent\rule{\textwidth}{0.1pt}}

%%%%%%%%%%%%%%%%%%%%%%%%%%%%%
% Math commands
%%%%%%%%%%%%%%%%%%%%%%%%%%%%%
% Set Theory
\newcommand{\card}[1]{\left|#1\right|}
\newcommand{\set}[1]{\left\{#1\right\}}
\newcommand{\ps}[1]{\mathcal{P}\left(#1\right)}
\newcommand{\pfinite}[1]{\mathcal{P}^{\ptxt{finite}}\left(#1\right)}
\newcommand{\naturals}{\mathbb{N}}
\newcommand{\N}{\naturals}
\newcommand{\integers}{\mathbb{Z}}
\newcommand{\Z}{\integers}
\newcommand{\rationals}{\mathbb{Q}}
\newcommand{\Q}{\rationals}
\newcommand{\reals}{\mathbb{R}}
\newcommand{\R}{\reals}
\newcommand{\complex}{\mathbb{C}}
\newcommand{\C}{\complex}
\newcommand{\comp}{^{\complement}}
\DeclareMathOperator{\Hom}{Hom}
\newcommand{\Ind}{\mathbbm{1}}
\newcommand{\cut}{\setminus}

% Graph Theory
\let\deg\relax
\DeclareMathOperator{\deg}{deg}
\newcommand{\degp}{\ptxt{deg}^{+}}
\newcommand{\degn}{\ptxt{deg}^{-}}
\newcommand{\precdot}{\mathrel{\ooalign{$\prec$\cr\hidewidth\hbox{$\cdot\mkern0.5mu$}\cr}}}
\newcommand{\succdot}{\mathrel{\ooalign{$\cdot\mkern0.5mu$\cr\hidewidth\hbox{$\succ$}\cr\phantom{$\succ$}}}}
\DeclareMathOperator{\cl}{cl}
\DeclareMathOperator{\affdim}{affdim}

% Probability
\newcommand{\Prob}{\mathbb{P}}
\newcommand{\Avg}{\mathbb{E}}

% Standard Math
\newcommand{\inv}{^{-1}}
\newcommand{\abs}[1]{\left|#1\right|}
\newcommand{\ceil}[1]{\left\lceil{}#1\right\rceil{}}
\newcommand{\floor}[1]{\left\lfloor{}#1\right\rfloor{}}
\newcommand{\conj}[1]{\overline{#1}}
\newcommand{\of}{\circ}
\newcommand{\tri}{\triangle}
\newcommand{\inj}{\hookrightarrow}
\newcommand{\surj}{\twoheadrightarrow}
\newcommand{\mapsfrom}{\mathrel{\reflectbox{\ensuremath{\mapsto}}}}
\newcommand{\mapsdown}{\rotatebox[origin=c]{-90}{$\mapsto$}\mkern2mu}
\newcommand{\mapsup}{\rotatebox[origin=c]{90}{$\mapsto$}\mkern2mu}
\newcommand{\ndiv}{\nmid}
\renewcommand{\epsilon}{\varepsilon}
\newcommand{\divides}{\mid}
\newcommand{\ndivides}{\nmid}
\DeclareMathOperator{\lcm}{lcm}

% Linear Algebra
\newcommand{\Id}{\textrm{\textnormal{Id}}}
\newcommand{\im}{\textrm{\textnormal{im}}}
\newcommand{\norm}[1]{\abs{\abs{#1}}}
\newcommand{\tpose}{^{T}}
\newcommand{\iprod}[1]{\left<#1\right>}
\DeclareMathOperator{\trace}{tr}
\newcommand{\chgBasMat}[3]{\!\!\tensor*[_{#1}]{\left[#2\right]}{_{#3}}}
\newcommand{\vecBas}[2]{\tensor*[]{\left[#1\right]}{_{#2}}}
\DeclareMathOperator{\GL}{GL}
\DeclareMathOperator{\Mat}{Mat}
\DeclareMathOperator{\vspan}{span}
\DeclareMathOperator{\rank}{rank}
\newcommand{\V}[1]{\vec{#1}}

% Topology
\newcommand{\closure}[1]{\overline{#1}}
\newcommand{\uball}{\mathcal{U}}
\DeclareMathOperator{\Int}{Int}
\DeclareMathOperator{\Ext}{Ext}
\DeclareMathOperator{\Bd}{Bd}
\DeclareMathOperator{\rInt}{rInt}
\DeclareMathOperator{\ch}{ch}
\DeclareMathOperator{\ah}{ah}

% Analysis
\DeclareMathOperator{\Graph}{Graph}
\DeclareMathOperator{\epi}{epi}
\DeclareMathOperator{\hypo}{hypo}
\DeclareMathOperator{\supp}{supp}
\newcommand{\lint}[2]{\underset{#1}{\overset{#2}{{\color{black}\underline{{\color{white}\overline{{\color{black}\int}}\color{black}}}}}}}
\newcommand{\uint}[2]{\underset{#1}{\overset{#2}{{\color{white}\underline{{\color{black}\overline{{\color{black}\int}}\color{black}}}}}}}
\newcommand{\alignint}[2]{\underset{#1}{\overset{#2}{{\color{white}\underline{{\color{white}\overline{{\color{black}\int}}\color{black}}}}}}}
\newcommand{\extint}{\ptxt{ext}\int}
\newcommand{\extalignint}[2]{\ptxt{ext}\alignint{#1}{#2}}
\newcommand{\conv}{\ast}

% Proofs
\newcommand{\st}{s.t.}
\newcommand{\unique}{!}
\newcommand{\iffdef}{\overset{\ptxt{def}}{\Leftrightarrow}}
\newcommand{\eqdef}{\overset{\ptxt{def}}{=}}

% Brackets
\newcommand{\paren}[1]{\left(#1\right)}
\renewcommand{\brack}[1]{\left[#1\right]}
\renewcommand{\brace}[1]{\left\{#1\right\}}
\newcommand{\ang}[1]{\left<#1\right>}

% Algorithms
\algrenewcommand{\algorithmiccomment}[1]{\hskip 1em \texttt{// #1}}
\algrenewcommand\algorithmicrequire{\textbf{Input:}}
\algrenewcommand\algorithmicensure{\textbf{Output:}}
\newcommand{\parSymbol}{\P}
\renewcommand{\P}{\ptxt{\textbf{P}}}
\newcommand{\NP}{\ptxt{\textbf{NP}}}
\newcommand{\NPC}{\ptxt{\textbf{NP-Complete}}}
\newcommand{\NPH}{\ptxt{\textbf{NP-Hard}}}
\newcommand{\EXP}{\ptxt{\textbf{EXP}}}

%%%%%%%%%%%%%%%%%%%%%%%%%%%%%
% Other commands
%%%%%%%%%%%%%%%%%%%%%%%%%%%%%
\newcommand{\flag}[1]{\textbf{\textcolor{red}{#1}}}

%%%%%%%%%%%%%%%%%%%%%%%%%%%%%
% Make l's curvy in math environments
%%%%%%%%%%%%%%%%%%%%%%%%%%%%%
\mathcode`l="8000
\begingroup
\makeatletter
\lccode`\~=`\l
\DeclareMathSymbol{\lsb@l}{\mathalpha}{letters}{`l}
\lowercase{\gdef~{\ifnum\the\mathgroup=\m@ne \ell \else \lsb@l \fi}}%
\endgroup

\newcommand{\B}{
    \begin{tikzpicture}
    \filldraw [fill=red, draw=black] (0, 0) rectangle (0.37, 0.45);
    \draw [line width=0.5mm, white ] (0.1,0.08) -- (0.1,0.38);
    \draw[line width=0.5mm, white ] (0.1, 0.35) .. controls (0.2, 0.35) and (0.4, 0.2625) .. (0.1, 0.225);
    \draw[line width=0.5mm, white ] (0.1, 0.225) .. controls (0.2, 0.225) and (0.4, 0.1625) .. (0.1, 0.1);
    \end{tikzpicture}
}

\author{Thomas Cohn}
\title{The First Fundamental Theorem of Calculus for $1$-Forms (Part b)}
\date{11/30/18} % Can also use \today

\begin{document}
\maketitle
\setlength\RaggedRightParindent{\parindent}
\RaggedRight

\par\noindent Recall from Wednesday:\n
$1$-form $\omega=\omega_{1}dx_{1}+\cdots+\omega_{n}dx_{n}$ for $\omega_{i}$ scalar functions. Then\n
$\omega$ is \underline{closed} $\iffdef{}D_{k}\omega_{j}=D_{j}\omega_{k}$.\n
${}\qquad\Leftarrow\omega$ is \underline{exact} $\iffdef\omega=df$\n
$\displaystyle{}\qquad\overset{\ptxt{FTC1a}}{\Leftrightarrow}\int_{Y_{\alpha}}\omega=0$ when $\alpha\in{}C_{pw}^{2}([a,b],A)$, and $\alpha(a)=\alpha(b)$.\n

\thm{FTC1b for $1$-forms\n
$\omega$ closed $1$-form on $A\subseteq\R^{n}$ open and convex $\Rightarrow$ $\omega$ is exact on $A$.}

\lemma{(1) $\omega$ $C^{1}$ closed $1$-form, $\alpha$ $C^{1}$ map $\Rightarrow$ $\alpha^{*}\omega$ closed.}

\lemma{(2) $\omega$ $C^{1}$ $1$-form on open set containing $R^{\ptxt{box}}\subseteq\R^{2}$ $\Rightarrow$ $\displaystyle\int_{\Bd{}R\ptxt{(counterclockwise)}}\omega=\int_{R}(D_{1}\omega_{2}-D_{2}\omega_{1})$}

\cor{Also assume $\omega$ closed. Then $\int_{\Bd{}R}\omega=0$.}

\ex{$\omega=\frac{-x_{2}}{x_{1}^{2}+x_{2}^{2}}dx_{1}+\frac{x_{1}}{x_{1}^{2}+x_{2}^{2}}dx_{2}$\n
Exercise: $\omega$ closed on $\R^{2}\cut\set{\vec{0}}$\n
Exercise: $\omega=d(\arctan\frac{y}{x})$ on $(0,+\infty)\times\R$\n
Part for $\alpha:[0,2\pi]\to\R^{2}$, $t\mapsto(\cos{}t,\sin{}t)$, have $\displaystyle\int_{Y_{\alpha}}\omega=\int_{0}^{2\pi}-\sin{}td\cos{}t+\cos{}td\sin{}t=\int_{0}^{2\pi}1dt=2\pi\ne{}0$.\n
Hence, $\omega$ is not exact.}

\par\noindent Proof of lemma 2: $\displaystyle\int_{\Bd{}R}\omega=\int_{a_{1}}^{b_{1}}\omega_{1}(x_{1},a_{2})dx_{1}+\int_{a_{2}}^{b_{2}}\omega_{2}(b_{1},x_{2})dx_{2}-\int_{a_{1}}^{b_{1}}\omega_{1}(x_{1},b_{2})dx_{1}-\int_{a_{2}}^{b_{2}}\omega_{2}(a_{1},x_{2})dx_{2}=$\n
$\displaystyle=-\int_{a_{1}}^{b_{1}}\int_{a_{2}}^{b_{2}}D_{2}\omega_{1}(x_{1},x_{2})dx_{2}dx_{1}+\ptxt{(reverse)}=\int_{R}D_{1}\omega_{2}-D_{2}\omega_{1}$. \checkmark\n

\par\noindent Proof of FTC1b\n
Check that $\displaystyle\int_{Y_{\alpha}}\omega=0$ when $\alpha\in{}C_{pw}^{2}([a,b],A)$, $\alpha(a)=\alpha(b)$.\n
Define $\tilde{\alpha}:[a,b]\times[0,1]\to{}A$. $\tilde{\alpha}$ is affine on each vertical line segment. $\tilde{\alpha}$ is $C^{2}$ on each subbox $R_{j}$.\n
So $\displaystyle\int_{\Bd{}R_{j}}\tilde{\alpha}^{*}\omega=0$ by lemma 2 corollary. Thus $\displaystyle{}0=\sum\int_{\Bd{}R_{j}}\tilde{\alpha}^{*}\omega=\int_{\Bd{}R}\tilde{\alpha}^{*}\omega=\int_{[a,b]}\alpha^{*}\omega$.

\par\noindent Remark: FTC1b also works for $A$ $C^{2}$-diffeomorphic to a convex set.
\begin{align*}
\omega\ptxt{ closed on }A\xRightarrow{\ptxt{Lemma 1}} & \gamma^{*}\omega\ptxt{ closed on }B\\
\xRightarrow{\ptxt{FTC1b}} & \gamma^{*}\omega=df\ptxt{ on }B\\
\Rightarrow & \beta^{*}\gamma^{*}\omega=\beta^{*}df\\
\Rightarrow & \omega=(\gamma\of\beta)^{*}\omega{}d\beta^{*}f
\end{align*}

\par\noindent Hence, $\omega$ closed but \textit{not} exact on $\R^{2}\cut\set{\vec{0}}$, so $\R^{2}\cut\set{\vec{0}}$ is not diffeomorphic to a convex set.\n

\thm{$\exists{}E\subset[0,1]$ such that
\begin{enumerate}[label=(\arabic*),leftmargin=1.5cm]
	\item $t_{1},t_{2}$ distinct rational numbers $\Rightarrow$ $(E+t_{1})\cap(E+t_{2})=\emptyset$
	\item $\displaystyle\R=\bigcup_{t\in\Q}(E+t)$
\end{enumerate}\up\n
Proof: $\Q$ is a subgroup of $\R$. So we get an equivalence relation on $\R$: $x\sim{}y\Leftrightarrow{}x-y\in\Q$. Equivalence classes are called \underline{cosets}. Thus $\R$ is the disjoint union of cosets, where each coset is dense, and each coset $C$ can be written as $C=\Q+x$ for some $x\in{}C\cap[0,1]$. For each coset, pick such an $x$ (we can do this because of the axiom of choice).\n
For every $y\in\R$, $y$ has a unique representation $y=x+t$ for $x\in{}E$, $t\in\Q$.\n
Therefore, $\displaystyle\R=\bigcup_{t\in\Q}(E+t)$.\proven}

\par\noindent $A^{\ptxt{osso}\R^{k}}\overset{\alpha}{\to}M\subset\R^{n}$, $\alpha\in{}C^{r}$, and $\alpha$ injective. Then each $D\alpha(\vec{x})$ has maximal rank.

\end{document}