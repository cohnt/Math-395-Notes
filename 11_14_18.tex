\documentclass[10pt,letterpaper]{article}
\usepackage[utf8]{inputenc}
\usepackage[intlimits]{amsmath}
\usepackage{amsfonts}
\usepackage{amssymb}
\usepackage{ragged2e}
\usepackage[letterpaper, margin=1in]{geometry}
\usepackage{graphicx}
\usepackage{cancel}
\usepackage{mathtools}
\usepackage{tabularx}
\usepackage{arydshln}
\usepackage{tensor}
\usepackage{array}
\usepackage{xcolor}
\usepackage[boxed]{algorithm}
\usepackage[noend]{algpseudocode}
\usepackage{listings}
\usepackage{textcomp}
\usepackage[pdf,tmpdir,singlefile]{graphviz}
\usepackage{mathrsfs}
\usepackage{bbm}
\usepackage{tikz}
\usepackage{enumitem}
\usepackage{arydshln}

%%%%%%%%%%%%%%%%%%%%%%%%%%%%%
% Formatting commands
%%%%%%%%%%%%%%%%%%%%%%%%%%%%%
\newcommand{\n}{\hfill\break}
\newcommand{\up}{\vspace{-\baselineskip}}
\newcommand{\lemma}[1]{\par\noindent\settowidth{\hangindent}{\textbf{Lemma: }}\textbf{Lemma: }#1}
\newcommand{\defn}[1]{\par\noindent\settowidth{\hangindent}{\textbf{Defn: }}\textbf{Defn: }#1\n}
\newcommand{\thm}[1]{\par\noindent\settowidth{\hangindent}{\textbf{Thm: }}\textbf{Thm: }#1\n}
\newcommand{\prop}[1]{\par\noindent\settowidth{\hangindent}{\textbf{Prop: }}\textbf{Prop: }#1\n}
\newcommand{\cor}[1]{\par\noindent\settowidth{\hangindent}{\textbf{Cor: }}\textbf{Cor: }#1\n}
\newcommand{\ex}[1]{\par\noindent\settowidth{\hangindent}{\textbf{Ex: }}\textbf{Ex: }#1\n}
\newcommand{\proven}{\;$\square$\n}
\newcommand{\problem}[1]{\par\noindent{#1}\n}
\newcommand{\problempart}[2]{\par\noindent\indent{}\settowidth{\hangindent}{\textbf{(#1)} \indent{}}\textbf{(#1)} #2\n}
\newcommand{\ptxt}[1]{\textrm{\textnormal{#1}}}
\newcommand{\inlineeq}[1]{\centerline{$\displaystyle #1$}}
\newcommand{\pageline}{\noindent\rule{\textwidth}{0.1pt}}

%%%%%%%%%%%%%%%%%%%%%%%%%%%%%
% Math commands
%%%%%%%%%%%%%%%%%%%%%%%%%%%%%
% Set Theory
\newcommand{\card}[1]{\left|#1\right|}
\newcommand{\set}[1]{\left\{#1\right\}}
\newcommand{\ps}[1]{\mathcal{P}\left(#1\right)}
\newcommand{\pfinite}[1]{\mathcal{P}^{\ptxt{finite}}\left(#1\right)}
\newcommand{\naturals}{\mathbb{N}}
\newcommand{\N}{\naturals}
\newcommand{\integers}{\mathbb{Z}}
\newcommand{\Z}{\integers}
\newcommand{\rationals}{\mathbb{Q}}
\newcommand{\Q}{\rationals}
\newcommand{\reals}{\mathbb{R}}
\newcommand{\R}{\reals}
\newcommand{\complex}{\mathbb{C}}
\newcommand{\C}{\complex}
\newcommand{\comp}{^{\complement}}
\DeclareMathOperator{\Hom}{Hom}
\newcommand{\Ind}{\mathbbm{1}}
\newcommand{\cut}{\setminus}

% Graph Theory
\let\deg\relax
\DeclareMathOperator{\deg}{deg}
\newcommand{\degp}{\ptxt{deg}^{+}}
\newcommand{\degn}{\ptxt{deg}^{-}}
\newcommand{\precdot}{\mathrel{\ooalign{$\prec$\cr\hidewidth\hbox{$\cdot\mkern0.5mu$}\cr}}}
\newcommand{\succdot}{\mathrel{\ooalign{$\cdot\mkern0.5mu$\cr\hidewidth\hbox{$\succ$}\cr\phantom{$\succ$}}}}
\DeclareMathOperator{\cl}{cl}
\DeclareMathOperator{\affdim}{affdim}

% Probability
\newcommand{\Prob}{\mathbb{P}}
\newcommand{\Avg}{\mathbb{E}}

% Standard Math
\newcommand{\inv}{^{-1}}
\newcommand{\abs}[1]{\left|#1\right|}
\newcommand{\ceil}[1]{\left\lceil{}#1\right\rceil{}}
\newcommand{\floor}[1]{\left\lfloor{}#1\right\rfloor{}}
\newcommand{\conj}[1]{\overline{#1}}
\newcommand{\of}{\circ}
\newcommand{\tri}{\triangle}
\newcommand{\inj}{\hookrightarrow}
\newcommand{\surj}{\twoheadrightarrow}
\newcommand{\mapsfrom}{\mathrel{\reflectbox{\ensuremath{\mapsto}}}}
\newcommand{\mapsdown}{\rotatebox[origin=c]{-90}{$\mapsto$}\mkern2mu}
\newcommand{\mapsup}{\rotatebox[origin=c]{90}{$\mapsto$}\mkern2mu}
\newcommand{\ndiv}{\nmid}
\renewcommand{\epsilon}{\varepsilon}
\newcommand{\divides}{\mid}
\newcommand{\ndivides}{\nmid}
\DeclareMathOperator{\lcm}{lcm}

% Linear Algebra
\newcommand{\Id}{\textrm{\textnormal{Id}}}
\newcommand{\im}{\textrm{\textnormal{im}}}
\newcommand{\norm}[1]{\abs{\abs{#1}}}
\newcommand{\tpose}{^{T}}
\newcommand{\iprod}[1]{\left<#1\right>}
\DeclareMathOperator{\trace}{tr}
\newcommand{\chgBasMat}[3]{\!\!\tensor*[_{#1}]{\left[#2\right]}{_{#3}}}
\newcommand{\vecBas}[2]{\tensor*[]{\left[#1\right]}{_{#2}}}
\DeclareMathOperator{\GL}{GL}
\DeclareMathOperator{\Mat}{Mat}
\DeclareMathOperator{\vspan}{span}
\DeclareMathOperator{\rank}{rank}
\newcommand{\V}[1]{\vec{#1}}

% Topology
\newcommand{\closure}[1]{\overline{#1}}
\newcommand{\uball}{\mathcal{U}}
\DeclareMathOperator{\Int}{Int}
\DeclareMathOperator{\Ext}{Ext}
\DeclareMathOperator{\Bd}{Bd}
\DeclareMathOperator{\rInt}{rInt}
\DeclareMathOperator{\ch}{ch}
\DeclareMathOperator{\ah}{ah}

% Analysis
\DeclareMathOperator{\Graph}{Graph}
\DeclareMathOperator{\epi}{epi}
\DeclareMathOperator{\hypo}{hypo}
\DeclareMathOperator{\supp}{supp}
\newcommand{\lint}[2]{\underset{#1}{\overset{#2}{{\color{black}\underline{{\color{white}\overline{{\color{black}\int}}\color{black}}}}}}}
\newcommand{\uint}[2]{\underset{#1}{\overset{#2}{{\color{white}\underline{{\color{black}\overline{{\color{black}\int}}\color{black}}}}}}}
\newcommand{\alignint}[2]{\underset{#1}{\overset{#2}{{\color{white}\underline{{\color{white}\overline{{\color{black}\int}}\color{black}}}}}}}
\newcommand{\extint}{\ptxt{ext}\int}
\newcommand{\extalignint}[2]{\ptxt{ext}\alignint{#1}{#2}}
\newcommand{\conv}{\ast}

% Proofs
\newcommand{\st}{s.t.}
\newcommand{\unique}{!}

% Brackets
\newcommand{\paren}[1]{\left(#1\right)}
\renewcommand{\brack}[1]{\left[#1\right]}
\renewcommand{\brace}[1]{\left\{#1\right\}}
\newcommand{\ang}[1]{\left<#1\right>}

% Algorithms
\algrenewcommand{\algorithmiccomment}[1]{\hskip 1em \texttt{// #1}}
\algrenewcommand\algorithmicrequire{\textbf{Input:}}
\algrenewcommand\algorithmicensure{\textbf{Output:}}
\newcommand{\parSymbol}{\P}
\renewcommand{\P}{\ptxt{\textbf{P}}}
\newcommand{\NP}{\ptxt{\textbf{NP}}}
\newcommand{\NPC}{\ptxt{\textbf{NP-Complete}}}
\newcommand{\NPH}{\ptxt{\textbf{NP-Hard}}}
\newcommand{\EXP}{\ptxt{\textbf{EXP}}}

%%%%%%%%%%%%%%%%%%%%%%%%%%%%%
% Other commands
%%%%%%%%%%%%%%%%%%%%%%%%%%%%%
\newcommand{\flag}[1]{\textbf{\textcolor{red}{#1}}}

%%%%%%%%%%%%%%%%%%%%%%%%%%%%%
% Make l's curvy in math environments
%%%%%%%%%%%%%%%%%%%%%%%%%%%%%
\mathcode`l="8000
\begingroup
\makeatletter
\lccode`\~=`\l
\DeclareMathSymbol{\lsb@l}{\mathalpha}{letters}{`l}
\lowercase{\gdef~{\ifnum\the\mathgroup=\m@ne \ell \else \lsb@l \fi}}%
\endgroup

\newcommand{\B}{
    \begin{tikzpicture}
    \filldraw [fill=red, draw=black] (0, 0) rectangle (0.37, 0.45);
    \draw [line width=0.5mm, white ] (0.1,0.08) -- (0.1,0.38);
    \draw[line width=0.5mm, white ] (0.1, 0.35) .. controls (0.2, 0.35) and (0.4, 0.2625) .. (0.1, 0.225);
    \draw[line width=0.5mm, white ] (0.1, 0.225) .. controls (0.2, 0.225) and (0.4, 0.1625) .. (0.1, 0.1);
    \end{tikzpicture}
}

\author{Professor David Barrett\\ \small\textit{Transcribed by Thomas Cohn}}
\title{Partitions of Unity and Proving the Change of Variables Theorem}
\date{11/14/18} % Can also use \today

\begin{document}
\maketitle
\setlength\RaggedRightParindent{\parindent}
\RaggedRight

\par\noindent Recall\n
Type (1) diffeomorphisms: coordinate transposition\n
Type (2) diffeomorphisms: $\paren{\begin{array}{c}x_{1}\\ \vdots\\ x_{n-1}\\ x_{n}\end{array}}\mapsto\paren{\begin{array}{c}x_{1}\\ \vdots\\ x_{n-1}\\ \alpha(\vec{x})\end{array}}$\n

\par\noindent We can combine these to obtain type (3) diffeomorphisms: ``generalized shears''
\[
\paren{\begin{array}{c}x_{1}\\ \vdots\\ x_{j-1}\\ x_{j}\\ x_{j+1}\\ \vdots\\ x_{n}\end{array}}\mapsto\paren{\begin{array}{c}x_{1}\\ \vdots\\ x_{j-1}\\ \eta(\vec{x})\\ x_{j+1}\\ \vdots\\ x_{n}\end{array}}
\]

\prop{Any invertible affine map may be factored into a composition of affine maps of type (1) or (2) (equivalently, type (1) or (3)).\n
Proof: For linear maps, use ``elementary matrix factorization'' (Thm 2.4)\n
For translations, move coord at a time.\n
\proven}

\par\noindent Fun fact: type (1) is just the composition of three type (3) maps.\n

\ex{$\paren{\begin{array}{cc}0 & 1\\ 1 & 0\end{array}}=\paren{\begin{array}{cc}-1 & 1\\ 0 & 1\end{array}}\paren{\begin{array}{cc}1 & 0\\ 1 & 1\end{array}}\paren{\begin{array}{cc}1 & -1\\ 0 & 1\end{array}}$}

\thm{In some neighborhood of $\vec{p}$, $g$ can be factored into a composition of diffeomorphisms of type (1) or (2) (equivalently (1) or (3)).\n
\n
Step 1: Pick $T_{1},T_{2}:\R^{n}\to\R^{n}$ invertible, affine such that $g=T_{1}\of\tilde{g}\of{}T_{2}$ with $T_{2}(\vec{p})=\vec{0}$ and $T_{1}(\vec{0})=\vec{q}$, $D\tilde{g}(\vec{0})=\Id$. So $\tilde{g}(\vec{0})=\vec{0}$, and we can take $DT_{2}=Dg(\vec{p})$ and $DT_{1}=\Id$.
\n
\n
\newpage
Step 2: $\begin{array}{cccl}
\paren{\begin{array}{c}x_{1}\\ x_{2}\\ x_{3}\\ \vdots\\ x_{n}\end{array}} & \mapsto & \paren{\begin{array}{c}\tilde{g}_{1}(\vec{x})\\ x_{2}\\ x_{3}\\ \vdots\\ x_{n}\end{array}} & \ptxt{So $D\tilde{g}_{k}(\vec{0})=\vec{e_{k}}$, so the derivative at $\vec{0}$ is $\Id$, so it's locally diffeomorphic.}\\
 & & \mapsdown & \ptxt{local diffeomorphism*}\\
\paren{\begin{array}{c}x_{1}\\ x_{2}\\ x_{3}\\ \vdots\\ x_{n}\end{array}} & \mapsto & \paren{\begin{array}{c}\tilde{g}_{1}(\vec{x})\\ \tilde{g}_{2}(\vec{x})\\ x_{3}\\ \vdots\\ x_{n}\end{array}} & \ptxt{Also has derivative at $\vec{0}$ is $\Id$, so locally diffeomorphic.}\\
 & & \mapsdown & \ptxt{local diffeomorphism*}\\
\paren{\begin{array}{c}x_{1}\\ x_{2}\\ x_{3}\\ \vdots\\ x_{n}\end{array}} & \mapsto & \paren{\begin{array}{c}\tilde{g}_{1}(\vec{x})\\ \tilde{g}_{2}(\vec{x})\\ \tilde{g}_{3}(\vec{x})\\ \vdots\\ x_{n}\end{array}} & \ptxt{''}\\
 & & \mapsdown & \ptxt{local diffeomorphism*}\\
 & & \mapsdown & \ptxt{local diffeomorphism*}\\
\end{array}$\n
So $\vec{x}\mapsto\tilde{g}(\vec{x})$ has derivative $\Id$ at $\vec{0}$, so it's locally diffeomorphic.\n
\n
* These diffeomorphisms preserve $n-1$ coordinates, so they're type (3).\n
\n
And $T_{1},T_{2}$ are type (1) diffeomorphisms.\n
\proven}

\defn{Consider $f:X^{\ptxt{metric space}}\to{}V^{\ptxt{vector space}}$. $\supp{}f\overset{\ptxt{def}}{=}\closure{\set{\vec{x}:f(\vec{x})\ne\vec{0}}}$.\n
So $\vec{x}\not\in\supp{}f\Leftrightarrow\exists\epsilon>0$ \st{} $f\equiv\vec{0}$ on $U(\vec{x},\epsilon)$.}

\cor{(of factorization and results from Monday):\n
There exists a neighborhood $U$ of $\vec{q}$ such that the COVT holds when $\supp{}f\subset{}U$.\n
Proof: Picture.\proven}

\par\noindent We now have a local version of the COVT!\n

\thm{(Partition of Unity) Given $\Omega^{\ptxt{open}}\subset\R^{n}$ with $\Omega=\bigcup_{\alpha\in\Gamma}U_{\alpha}^{\ptxt{open}}$, then\n
$\exists\varphi_{1},\varphi_{2},\ldots\in{}C^{\infty}(\Omega,[0,+\infty))$ \st{}
\begin{enumerate}[label=(\roman*),leftmargin=2cm]
	\item each $\supp\varphi_{j}\subset\ptxt{some }U_{\alpha_{j}}$
	\item each $\supp\varphi_{j}$ compact
	\item each $\vec{x}\in\Omega$ has a nbd meeting (i.e., non-empty intersection) only finitely many $\supp\varphi_{j}$
	\item $\sum_{j=1}^{\infty}\varphi_{j}(\vec{x})=1$ for al $\vec{x}\in\Omega$ (locally finite sum)
\end{enumerate}\up\n
Then $\set{\varphi_{j}}$ is a ``partition of unity dominated by $\set{U_{\alpha}}$''.\n
Proof: Nov 21, or read \S{}16.}

\lemma{Given $f\in{}C(B^{\ptxt{osso}\R^{n}},\R)$, $\extint_{B}f$ exists, $\set{\varphi_{j}}$ satisfies (ii), (iii), (iv).\n
Then $\displaystyle\extint_{B}f=\sum_{j=1}^{\infty}\int_{B}\varphi_{j}\cdot{}f$.\n
\n
Proof: strategy is tackle $f\ge{}0$, then apply previous result to $f_{+}$, $f_{-}$, and combine.\n
\n
Assume $f\ge{}0$. Then $E^{\ptxt{cpt},\ptxt{rect}}\subset{}B\Rightarrow\exists{}M$ \st{} $U_{j}\equiv{}0$ on $E$ for $j\ge{}M$. This is compact using (iii).\n
Thus, $\displaystyle\int_{E}f=\int_{E}\sum_{i=1}^{M}\varphi_{j}\cdot{}f=\sum_{i=1}^{M}\int_{E}\varphi_{j}\cdot{}f\le\sum_{j=1}^{M}\int_{B}\varphi_{j}\cdot{}f\le\sum_{j=1}^{\infty}\int_{B}\varphi_{j}\cdot{}f$.\n
Take the supremum over $E$, obtain $\displaystyle\extint_{E}f\le\sum_{j=1}^{\infty}\int_{B}\varphi_{j}\cdot{}f$\n
Also, $\displaystyle\sum_{j=1}^{\infty}\int_{B}\varphi_{j}\cdot{}f=\lim_{M\to\infty}\sum_{j=1}^{M}\int_{B}\varphi_{j}\cdot{}f=\lim_{M\to\infty}\int_{B}\sum_{j=1}^{M}\varphi_{j}\cdot{}f\le\extint_{B}f$.\n
\n
Apply to $f_{+},f_{-}$.\n
Combine.\proven}

\par\noindent Proof of the Change of Variables Theorem: For $\vec{y}\in{}B$, choose $U_{\vec{y}}$ \st{} $g$ factors on $U_{\vec{y}}$. Choose partition of unity dominated by $\set{U_{\vec{y}}:\vec{y}\in{}B}$. Then
\[
\extint_{B}f_{+}\overset{\ptxt{lemma}}{=}\sum\int_{B}\varphi_{j}f_{+}=\sum\int_{A}(\varphi_{1}\of{}g)(f_{+}\of{}g)\abs{\det{}Dg}\overset{\ptxt{lemma}}{=}\extint_{A}(f_{+}\of{}g)\cdot\abs{\det{}Dg}
\]

\par\noindent Similarly, $\displaystyle\extint_{A}f_{-}=\extint_{A}(f_{-}\of{}g)\cdot\abs{\det{}Dg}$.\n

\par\noindent Now combine (unles both terms are infinite).\proven

\end{document}