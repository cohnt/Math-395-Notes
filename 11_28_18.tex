\documentclass[10pt,letterpaper]{article}
\usepackage[utf8]{inputenc}
\usepackage[intlimits]{amsmath}
\usepackage{amsfonts}
\usepackage{amssymb}
\usepackage{ragged2e}
\usepackage[letterpaper, margin=1in]{geometry}
\usepackage{graphicx}
\usepackage{cancel}
\usepackage{mathtools}
\usepackage{tabularx}
\usepackage{arydshln}
\usepackage{tensor}
\usepackage{array}
\usepackage{xcolor}
\usepackage[boxed]{algorithm}
\usepackage[noend]{algpseudocode}
\usepackage{listings}
\usepackage{textcomp}
\usepackage[pdf,tmpdir,singlefile]{graphviz}
\usepackage{mathrsfs}
\usepackage{bbm}
\usepackage{tikz}
\usepackage{enumitem}
\usepackage{arydshln}

%%%%%%%%%%%%%%%%%%%%%%%%%%%%%
% Formatting commands
%%%%%%%%%%%%%%%%%%%%%%%%%%%%%
\newcommand{\n}{\hfill\break}
\newcommand{\up}{\vspace{-\baselineskip}}
\newcommand{\lemma}[1]{\par\noindent\settowidth{\hangindent}{\textbf{Lemma: }}\textbf{Lemma: }#1}
\newcommand{\defn}[1]{\par\noindent\settowidth{\hangindent}{\textbf{Defn: }}\textbf{Defn: }#1\n}
\newcommand{\thm}[1]{\par\noindent\settowidth{\hangindent}{\textbf{Thm: }}\textbf{Thm: }#1\n}
\newcommand{\prop}[1]{\par\noindent\settowidth{\hangindent}{\textbf{Prop: }}\textbf{Prop: }#1\n}
\newcommand{\cor}[1]{\par\noindent\settowidth{\hangindent}{\textbf{Cor: }}\textbf{Cor: }#1\n}
\newcommand{\ex}[1]{\par\noindent\settowidth{\hangindent}{\textbf{Ex: }}\textbf{Ex: }#1\n}
\newcommand{\proven}{\;$\square$\n}
\newcommand{\problem}[1]{\par\noindent{#1}\n}
\newcommand{\problempart}[2]{\par\noindent\indent{}\settowidth{\hangindent}{\textbf{(#1)} \indent{}}\textbf{(#1)} #2\n}
\newcommand{\ptxt}[1]{\textrm{\textnormal{#1}}}
\newcommand{\inlineeq}[1]{\centerline{$\displaystyle #1$}}
\newcommand{\pageline}{\noindent\rule{\textwidth}{0.1pt}}

%%%%%%%%%%%%%%%%%%%%%%%%%%%%%
% Math commands
%%%%%%%%%%%%%%%%%%%%%%%%%%%%%
% Set Theory
\newcommand{\card}[1]{\left|#1\right|}
\newcommand{\set}[1]{\left\{#1\right\}}
\newcommand{\ps}[1]{\mathcal{P}\left(#1\right)}
\newcommand{\pfinite}[1]{\mathcal{P}^{\ptxt{finite}}\left(#1\right)}
\newcommand{\naturals}{\mathbb{N}}
\newcommand{\N}{\naturals}
\newcommand{\integers}{\mathbb{Z}}
\newcommand{\Z}{\integers}
\newcommand{\rationals}{\mathbb{Q}}
\newcommand{\Q}{\rationals}
\newcommand{\reals}{\mathbb{R}}
\newcommand{\R}{\reals}
\newcommand{\complex}{\mathbb{C}}
\newcommand{\C}{\complex}
\newcommand{\comp}{^{\complement}}
\DeclareMathOperator{\Hom}{Hom}
\newcommand{\Ind}{\mathbbm{1}}
\newcommand{\cut}{\setminus}

% Graph Theory
\let\deg\relax
\DeclareMathOperator{\deg}{deg}
\newcommand{\degp}{\ptxt{deg}^{+}}
\newcommand{\degn}{\ptxt{deg}^{-}}
\newcommand{\precdot}{\mathrel{\ooalign{$\prec$\cr\hidewidth\hbox{$\cdot\mkern0.5mu$}\cr}}}
\newcommand{\succdot}{\mathrel{\ooalign{$\cdot\mkern0.5mu$\cr\hidewidth\hbox{$\succ$}\cr\phantom{$\succ$}}}}
\DeclareMathOperator{\cl}{cl}
\DeclareMathOperator{\affdim}{affdim}

% Probability
\newcommand{\Prob}{\mathbb{P}}
\newcommand{\Avg}{\mathbb{E}}

% Standard Math
\newcommand{\inv}{^{-1}}
\newcommand{\abs}[1]{\left|#1\right|}
\newcommand{\ceil}[1]{\left\lceil{}#1\right\rceil{}}
\newcommand{\floor}[1]{\left\lfloor{}#1\right\rfloor{}}
\newcommand{\conj}[1]{\overline{#1}}
\newcommand{\of}{\circ}
\newcommand{\tri}{\triangle}
\newcommand{\inj}{\hookrightarrow}
\newcommand{\surj}{\twoheadrightarrow}
\newcommand{\mapsfrom}{\mathrel{\reflectbox{\ensuremath{\mapsto}}}}
\newcommand{\mapsdown}{\rotatebox[origin=c]{-90}{$\mapsto$}\mkern2mu}
\newcommand{\mapsup}{\rotatebox[origin=c]{90}{$\mapsto$}\mkern2mu}
\newcommand{\ndiv}{\nmid}
\renewcommand{\epsilon}{\varepsilon}
\newcommand{\divides}{\mid}
\newcommand{\ndivides}{\nmid}
\DeclareMathOperator{\lcm}{lcm}

% Linear Algebra
\newcommand{\Id}{\textrm{\textnormal{Id}}}
\newcommand{\im}{\textrm{\textnormal{im}}}
\newcommand{\norm}[1]{\abs{\abs{#1}}}
\newcommand{\tpose}{^{T}}
\newcommand{\iprod}[1]{\left<#1\right>}
\DeclareMathOperator{\trace}{tr}
\newcommand{\chgBasMat}[3]{\!\!\tensor*[_{#1}]{\left[#2\right]}{_{#3}}}
\newcommand{\vecBas}[2]{\tensor*[]{\left[#1\right]}{_{#2}}}
\DeclareMathOperator{\GL}{GL}
\DeclareMathOperator{\Mat}{Mat}
\DeclareMathOperator{\vspan}{span}
\DeclareMathOperator{\rank}{rank}
\newcommand{\V}[1]{\vec{#1}}

% Topology
\newcommand{\closure}[1]{\overline{#1}}
\newcommand{\uball}{\mathcal{U}}
\DeclareMathOperator{\Int}{Int}
\DeclareMathOperator{\Ext}{Ext}
\DeclareMathOperator{\Bd}{Bd}
\DeclareMathOperator{\rInt}{rInt}
\DeclareMathOperator{\ch}{ch}
\DeclareMathOperator{\ah}{ah}

% Analysis
\DeclareMathOperator{\Graph}{Graph}
\DeclareMathOperator{\epi}{epi}
\DeclareMathOperator{\hypo}{hypo}
\DeclareMathOperator{\supp}{supp}
\newcommand{\lint}[2]{\underset{#1}{\overset{#2}{{\color{black}\underline{{\color{white}\overline{{\color{black}\int}}\color{black}}}}}}}
\newcommand{\uint}[2]{\underset{#1}{\overset{#2}{{\color{white}\underline{{\color{black}\overline{{\color{black}\int}}\color{black}}}}}}}
\newcommand{\alignint}[2]{\underset{#1}{\overset{#2}{{\color{white}\underline{{\color{white}\overline{{\color{black}\int}}\color{black}}}}}}}
\newcommand{\extint}{\ptxt{ext}\int}
\newcommand{\extalignint}[2]{\ptxt{ext}\alignint{#1}{#2}}
\newcommand{\conv}{\ast}

% Proofs
\newcommand{\st}{s.t.}
\newcommand{\unique}{!}

% Brackets
\newcommand{\paren}[1]{\left(#1\right)}
\renewcommand{\brack}[1]{\left[#1\right]}
\renewcommand{\brace}[1]{\left\{#1\right\}}
\newcommand{\ang}[1]{\left<#1\right>}

% Algorithms
\algrenewcommand{\algorithmiccomment}[1]{\hskip 1em \texttt{// #1}}
\algrenewcommand\algorithmicrequire{\textbf{Input:}}
\algrenewcommand\algorithmicensure{\textbf{Output:}}
\newcommand{\parSymbol}{\P}
\renewcommand{\P}{\ptxt{\textbf{P}}}
\newcommand{\NP}{\ptxt{\textbf{NP}}}
\newcommand{\NPC}{\ptxt{\textbf{NP-Complete}}}
\newcommand{\NPH}{\ptxt{\textbf{NP-Hard}}}
\newcommand{\EXP}{\ptxt{\textbf{EXP}}}

%%%%%%%%%%%%%%%%%%%%%%%%%%%%%
% Other commands
%%%%%%%%%%%%%%%%%%%%%%%%%%%%%
\newcommand{\flag}[1]{\textbf{\textcolor{red}{#1}}}

%%%%%%%%%%%%%%%%%%%%%%%%%%%%%
% Make l's curvy in math environments
%%%%%%%%%%%%%%%%%%%%%%%%%%%%%
\mathcode`l="8000
\begingroup
\makeatletter
\lccode`\~=`\l
\DeclareMathSymbol{\lsb@l}{\mathalpha}{letters}{`l}
\lowercase{\gdef~{\ifnum\the\mathgroup=\m@ne \ell \else \lsb@l \fi}}%
\endgroup

\newcommand{\B}{
    \begin{tikzpicture}
    \filldraw [fill=red, draw=black] (0, 0) rectangle (0.37, 0.45);
    \draw [line width=0.5mm, white ] (0.1,0.08) -- (0.1,0.38);
    \draw[line width=0.5mm, white ] (0.1, 0.35) .. controls (0.2, 0.35) and (0.4, 0.2625) .. (0.1, 0.225);
    \draw[line width=0.5mm, white ] (0.1, 0.225) .. controls (0.2, 0.225) and (0.4, 0.1625) .. (0.1, 0.1);
    \end{tikzpicture}
}

\author{Professor David Barrett\\ \small\textit{Transcribed by Thomas Cohn}}
\title{The First Fundamental Theorem of Calculus for $1$-Forms}
\date{11/28/18} % Can also use \today

\begin{document}
\maketitle
\setlength\RaggedRightParindent{\parindent}
\RaggedRight

\par\noindent Let $f:[a,b]\to\R^{n}$\n

\prop{TFAE:
\begin{enumerate}[label=(\arabic*)]
	\item $f$ extends to a function in $C^{k}(\R,\R^{n})$
	\item $f\in{}C([a,b])$ and $f|_{(a,b)}$ is $C^{k}$ and $\lim_{t\searrow{}a}f^{(j)}(t),\lim_{t\nearrow{}b}f^{(j)}(t)$ exist and are finite for $j=1,\ldots,k$.
\end{enumerate}}

\par\noindent Proof (1)$\Rightarrow$(2): Trivial\n
Proof (2)$\Rightarrow$(1): Use Taylor polynomials to extend\n

\defn{$f\in{}C_{pw}^{k}\overset{\ptxt{def}}{\Leftrightarrow}f\ptxt{ cts on }[a,b]$ and $f|_{[t_{j-1},t_{j}]}\in{}C^{k}$ for each $j$ (where the $t_{j}$'s partition $[a,b]$). We say that $f$ is \underline{piecewise $C^{k}$}.}

\defn{Let $\omega$ be a $1$-form on $A^{\ptxt{open}}\subseteq\R^{n}$, $I=[a,b]\subset\R$, and $\alpha\in{}C_{pw}^{1}(I,A)$ (a ``path in $A$'').\n
Then $\displaystyle\int_{Y_{\alpha}}\omega\overset{\ptxt{def}}{=}\int_{I}\alpha^{*}\omega=\int_{I}(\omega\of\alpha)D\alpha$.}

\par\noindent Rewrite: $\displaystyle\int_{Y_{\alpha}}\sum_{i=1}^{n}\omega_{i}dx_{i}=\int_{I}\sum_{i=1}^{n}\omega_{i}(\alpha(t))\frac{dx_{i}}{dt}$\n

\par\noindent The idea is that $\vec{x}$ is the position at time $t$ given by $\alpha(t)$, i.e., $x_{j}=\alpha_{j}(t)$.\n

\par\noindent From Monday: For $\alpha\in{}C^{1}$, we have $\displaystyle\int_{Y_{\alpha}}df=\bigtriangleup_{Y_{\alpha}}f\overset{\ptxt{def}}{=}f(\alpha(b))-f(\alpha(a))$.\n
Exercise: show this still works for $\alpha\in{}C_{pw}^{1}$.\n

\thm{Given $A^{\ptxt{conn},\ptxt{open}}\subseteq\R^{n}$, $f\in{}C^{1}(A,\R)$. Then $df=0$ on $A$ if and only if $f$ is constant.}

\par\noindent HW3\#4: Choose $\alpha\in{}C([0,1],A)$ such that $\alpha(0)=a$, and $\alpha(1)=b$.\n
Proof $\Leftarrow$: trivial\n
Proof $\Rightarrow$: Use $\bigtriangleup_{Y_{\alpha}}f=\sum_{j}\bigtriangleup_{j^{\ptxt{th}}\ptxt{ piece}}f=0$.\n

\par\noindent Note: for $A$ open, \textit{disconnected}, $df=0\Leftrightarrow{}f$ is constant on each connected component of $A$.\n

\par\noindent Problem: Can we rewrite a $1$-form integral $\int_{Y_{\alpha}}\omega$ as a scalar integral $\int_{Y_{\alpha}}gds$?\n
Recall: $\displaystyle\int_{Y_{\alpha}}gds=\int_{I}(g\of\alpha)V(D\alpha)=\int_{I}(g\of\alpha)\sqrt{\det{}D\alpha\tpose{}D\alpha}=\int_{I}(g\of\alpha)\norm{D_{\alpha}}=\int_{I}(g\of\alpha)\norm{\alpha'}$\n
$\displaystyle\int_{Y_{\alpha}}\omega=\int_{I}(\omega\of\alpha)D\alpha=\int_{I}(\omega\of\alpha)a'$.\n

\par\noindent So match if $(g\of\alpha)\norm{a'}=(\omega\of\alpha)\alpha'$, i.e., if for any point $\vec{p}=\alpha(t)\in{}Y$, we have $g(\vec{p})\norm{\alpha'(t)}=\omega(\vec{p})\alpha'(t)$ -- $g(\vec{p})=\omega(\vec{p})\cdot\frac{\alpha'(t)}{\norm{\alpha'(t)}}$.\n

\par\noindent Trouble if $\alpha'(t)=0$. But if $\alpha\in{}C^{1}$, $\alpha$ injective, then $\alpha'\ne{}0$.\n

\par\noindent Set $T:Y\to\R^{n}$, with $\alpha(t)\mapsto\frac{\alpha'(t)}{\norm{\alpha'(t)}}$. $T$ is the unit tangent vector function.\n

\par\noindent We get $\displaystyle\int_{Y_{\alpha}}(\omega\cdot{}T)ds=\int_{Y_{\alpha}}\omega$. Reverse: $\displaystyle\int{}gds=\int{}gT^{\tpose}$.\n

\thm{FTC1a for $1$-forms\n
Given $\omega$ $1$-form on $A^{\ptxt{open},\ptxt{conn}}\subset\R^{n}$.\n
Then TFAE:\n
\begin{enumerate}[label=(\arabic*)]
	\item $\omega=df$ for some $f\in{}C^{1}(A,\R)\overset{\ptxt{def}}{\Leftrightarrow}\omega$ is \underline{exact} on $A$.
	\item $\int_{Y_{\alpha}}\omega=0$ when $\alpha\in{}C_{pw}^{1}([a,b],A)$ with $\alpha(a)=\alpha(b)$.
	\item $\int_{Y_{\alpha_{1}}}\omega=\int_{Y_{\alpha_{2}}}\omega$ when $\alpha_{j}\in{}C_{pw}^{1}([a_{j},b_{j}],A)$ with $\alpha_{1}(a_{1})=\alpha_{2}(a_{2})$ and $\alpha_{1}(b_{1})=\alpha_{2}(b_{2})$.\n
	``Path Independence''
\end{enumerate}}

\par\noindent Proof (1)$\Rightarrow$(2): $\int_{Y_{\alpha}}df\overset{ptxt{FTC2}}{=}f(\alpha(b))-f(\alpha(a))=0$.\n

\par\noindent Proof (2)$\Rightarrow$(3): Form a single path $\alpha$ from $\alpha_{1}$ and the reverse of $\alpha_{2}$.\n
$\displaystyle\int_{Y_{\alpha_{1}}}\omega-\int_{Y_{\alpha_{2}}}\omega=\int_{Y_{\alpha}}\omega=0$. So $\displaystyle\int_{Y_{\alpha_{1}}}\omega=\int_{Y_{\alpha_{2}}}\omega$\n

\par\noindent Proof (3)$\Rightarrow$(1): We can define $\int_{x}^{y}\omega$ for $x,y\in{}A$ with $\int_{x}^{y}\omega+\int_{y}^{z}\omega=\int_{x}^{z}\omega$. Fix $x_{0}\in{}A$, define $f(x)=\int_{x_{0}}^{x}\omega$.\n
Claim: $df=\omega$, i.e., $\star=\frac{f(x+h)-f(x)-\omega(x)\cdot{}h}{\norm{h}}\to{}0$ as $h\to{}0$.\n
But $f(x+h)-f(x)=\int_{x}^{x+h}\omega=\int_{0}^{1}\omega(x+th)\cdot{}h$. So $\star=\frac{\int_{0}^{1}(\omega(x+th)-\omega(x))\cdot{}hdt}{\norm{h}}$\n
Thus, $\norm{\star}\le\underset{0\le{}t\le{}h}{\max}\norm{\omega(x+th)-\omega(x)}\to{}0$ as $h\to{}0$.\n

\par\noindent Exercise: this still works for $C_{pw}^{1}$, $C_{pw}^{k}$, and $C_{pw}^{\infty}$.\n

\par\noindent An alternate approach: Need $D_{j}f=\omega_{j}$ (a system of partial differential equations).\n

\par\noindent Recall: Thm 6.3 gives us $f\in{}C^{2}\Rightarrow{}D_{k}D_{j}f=D_{j}D_{k}f$. $D_{j}f=\omega_{j}$ and $D_{k}f=\omega_{k}$. So $D_{k}\omega_{j}=D_{j}\omega_{k}$.\n
So $\omega$ is $C^{1}$ and exact on $A$, and thus $D_{k}\omega_{j}=D_{j}\omega_{k}\overset{\ptxt{def}}{\Leftrightarrow}\omega$ is closed on $A$.\n

\thm{FTC1b for $1$-forms\n
Given $\omega$ $C^{1}$ closed $1$-form, $\alpha\in{}C^{2}$, then $\alpha^{*}\omega$ closed.\n
Pf1: Wait for Thm 32.3\n
Pf2: Read $4$-line computation in Lemma J.6}

\end{document}