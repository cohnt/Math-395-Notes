\documentclass[10pt,letterpaper]{article}
\usepackage[utf8]{inputenc}
\usepackage{amsmath}
\usepackage{amsfonts}
\usepackage{amssymb}
\usepackage{ragged2e}
\usepackage[letterpaper, margin=1in]{geometry}
\usepackage{graphicx}
\usepackage{cancel}
\usepackage{mathtools}
\usepackage{tabularx}
\usepackage{arydshln}
\usepackage{tensor}
\usepackage{array}
\usepackage{xcolor}
\usepackage[boxed]{algorithm}
\usepackage[noend]{algpseudocode}
\usepackage{listings}
\usepackage{textcomp}
\usepackage[pdf,tmpdir,singlefile]{graphviz}

%%%%%%%%%%%%%%%%%%%%%%%%%%%%%
% Formatting commands
%%%%%%%%%%%%%%%%%%%%%%%%%%%%%
\newcommand{\n}{\hfill\break}
\newcommand{\lemma}[1]{\par\noindent\settowidth{\hangindent}{\textbf{Lemma: }}\textbf{Lemma: }#1\n}
\newcommand{\defn}[1]{\par\noindent\settowidth{\hangindent}{\textbf{Defn: }}\textbf{Defn: }#1\n}
\newcommand{\thm}[1]{\par\noindent\settowidth{\hangindent}{\textbf{Thm: }}\textbf{Thm: }#1\n}
\newcommand{\prop}[1]{\par\noindent\settowidth{\hangindent}{\textbf{Prop: }}\textbf{Prop: }#1\n}
\newcommand{\cor}[1]{\par\noindent\settowidth{\hangindent}{\textbf{Cor: }}\textbf{Cor: }#1\n}
\newcommand{\ex}[1]{\par\noindent\settowidth{\hangindent}{\textbf{Ex: }}\textbf{Ex: }#1\n}
\newcommand{\proven}{\;$\square$\n}
\newcommand{\problem}[1]{\par\noindent{#1}\n}
\newcommand{\problempart}[2]{\par\noindent\indent{}\settowidth{\hangindent}{\textbf{(#1)} \indent{}}\textbf{(#1)} #2\n}
\newcommand{\ptxt}[1]{\textrm{\textnormal{#1}}}
\newcommand{\inlineeq}[1]{\centerline{$\displaystyle #1$}}
\newcommand{\pageline}{\noindent\rule{\textwidth}{0.1pt}}

%%%%%%%%%%%%%%%%%%%%%%%%%%%%%
% Math commands
%%%%%%%%%%%%%%%%%%%%%%%%%%%%%
% Set Theory
\newcommand{\card}[1]{\left|#1\right|}
\newcommand{\set}[1]{\left\{#1\right\}}
\newcommand{\ps}[1]{\mathcal{P}\left(#1\right)}
\newcommand{\pfinite}[1]{\mathcal{P}^{\ptxt{finite}}\left(#1\right)}
\newcommand{\naturals}{\mathbb{N}}
\newcommand{\N}{\naturals}
\newcommand{\integers}{\mathbb{Z}}
\newcommand{\Z}{\integers}
\newcommand{\rationals}{\mathbb{Q}}
\newcommand{\Q}{\rationals}
\newcommand{\reals}{\mathbb{R}}
\newcommand{\R}{\reals}
\newcommand{\complex}{\mathbb{C}}
\newcommand{\C}{\complex}
\newcommand{\comp}{^{\complement}}
\newcommand{\Hom}{\ptxt{Hom}\>}

% Graph Theory
\renewcommand{\deg}[1]{\ptxt{deg}\left(#1\right)}
\newcommand{\degp}[1]{\ptxt{deg}^{+}\!\!\left(#1\right)}
\newcommand{\degn}[1]{\ptxt{deg}^{-}\!\!\left(#1\right)}

% Standard Math
\newcommand{\inv}{^{-1}}
\newcommand{\abs}[1]{\left|#1\right|}
\newcommand{\ceil}[1]{\left\lceil{}#1\right\rceil}
\newcommand{\floor}[1]{\left\lfloor{}#1\right\rfloor{}}
\newcommand{\conj}[1]{\overline{#1}}
\newcommand{\of}{\circ}
\newcommand{\tri}{\triangle}
\newcommand{\inj}{\hookrightarrow}
\newcommand{\surj}{\twoheadrightarrow}
\newcommand{\mapsfrom}{\mathrel{\reflectbox{\ensuremath{\mapsto}}}}
\newcommand{\Graph}{\ptxt{Graph}\>}
\newcommand{\ndiv}{\nmid}
\renewcommand{\epsilon}{\varepsilon}

% Linear Algebra
\newcommand{\Id}{\textrm{\textnormal{Id}}}
\newcommand{\im}{\textrm{\textnormal{im}}}
\newcommand{\norm}[1]{\abs{\abs{#1}}}
\newcommand{\tpose}{^{T}}
\newcommand{\iprod}[1]{\left<#1\right>}
\newcommand{\trace}{\ptxt{tr}}
\newcommand{\chgBasMat}[3]{\!\!\tensor*[_{#1}]{\left[#2\right]}{_{#3}}}
\newcommand{\vecBas}[2]{\tensor*[]{\left[#1\right]}{_{#2}}}
\newcommand{\GL}{\ptxt{GL}\>}
\newcommand{\Mat}{\ptxt{Mat}\>}

% Topology
\newcommand{\closure}[1]{\overline{#1}}
\newcommand{\uball}{\mathcal{U}}
\newcommand{\Int}{\ptxt{Int}\>}
\newcommand{\Ext}{\ptxt{Ext}\>}
\newcommand{\Bd}{\ptxt{Bd}\>}

% Proofs
\newcommand{\st}{s.t.}
\newcommand{\unique}{!}

% Algorithms
\algrenewcommand{\algorithmiccomment}[1]{\hskip 1em \texttt{// #1}}
\algrenewcommand\algorithmicrequire{\textbf{Input:}}
\algrenewcommand\algorithmicensure{\textbf{Output:}}

%%%%%%%%%%%%%%%%%%%%%%%%%%%%%
% Other commands
%%%%%%%%%%%%%%%%%%%%%%%%%%%%%
\newcommand{\flag}[1]{\textbf{\textcolor{red}{#1}}}

%%%%%%%%%%%%%%%%%%%%%%%%%%%%%
% Make l's curvy in math environments
%%%%%%%%%%%%%%%%%%%%%%%%%%%%%
\mathcode`l="8000
\begingroup
\makeatletter
\lccode`\~=`\l
\DeclareMathSymbol{\lsb@l}{\mathalpha}{letters}{`l}
\lowercase{\gdef~{\ifnum\the\mathgroup=\m@ne \ell \else \lsb@l \fi}}%
\endgroup

\author{Thomas Cohn}
\title{Inverse Functions}
\date{10/1/18} % Can also use \today

\begin{document}
\maketitle
\setlength\RaggedRightParindent{\parindent}
\RaggedRight

\par\noindent Let $A^{\ptxt{open}}\subset{}V$, $B^{\ptxt{open}}\subset{}W$. Suppose the following:\n
$f:A\to{}B$ is differentiable at $\vec{a}$\n
$g:B\to{}A$ is differentiable at $\vec{b}=f(\vec{a})$\n
$g\of{}f=\Id_{A}$ (i.e. $g(f(x))=x$)\n

\par\noindent Then $Dg(\vec{b})\of{}Df(\vec{a})=\Id_{A}$\n
$Dg(\vec{b})$ is a left inverse of $Df(\vec{a})$\n
$\dim{}V\le\dim{}W$.\n

\begin{enumerate}
	\item If also $f\of{}g=\Id_{B}$, then
	\begin{itemize}
		\item $Df(\vec{a})\of{}Dg(\vec{b})=\Id_{B}$
		\item $Dg(\vec{b})$ is a $2$-sided inverse of $Df(\vec{a})$
		\item $\dim{}V\ge\dim{}W$, so $\dim{}V=\dim{}W$
	\end{itemize}
	\item If instead we have $\dim{}V=\dim{}W<+\infty$, then
	\begin{itemize}
		\item $Dg(\vec{b})$ is \cancel{a} the $2$-sided inverse of $Df(\vec{a})$
	\end{itemize}
	\item If $A,B,f,g$ as above, $\dim{}V,\dim{}W<+\infty$, and $f,g$ are continuous, then
	\begin{itemize}
		\item $g\of{}f=\Id_{A}$ and $f\of{}g=\Id_{B}\to\dim{}V=\dim{}W$. Proof of this is very hard. It requires new tools, so we'll return to it another time.
	\end{itemize}
	\item $\exists{}f:\R\to\R^{2}$ continuous and surjective.
\end{enumerate}

\defn{A \underline{homeomorphism} is a continuous bijection $f:A\to{}B$ ($A$, $B$ topological spaces) such that $f\inv$ is continuous.}

\ex{$\begin{array}{l}f:[0,2\pi)\to{}S'=\set{(x,y)\in\R^{2}:x^{2}+y^{2}=1}\\ t\mapsto(\cos{}t,\sin{}t)\end{array}$\n
$f$ is a continuous bijection, but not a homeomorphism.\n
$f\inv\left(\cos\frac{1}{n},\sin\frac{1}{n}\right)=2\pi-\frac{1}{n}$, so as $n\to+\infty$, $f\inv\to{}2\pi$. But $f\inv(1,0)=0$, so $f\inv$ is not continuous.}

\defn{A \underline{$C^{r}$-diffeomorphism} is a $C^{r}$ bijection $f:A^{\ptxt{osso}V}\to{}B^{\ptxt{osso}W}$ ($V,W$ normed vector spaces) such that $f\inv$ is also $C^{r}$.}

\ex{$\begin{array}{l}f:\R\to\R\\ t\mapsto{}t^{3}\end{array}$ is a homeomorphism, but not a $C^{1}$-diffeomorphism.}

\defn{A complete, normed vector space is called a \underline{Banach space}.}

\thm{(Inverse Function) Given $\vec{a}\in{}A^{\ptxt{open}}\subset\R^{n}$, $f\in{}C^{r}(A,\R^{n})$ for $r\in\N$, $Df(\vec{a})$ is invertible, then there is a $\uball^{\ptxt{open}}$ with $\vec{a}\in\uball$ such that $f|_{_{\uball}}$ is a $C^{r}$-diffeomorphism, i.e., $f$ maps $\uball$ injectively to an open set, $f\inv$ is $C^{r}$.\n
It turns out this is ok if the dimension is infinite, so long as $V$ and $W$ are Banach spaces.}

\ex{$A=\set{\left(\begin{array}{c}r\\ \theta\end{array}\right)\in\R^{2}:r>0}$\n
$f\left(\begin{array}{c}r\\ \theta\end{array}\right)=\left(\begin{array}{c}r\cos\theta\\ r\sin\theta\end{array}\right)$\n
$Df\left(\begin{array}{c}r\\ \theta\end{array}\right)=\left[\begin{array}{cc}\cos\theta & -r\sin\theta\\ \sin\theta & r\cos\theta\end{array}\right]\qquad\det\left(Df\left(\begin{array}{c}r\\ \theta\end{array}\right)\right)=r\cos^{2}\theta-(-r\sin^{2}\theta)=r>0$\n
So $Df\left(\begin{array}{c}r\\ \theta\end{array}\right)$ is invertible for all $\left(\begin{array}{c}r\\ \theta\end{array}\right)\in{}A$, \underline{but} $f$ is not injective on $A$.\n
$f\left(\begin{array}{c}r\\ \theta\end{array}\right)=f\left(\begin{array}{c}r\\ \theta+2\pi\end{array}\right)$. So we can get local $C^{\infty}$ inverses, but no global inverse.\n
$f[A]=\R^{2}\setminus\set{\left(\begin{array}{c}0\\ 0\end{array}\right)}$.}

\par\noindent Some notes: $E=\set{\vec{x}\in{}A:Df(\vec{x})\ptxt{ invertible}}=\set{\vec{x}\in{}A:\deg{}Df(\vec{x})\ne{}0}$.\n
$E$ is an open set containing $\vec{a}$. The inverse function theorem doesn't assume $E=A$, but it could.\n

\par\noindent\textbf{Proof of the Inverse Function Theorem}\n

\par\noindent Preliminaries: Let $T_{\vec{a}}:\vec{x}\mapsto\vec{x}+\vec{a}$.\n
\phantom{Preliminaries:} $DT_{\vec{a}}=\Id$\n
\phantom{Preliminaries:} $g=Df(\vec{a})\inv\of{}T_{-f(\vec{a})}\of{}f\of{}T_{\vec{a}}$\n

\par\noindent Check: $\left.\begin{array}{l}g(\vec{0})=\vec{0}\\ Dg(\vec{0})=\Id\\ f=T_{f(\vec{a})}\of{}Df(\vec{a})\of{}g\of{}T_{-\vec{a}}\end{array}\right\}$ E.T.S. $g$ is a $C^{r}$-diffeomorphism on some open set containing $\vec{0}$.\n
\n

\par\noindent In proving the inverse function theorem, we may assume $\vec{a}=\vec{0}$, and $Dg(\vec{0})=\Id$.\n

\par\noindent Let $h=g-\Id$, $Dh=Dg-\Id$, $Dh(\vec{0})=0$, $Dh:A\to\Mat(n,m)$ is continuous.\n
Fix $0<\epsilon<1$. Then $\exists\delta>0$ \st{} $\norm{Dh}<\epsilon$ on $\uball(\vec{0},\delta)$. ($\norm{Dh}$ is defined in the HW3 handout.)\n

\lemma{Given $A^{\ptxt{convex open}}\subset{}V$, $\varphi:V\to{}W$ differentiable, $\norm{D\varphi(\vec{p})}\le{}M\;\forall\vec{p}\in{}A$.\n
Then $\norm{\varphi(\vec{y})-\varphi(\vec{x})}\le{}M\norm{\vec{y}-\vec{x}}\;\forall\vec{x},\vec{y}\in{}A$.\n
Proof: HW 5.\proven}

\par\noindent So, for $\vec{x}\in\uball(\vec{0},\delta)$, we have $\norm{h(\vec{x})}=\norm{h(\vec{x})-h(\vec{0})}\le\epsilon\norm{\vec{x}}$ ($\star$).\n

\par\noindent Also, for $\vec{x},\vec{y}\in\uball(\vec{0},\delta)$, we have\n
$(1-\epsilon)\norm{\vec{y}-\vec{x}}\le\norm{\vec{y}-\vec{x}}-\norm{h(\vec{y})-h(\vec{x})}\le\norm{(\vec{y}-\vec{x})+(h(\vec{y})-h(\vec{x}))}=\norm{g(\vec{y})-g(\vec{x})}$\n
$\norm{(\vec{y}-\vec{x})+(h(\vec{y})-h(\vec{x}))}\le\norm{\vec{y}-\vec{x}}+\norm{h(\vec{y})-h(\vec{x})}\le(1+\epsilon)\norm{\vec{y}-\vec{x}}$.\n

\par\noindent So $(1-\epsilon)\norm{\vec{y}-\vec{x}}\le\norm{g(\vec{y})-g(\vec{x})}\le(1+\epsilon)\norm{\vec{y}-\vec{x}}$. Thus $g$ is Bi-Lipschitz on $\uball(\vec{0},\delta)$.\n
And $g(\uball(\vec{0},\delta))\subset\uball(\vec{0},(1+\epsilon)\delta)$.\n
And $g$ is injective on $\uball(\vec{0},\delta)$.\n

\par\noindent\flag{TO BE CONTINUED...}

\end{document}