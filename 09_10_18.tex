\documentclass[10pt,letterpaper]{article}
\usepackage[utf8]{inputenc}
\usepackage{amsmath}
\usepackage{amsfonts}
\usepackage{amssymb}
\usepackage{ragged2e}
\usepackage[letterpaper, margin=1in]{geometry}
\usepackage{graphicx}
\usepackage{cancel}
\usepackage{mathtools}
\usepackage{tabularx}
\usepackage{arydshln}
\usepackage{tensor}
\usepackage{array}
\usepackage{xcolor}

%%%%%%%%%%%%%%%%%%%%%%%%%%%%%
% Formatting commands
%%%%%%%%%%%%%%%%%%%%%%%%%%%%%
\newcommand{\n}{\hfill\break}
\newcommand{\lemma}[1]{\par\noindent\settowidth{\hangindent}{\textbf{Lemma: }}\textbf{Lemma: }#1\n}
\newcommand{\defn}[1]{\par\noindent\settowidth{\hangindent}{\textbf{Defn: }}\textbf{Defn: }#1\n}
\newcommand{\thm}[1]{\par\noindent\settowidth{\hangindent}{\textbf{Thm: }}\textbf{Thm: }#1\n}
\newcommand{\prop}[1]{\par\noindent\settowidth{\hangindent}{\textbf{Prop: }}\textbf{Prop: }#1\n}
\newcommand{\ex}[1]{\par\noindent\settowidth{\hangindent}{\textbf{Ex: }}\textbf{Ex: }#1\n}
\newcommand{\proven}{\;$\square$\n}
\newcommand{\problem}[1]{\par\noindent{#1}\n}
\newcommand{\problempart}[2]{\par\settowidth{\hangindent}{\textbf{(#1)} \indent{}}\textbf{(#1)} #2\n}
\newcommand{\ptxt}[1]{\textrm{\textnormal{#1}}}
\newcommand{\inlineeq}[1]{\n\centerline{$\displaystyle #1$}}
\newcommand{\pageline}{\noindent\rule{\textwidth}{0.1pt}}

%%%%%%%%%%%%%%%%%%%%%%%%%%%%%
% Math commands
%%%%%%%%%%%%%%%%%%%%%%%%%%%%%
% Set Theory
\newcommand{\card}[1]{\left|#1\right|}
\newcommand{\set}[1]{\left\{#1\right\}}
\newcommand{\ps}[1]{\mathcal{P}\left(#1\right)}
\newcommand{\pfinite}[1]{\mathcal{P}^{\ptxt{finite}}\left(#1\right)}
\newcommand{\naturals}{\mathbb{N}}
\newcommand{\N}{\naturals}
\newcommand{\integers}{\mathbb{Z}}
\newcommand{\Z}{\integers}
\newcommand{\rationals}{\mathbb{Q}}
\newcommand{\Q}{\rationals}
\newcommand{\reals}{\mathbb{R}}
\newcommand{\R}{\reals}
\newcommand{\complex}{\mathbb{C}}
\newcommand{\C}{\complex}
\newcommand{\comp}{^{\complement}}

% Graph Theory
\renewcommand{\deg}[1]{\ptxt{deg}\left(#1\right)}

% Standard Math
\newcommand{\inv}{^{-1}}
\newcommand{\abs}[1]{\left|#1\right|}
\newcommand{\ceil}[1]{\left\lceil{}#1\right\rceil}
\newcommand{\floor}[1]{\left\lfloor{}#1\right\rfloor{}}
\newcommand{\conj}[1]{\overline{#1}}
\newcommand{\of}{\circ}
\newcommand{\tri}{\triangle}
\newcommand{\inj}{\hookrightarrow}
\newcommand{\surj}{\twoheadrightarrow}
\newcommand{\mapsfrom}{\mathrel{\reflectbox{\ensuremath{\mapsto}}}}

% Linear Algebra
\newcommand{\Id}{\textrm{\textnormal{Id}}}
\newcommand{\im}{\textrm{\textnormal{im}}}
\newcommand{\norm}[1]{\abs{\abs{#1}}}
\newcommand{\tpose}{^{T}}
\newcommand{\iprod}[1]{\left<#1\right>}
\newcommand{\trace}{\ptxt{tr}}
\newcommand{\chgBasMat}[3]{\!\!\tensor*[_{#1}]{\left[#2\right]}{_{#3}}}
\newcommand{\vecBas}[2]{\tensor*[]{\left[#1\right]}{_{#2}}}

% Topology
\newcommand{\closure}[1]{\bar{#1}}
\newcommand{\uball}{\mathcal{U}}
\newcommand{\Int}{\ptxt{Int}\>}
\newcommand{\Ext}{\ptxt{Ext}\>}
\newcommand{\Bd}{\ptxt{Bd}\>}

% Proofs
\newcommand{\st}{s.t.}
\newcommand{\unique}{!}

%%%%%%%%%%%%%%%%%%%%%%%%%%%%%
% Other commands
%%%%%%%%%%%%%%%%%%%%%%%%%%%%%
\newcommand{\flag}[1]{\textbf{\textcolor{red}{#1}}}

%%%%%%%%%%%%%%%%%%%%%%%%%%%%%
% Make l's curvy in math environments
%%%%%%%%%%%%%%%%%%%%%%%%%%%%%
\mathcode`l="8000
\begingroup
\makeatletter
\lccode`\~=`\l
\DeclareMathSymbol{\lsb@l}{\mathalpha}{letters}{`l}
\lowercase{\gdef~{\ifnum\the\mathgroup=\m@ne \ell \else \lsb@l \fi}}%
\endgroup

\author{Professor David Barrett\\ \small\textit{Transcribed by Thomas Cohn}}
\title{Metric Spaces}
\date{9/10/18} % Can also use \today

\begin{document}
\maketitle
\setlength\RaggedRightParindent{\parindent}
\RaggedRight

\par\noindent Exercises to study: Munkres \S{3} \#1,3,5\n

\defn{A \underline{metric} on set $X$ is a function $d:X\times{}X\to\R$ \st\n
(1) $d(x,y)=d(y,x)$\n
(2) a) $d(x,y)\ge{}0$\n
\phantom{(2) }b) $d(x,y)=0\leftrightarrow{}x=y$\n
\phantom{(2) }c) $d(x,z)\le{}d(x,y)+d(y,z)$.}

\defn{A \underline{metric space} is a set $X$ equipped with a metric $d$.}

\defn{With $Y\subset{}X$, $d|_{Y\times{}Y}$ is a metric on $Y$, called the \underline{induced metric}.}

\par\noindent The most important exapmles for the Munkres material:
\begin{enumerate}
	\item $X=\R^{n}$, $d_{\ptxt{eucl}}(\vec{x},\vec{y})=\sqrt{\sum_{j=1}^{n}(y_{j}-x_{j})^{2}}$.
	\item $Y\subset{}X$ with the induced metric.
\end{enumerate}

\defn{For $x_{0}\in{}X$, $\varepsilon>0$, the set $\uball(x_{0},\varepsilon)=\set{x\in{}X:d(x_{0},x)<\varepsilon}$. This is the \underline{$\varepsilon$-neighborhood of $x_{0}$}, or the \underline{$\varepsilon$-ball centered at $x_{0}$}.}

\par\noindent Consider $A\subset{}X$.\n

\defn{$x_{0}\in{}X$ is \underline{interior} to $A$ $\leftrightarrow$ $\exists\varepsilon>0$ \st{} $\uball(x_{0},\varepsilon)\subset{}A$.\n
$\Int{}A$ is the set of interior points to $A$.}
\defn{$x_{0}\in{}X$ is \underline{exterior} to $A$ $\leftrightarrow$ $\exists\varepsilon>0$ \st{} $\uball(x_{0},\varepsilon)\cap{}A=\emptyset$.\n
$\Ext{}A$ is the set of exterior points to $A$.}
\defn{$x_{0}\in{}X$ is a \underline{boundary point} of $A$ $\leftrightarrow$ $x_{0}$ is neither interior nor exterior to $A$.\n
\phantom{$x_{0}\in{}X$ is a \underline{boundary point} of $A$ }$\leftrightarrow$ each $\uball(x_{0},\varepsilon)$ intersects $A$ and $X\setminus{}A$.\n
$\Bd{}A$ is the set of boundary points of $A$.}

\par\noindent Note that we have $X=\Int{}A\sqcup\Ext{}A\sqcup\Bd{}A$. Note that $\sqcup$ can have multiple meanings (but it's ok for us to use it this way). For more on the different meanings of ``$\sqcup$'', look up ``disjoint union'' on Wikipedia.\n

\par\noindent Note that $\Ext{}A=\Int(X\setminus{}A)$.\n

\defn{$A$ is \underline{open} $\leftrightarrow$ $A=\Int{}A$.}

\prop{This defines a topology on $X$.\n
Proof: See Munkres \S 3.}

\par\noindent Some facts:
\begin{enumerate}
	\item Each $\uball(x_{0},\varepsilon)$ is open (Munkres \S 3 \#1).
	\item $\Int{}A$ is the largest open subset of $A$.
	\item $A$ is closed $\leftrightarrow$ $A=\Int{}A\cup\Bd{}A$.
	\item $\closure{A}$ is defined as $\Int{}A\cup\Bd{}A$.
	\item $\closure{A}$ is the smallest closed superset of $A$.
	\item $\Bd(X\setminus{}A)=\Bd{}A$.
	\item $\Bd{}A$ is closed.
\end{enumerate}\n

\defn{Given $(x_{n})$ sequence in $X$ $(x_{n}):\N\to{}X$, $n\mapsto{}x_{n}$, $(X,d)$ metric\n
$x_{n}\to{}x\leftrightarrow\forall\varepsilon>0$, $\exists{}N\in\N$ \st{} $d(x_{n},x)<\varepsilon$ when $n>N$.\n
$(x_{n})$ \underline{converges} if $\exists{}x$ \st{} $x_{n}\to{}x$.}

\par\noindent If $x_{n}\to{}x$ and $x_{n}\to{}y$, then $x=y$.\n

\defn{$f:(X,d_{x})\to(Y,d_{y})$ is \underline{sequentially continuous} $\leftrightarrow$ $x_{n}\to{}x$ implies $f(x_{n})\to{}f(x)$.}

\thm{$f$ is sequentially continuous $\leftrightarrow$ $f$ is continuous.\n
Proof: We basically did this last year.}

\defn{$(x_{n})$ is \underline{Cauchy} $\leftrightarrow$ $\forall\varepsilon>0$ $\exists{}N\in\N$ \st{} $d(x_{n},x_{m})<\varepsilon$ $\forall{}n,m>N$.}

\defn{A metric space is \underline{complete} iff all Cauchy sequences converge.}

\par\noindent Some facts:
\begin{enumerate}
	\item $(\R^{n},d_{\ptxt{eucl}})$ is complete.
	\item For $Y\subset\R^{n}$, $(Y,d_{\ptxt{eucl}})$ is complete iff $Y$ is closed.
\end{enumerate}\n

\par\noindent $Z\subset{}X$ is closed $\leftrightarrow$ $Z$ is sequentially closed.\n

\par\noindent Weird example of a metric: We can map between $\R$ and $\left(-\frac{\pi}{2},\frac{\pi}{2}\right)$ with $\tan$ and $\arctan$. Let $\widetilde{d}(x,y)=\abs{\arctan{x}-\arctan{y}}$.\n

\defn{A topological space $X$ is \underline{compact} $\leftrightarrow$ every open cover of $X$ has a finite subcover.\n
\phantom{A topological space $X$ is \underline{compact} }$\leftrightarrow$ if $X=\bigcup_{\alpha\in{}A}X_{\alpha}$ with $X_{\alpha}$ open,\n
\phantom{\phantom{A topological space $X$ is \underline{compact} }$\leftrightarrow$ }then $\exists{}\alpha_{1},\ldots,\alpha_{k}\in{}A$ \st{} $X=X_{\alpha_{1}}\cup\cdots\cup{}X_{\alpha_{k}}$.}

\thm{(Bolzano-Weierstrass) $(X,d)$ is compact if and only if every sequence in $X$ admits a convergent subsequence (``sequential compactness'').\n
Proof: We will do this on Wednesday.}

\thm{(Heine-Borel) If $Y\subset\R^{n}$ then $(Y,d_{\ptxt{eucl}})$ is compact $\leftrightarrow$ $Y$ is closed and bounded.\n
Proof: Math 296 \#158 or Math 297 \#100 or Munkres Thms 4.3, 4.9.}

\ex{Set $S$, with $\pfinite{test}$.\n
$d_{1}(A,B)=\left\{\begin{array}{ll}0 & A=B\\ 1 & A\ne{}B\end{array}\right.$ gives us the discrete topology.\n
$d_{2}(A,B)=\card{A\tri{}B}$ \textit{also} gives us the discrete topology.}

\end{document}