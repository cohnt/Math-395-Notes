\documentclass[10pt,letterpaper]{article}
\usepackage[utf8]{inputenc}
\usepackage{amsmath}
\usepackage{amsfonts}
\usepackage{amssymb}
\usepackage{ragged2e}
\usepackage[letterpaper, margin=1in]{geometry}
\usepackage{graphicx}
\usepackage{cancel}
\usepackage{mathtools}
\usepackage{tabularx}
\usepackage{arydshln}
\usepackage{tensor}
\usepackage{array}
\usepackage{xcolor}
\usepackage[boxed]{algorithm}
\usepackage[noend]{algpseudocode}
\usepackage{listings}
\usepackage{textcomp}
\usepackage[pdf,tmpdir,singlefile]{graphviz}
\usepackage{mathrsfs}

%%%%%%%%%%%%%%%%%%%%%%%%%%%%%
% Formatting commands
%%%%%%%%%%%%%%%%%%%%%%%%%%%%%
\newcommand{\n}{\hfill\break}
\newcommand{\lemma}[1]{\par\noindent\settowidth{\hangindent}{\textbf{Lemma: }}\textbf{Lemma: }#1}
\newcommand{\defn}[1]{\par\noindent\settowidth{\hangindent}{\textbf{Defn: }}\textbf{Defn: }#1\n}
\newcommand{\thm}[1]{\par\noindent\settowidth{\hangindent}{\textbf{Thm: }}\textbf{Thm: }#1\n}
\newcommand{\prop}[1]{\par\noindent\settowidth{\hangindent}{\textbf{Prop: }}\textbf{Prop: }#1\n}
\newcommand{\cor}[1]{\par\noindent\settowidth{\hangindent}{\textbf{Cor: }}\textbf{Cor: }#1\n}
\newcommand{\ex}[1]{\par\noindent\settowidth{\hangindent}{\textbf{Ex: }}\textbf{Ex: }#1\n}
\newcommand{\proven}{\;$\square$\n}
\newcommand{\problem}[1]{\par\noindent{#1}\n}
\newcommand{\problempart}[2]{\par\noindent\indent{}\settowidth{\hangindent}{\textbf{(#1)} \indent{}}\textbf{(#1)} #2\n}
\newcommand{\ptxt}[1]{\textrm{\textnormal{#1}}}
\newcommand{\inlineeq}[1]{\centerline{$\displaystyle #1$}}
\newcommand{\pageline}{\noindent\rule{\textwidth}{0.1pt}}

%%%%%%%%%%%%%%%%%%%%%%%%%%%%%
% Math commands
%%%%%%%%%%%%%%%%%%%%%%%%%%%%%
% Set Theory
\newcommand{\card}[1]{\left|#1\right|}
\newcommand{\set}[1]{\left\{#1\right\}}
\newcommand{\ps}[1]{\mathcal{P}\left(#1\right)}
\newcommand{\pfinite}[1]{\mathcal{P}^{\ptxt{finite}}\left(#1\right)}
\newcommand{\naturals}{\mathbb{N}}
\newcommand{\N}{\naturals}
\newcommand{\integers}{\mathbb{Z}}
\newcommand{\Z}{\integers}
\newcommand{\rationals}{\mathbb{Q}}
\newcommand{\Q}{\rationals}
\newcommand{\reals}{\mathbb{R}}
\newcommand{\R}{\reals}
\newcommand{\complex}{\mathbb{C}}
\newcommand{\C}{\complex}
\newcommand{\comp}{^{\complement}}
\newcommand{\Hom}{\ptxt{Hom}\>}

% Graph Theory
\renewcommand{\deg}[1]{\ptxt{deg}\left(#1\right)}
\newcommand{\degp}[1]{\ptxt{deg}^{+}\!\!\left(#1\right)}
\newcommand{\degn}[1]{\ptxt{deg}^{-}\!\!\left(#1\right)}
\newcommand{\Prob}{\mathbb{P}}
\newcommand{\Avg}{\mathbb{E}}

% Standard Math
\newcommand{\inv}{^{-1}}
\newcommand{\abs}[1]{\left|#1\right|}
\newcommand{\ceil}[1]{\left\lceil{}#1\right\rceil}
\newcommand{\floor}[1]{\left\lfloor{}#1\right\rfloor{}}
\newcommand{\conj}[1]{\overline{#1}}
\newcommand{\of}{\circ}
\newcommand{\tri}{\triangle}
\newcommand{\inj}{\hookrightarrow}
\newcommand{\surj}{\twoheadrightarrow}
\newcommand{\mapsfrom}{\mathrel{\reflectbox{\ensuremath{\mapsto}}}}
\newcommand{\Graph}{\ptxt{Graph}\>}
\newcommand{\ndiv}{\nmid}
\renewcommand{\epsilon}{\varepsilon}

% Linear Algebra
\newcommand{\Id}{\textrm{\textnormal{Id}}}
\newcommand{\im}{\textrm{\textnormal{im}}}
\newcommand{\norm}[1]{\abs{\abs{#1}}}
\newcommand{\tpose}{^{T}}
\newcommand{\iprod}[1]{\left<#1\right>}
\newcommand{\trace}{\ptxt{tr}}
\newcommand{\chgBasMat}[3]{\!\!\tensor*[_{#1}]{\left[#2\right]}{_{#3}}}
\newcommand{\vecBas}[2]{\tensor*[]{\left[#1\right]}{_{#2}}}
\newcommand{\GL}{\ptxt{GL}\>}
\newcommand{\Mat}{\ptxt{Mat}\>}
\newcommand{\Span}{\ptxt{Span}}
\newcommand{\rank}{\ptxt{rank}\>}

% Topology
\newcommand{\closure}[1]{\overline{#1}}
\newcommand{\uball}{\mathcal{U}}
\newcommand{\Int}{\ptxt{Int}\>}
\newcommand{\Ext}{\ptxt{Ext}\>}
\newcommand{\Bd}{\ptxt{Bd}\>}
\newcommand{\rInt}{\ptxt{rInt}\>}

% Proofs
\newcommand{\st}{s.t.}
\newcommand{\unique}{!}

% Algorithms
\algrenewcommand{\algorithmiccomment}[1]{\hskip 1em \texttt{// #1}}
\algrenewcommand\algorithmicrequire{\textbf{Input:}}
\algrenewcommand\algorithmicensure{\textbf{Output:}}

%%%%%%%%%%%%%%%%%%%%%%%%%%%%%
% Other commands
%%%%%%%%%%%%%%%%%%%%%%%%%%%%%
\newcommand{\flag}[1]{\textbf{\textcolor{red}{#1}}}

%%%%%%%%%%%%%%%%%%%%%%%%%%%%%
% Make l's curvy in math environments
%%%%%%%%%%%%%%%%%%%%%%%%%%%%%
\mathcode`l="8000
\begingroup
\makeatletter
\lccode`\~=`\l
\DeclareMathSymbol{\lsb@l}{\mathalpha}{letters}{`l}
\lowercase{\gdef~{\ifnum\the\mathgroup=\m@ne \ell \else \lsb@l \fi}}%
\endgroup

\author{Thomas Cohn}
\title{SpOoOoOky Halloween Lecture}
\date{10/31/18} % Can also use \today

\begin{document}
\maketitle
\setlength\RaggedRightParindent{\parindent}
\RaggedRight

\prop{$K^{\ptxt{cpt}}\subset\R^{n}\Rightarrow{}M^{*,J}(K)=m^{*}(K)$\n
Proof: We know $m^{*}(K)\le{}m^{*,J}(K)$ is always true. So it is enough to show $m^{*}(K)\ge{}m^{*,J}(K)$.\n
Pick boxes $Q_{j}$ ($j=1,2,\ldots$) with $\bigcup_{j=1}^{\infty}\rInt{}Q_{j}\supset{}K$.\n
Compactness implies that $\bigcup_{j=1}^{M}\rInt{}Q_{j}\supset{}K$. So\n
\inlineeq{\sum_{j=1}^{\infty}v(Q_{j})\ge\sum_{j=1}^{M}v(Q_{j})\ge{}m^{*,J}(K)}\n
Now we take the $\inf$ over the choice of $Q_{j}$'s.\n
Therefore $m^{*}(K)\ge{}M^{*,J}(K)$.\proven}

\par\noindent Recall the theorem from Friday:\n
For bounded $f:Q^{\ptxt{box}}\to\R$, the following are equivalent\n
\begin{enumerate}
	\item $f$ is integrable
	\item 
	\item 
	\item $\mathscr{D}\overset{\ptxt{def}}{=}\set{\vec{x}\in{}Q:f\ptxt{ not cts at }\vec{x}}$ has $m^{*}(\mathscr{D})=0$.
	\item 
\end{enumerate}

\prop{$\card{S_{1}\cup\cdots\cup{}S_{k}}\le\card{S_{1}}+\cdots+\card{S_{k}}$. (Cardinality maps to $\N\cup\set{0,+\infty}$).\n
We call this property finite subadditivity.\n
Proof 1: Induction.\n
Proof 2: Use $S_{1}\cup\cdots\cup{}S_{k}=S_{1}\sqcup(S_{2}\setminus{}S_{1})\sqcup(S_{3}\setminus(S_{1}\cup{}S_{2}))\sqcup\cdots\sqcup(S_{k}-\bigcup_{i=1}^{k-1}S_{i})$\n
$\card{S_{1}}+\cdots+\card{S_{k}}\le\card{RHS}=\card{S_{1}\cup\cdots\cup{}S_{k}}$.\proven}

\lemma{Given $B_{1}\cup\cdots\cup{}B_{j}\subset{}X_{1}\cup\cdots\cup{}X_{k}$ (all boxes), with $\Int{}B_{l}$ disjoint.\n
Then $v(B_{1})+\cdots+v(B_{j})\le{}v(X_{1})+\cdots+v(X_{n})$.\n
Proof 1: Chop into smaller pieces.\n
Proof 2: Exercise: $R$ box $\to{}v(R)=m_{\ptxt{pixel}}(R)=m_{\ptxt{pixel}}(\Int{}R)=m_{\ptxt{pixel}}(\rInt{}R)$\n
Where $\displaystyle{}m_{\ptxt{pixel}}(E)=\lim_{N\to\infty}\card{E\cap\frac{\Z^{n}}{2^{N}}}$\n
\n
$\displaystyle\sum_{l=1}\card{\Int{}B_{l}\cap\frac{\Z^{n}}{2^{nN}}}\le\card{\bigcup{}X_{p}\cap\frac{\Z^{n}}{2^{nN}}}\le\sum_{p=1}^{k}\card{X_{p}\cap\frac{\Z^{n}}{2^{nN}}}$\n
\n
Therefore $\sum{}m_{\ptxt{pixel}}(\Int{}B_{l})=v(B_{l})$.\proven}

\prop{Given $f$ integrable on $Q^{\ptxt{box}}$, $f\ge{}0$ on $Q$.\n
Then $\int_{Q}f=0$ iff $m^{*}(f\inv[(0,+\infty)])=0$.\n
Proof $\Rightarrow$: $f\inv[(0,+\infty)]\subset\mathscr{D}\cup\set{\vec{a}\in{}Q:f\ptxt{ is cts and positive at }\vec{a}}$.\n
For $\vec{a}$ in the second set, $\exists{}B^{\ptxt{box}}\supset\rInt{}B\ni\vec{a}$ with $f\ge\frac{f(\vec{a})}{2}\mathbb{I}_{B}$.\n
Then $\int_{Q}f\ge\int_{Q}\frac{f(\vec{a})}{2}\mathbb{I}_{B}\overset{\ptxt{exer}}{=}\frac{f(\vec{a})}{2}v(B)>0$. Oops! Hence no such $\vec{a}$ exists.\n
\n
Proof $\Leftarrow$: $f\inv[(0,+\infty)]$ contains no boxes of positive volume. To show this, it's enough to prove $m^{*}(B^{\ptxt{positive volume box}})>0$. This is equal to $m^{*,J}(B)$, so this is all true by previous lemma.\n
So each $L(f,P)=0$, so $\underline{\int_{Q}}f=0$, so $\int_{Q}f=0$.}

\par\noindent Consider $S^{\ptxt{bdd}}\subset\R^{n}$, with $f:S\to\R$ bounded, $f_{S}(\vec{x})=\left\{\begin{array}{ll}f(\vec{x}) & \quad\vec{x}\in{}S\\ 0 & \quad\vec{x}\not\in{}S\end{array}\right.$, and\n
$\int_{S}f\overset{\ptxt{def}}{=}\int_{Q}f_{S}$ for $Q^{\ptxt{box}}\supset{}S$.\n

\prop{Existence and value of $\int_{Q}f_{s}$ do not depend on the choice of $Q$.\n
Proof: Choose $Q_{3}$ \st{} $\overline{S},Q_{1},Q_{2}\supset\Int{}Q_{3}$. Then the discontinuities set for $Q_{3}$ is equal to the discontinuities set for $f$ on $S$.}

\end{document}