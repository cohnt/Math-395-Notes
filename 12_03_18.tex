\documentclass[10pt,letterpaper]{article}
\usepackage[utf8]{inputenc}
\usepackage[intlimits]{amsmath}
\usepackage{amsfonts}
\usepackage{amssymb}
\usepackage{ragged2e}
\usepackage[letterpaper, margin=1in]{geometry}
\usepackage{graphicx}
\usepackage{cancel}
\usepackage{mathtools}
\usepackage{tabularx}
\usepackage{arydshln}
\usepackage{tensor}
\usepackage{array}
\usepackage{xcolor}
\usepackage[boxed]{algorithm}
\usepackage[noend]{algpseudocode}
\usepackage{listings}
\usepackage{textcomp}
\usepackage[pdf,tmpdir,singlefile]{graphviz}
\usepackage{mathrsfs}
\usepackage{bbm}
\usepackage{tikz}
\usepackage{enumitem}
\usepackage{arydshln}

%%%%%%%%%%%%%%%%%%%%%%%%%%%%%
% Formatting commands
%%%%%%%%%%%%%%%%%%%%%%%%%%%%%
\newcommand{\n}{\hfill\break}
\newcommand{\up}{\vspace{-\baselineskip}}
\newcommand{\lemma}[1]{\par\noindent\settowidth{\hangindent}{\textbf{Lemma: }}\textbf{Lemma: }#1}
\newcommand{\defn}[1]{\par\noindent\settowidth{\hangindent}{\textbf{Defn: }}\textbf{Defn: }#1\n}
\newcommand{\thm}[1]{\par\noindent\settowidth{\hangindent}{\textbf{Thm: }}\textbf{Thm: }#1\n}
\newcommand{\prop}[1]{\par\noindent\settowidth{\hangindent}{\textbf{Prop: }}\textbf{Prop: }#1\n}
\newcommand{\cor}[1]{\par\noindent\settowidth{\hangindent}{\textbf{Cor: }}\textbf{Cor: }#1\n}
\newcommand{\ex}[1]{\par\noindent\settowidth{\hangindent}{\textbf{Ex: }}\textbf{Ex: }#1\n}
\newcommand{\proven}{\;$\square$\n}
\newcommand{\problem}[1]{\par\noindent{#1}\n}
\newcommand{\problempart}[2]{\par\noindent\indent{}\settowidth{\hangindent}{\textbf{(#1)} \indent{}}\textbf{(#1)} #2\n}
\newcommand{\ptxt}[1]{\textrm{\textnormal{#1}}}
\newcommand{\inlineeq}[1]{\centerline{$\displaystyle #1$}}
\newcommand{\pageline}{\noindent\rule{\textwidth}{0.1pt}}

%%%%%%%%%%%%%%%%%%%%%%%%%%%%%
% Math commands
%%%%%%%%%%%%%%%%%%%%%%%%%%%%%
% Set Theory
\newcommand{\card}[1]{\left|#1\right|}
\newcommand{\set}[1]{\left\{#1\right\}}
\newcommand{\ps}[1]{\mathcal{P}\left(#1\right)}
\newcommand{\pfinite}[1]{\mathcal{P}^{\ptxt{finite}}\left(#1\right)}
\newcommand{\naturals}{\mathbb{N}}
\newcommand{\N}{\naturals}
\newcommand{\integers}{\mathbb{Z}}
\newcommand{\Z}{\integers}
\newcommand{\rationals}{\mathbb{Q}}
\newcommand{\Q}{\rationals}
\newcommand{\reals}{\mathbb{R}}
\newcommand{\R}{\reals}
\newcommand{\complex}{\mathbb{C}}
\newcommand{\C}{\complex}
\newcommand{\comp}{^{\complement}}
\DeclareMathOperator{\Hom}{Hom}
\newcommand{\Ind}{\mathbbm{1}}
\newcommand{\cut}{\setminus}

% Graph Theory
\let\deg\relax
\DeclareMathOperator{\deg}{deg}
\newcommand{\degp}{\ptxt{deg}^{+}}
\newcommand{\degn}{\ptxt{deg}^{-}}
\newcommand{\precdot}{\mathrel{\ooalign{$\prec$\cr\hidewidth\hbox{$\cdot\mkern0.5mu$}\cr}}}
\newcommand{\succdot}{\mathrel{\ooalign{$\cdot\mkern0.5mu$\cr\hidewidth\hbox{$\succ$}\cr\phantom{$\succ$}}}}
\DeclareMathOperator{\cl}{cl}
\DeclareMathOperator{\affdim}{affdim}

% Probability
\newcommand{\Prob}{\mathbb{P}}
\newcommand{\Avg}{\mathbb{E}}

% Standard Math
\newcommand{\inv}{^{-1}}
\newcommand{\abs}[1]{\left|#1\right|}
\newcommand{\ceil}[1]{\left\lceil{}#1\right\rceil{}}
\newcommand{\floor}[1]{\left\lfloor{}#1\right\rfloor{}}
\newcommand{\conj}[1]{\overline{#1}}
\newcommand{\of}{\circ}
\newcommand{\tri}{\triangle}
\newcommand{\inj}{\hookrightarrow}
\newcommand{\surj}{\twoheadrightarrow}
\newcommand{\mapsfrom}{\mathrel{\reflectbox{\ensuremath{\mapsto}}}}
\newcommand{\mapsdown}{\rotatebox[origin=c]{-90}{$\mapsto$}\mkern2mu}
\newcommand{\mapsup}{\rotatebox[origin=c]{90}{$\mapsto$}\mkern2mu}
\newcommand{\ndiv}{\nmid}
\renewcommand{\epsilon}{\varepsilon}
\newcommand{\divides}{\mid}
\newcommand{\ndivides}{\nmid}
\DeclareMathOperator{\lcm}{lcm}

% Linear Algebra
\newcommand{\Id}{\textrm{\textnormal{Id}}}
\newcommand{\im}{\textrm{\textnormal{im}}}
\newcommand{\norm}[1]{\abs{\abs{#1}}}
\newcommand{\tpose}{^{T}}
\newcommand{\iprod}[1]{\left<#1\right>}
\DeclareMathOperator{\trace}{tr}
\newcommand{\chgBasMat}[3]{\!\!\tensor*[_{#1}]{\left[#2\right]}{_{#3}}}
\newcommand{\vecBas}[2]{\tensor*[]{\left[#1\right]}{_{#2}}}
\DeclareMathOperator{\GL}{GL}
\DeclareMathOperator{\Mat}{Mat}
\DeclareMathOperator{\vspan}{span}
\DeclareMathOperator{\rank}{rank}
\newcommand{\V}[1]{\vec{#1}}

% Topology
\newcommand{\closure}[1]{\overline{#1}}
\newcommand{\uball}{\mathcal{U}}
\DeclareMathOperator{\Int}{Int}
\DeclareMathOperator{\Ext}{Ext}
\DeclareMathOperator{\Bd}{Bd}
\DeclareMathOperator{\rInt}{rInt}
\DeclareMathOperator{\ch}{ch}
\DeclareMathOperator{\ah}{ah}

% Analysis
\DeclareMathOperator{\Graph}{Graph}
\DeclareMathOperator{\epi}{epi}
\DeclareMathOperator{\hypo}{hypo}
\DeclareMathOperator{\supp}{supp}
\newcommand{\lint}[2]{\underset{#1}{\overset{#2}{{\color{black}\underline{{\color{white}\overline{{\color{black}\int}}\color{black}}}}}}}
\newcommand{\uint}[2]{\underset{#1}{\overset{#2}{{\color{white}\underline{{\color{black}\overline{{\color{black}\int}}\color{black}}}}}}}
\newcommand{\alignint}[2]{\underset{#1}{\overset{#2}{{\color{white}\underline{{\color{white}\overline{{\color{black}\int}}\color{black}}}}}}}
\newcommand{\extint}{\ptxt{ext}\int}
\newcommand{\extalignint}[2]{\ptxt{ext}\alignint{#1}{#2}}
\newcommand{\conv}{\ast}

% Proofs
\newcommand{\st}{s.t.}
\newcommand{\unique}{!}
\newcommand{\iffdef}{\overset{\ptxt{def}}{\Leftrightarrow}}
\newcommand{\eqdef}{\overset{\ptxt{def}}{=}}

% Brackets
\newcommand{\paren}[1]{\left(#1\right)}
\renewcommand{\brack}[1]{\left[#1\right]}
\renewcommand{\brace}[1]{\left\{#1\right\}}
\newcommand{\ang}[1]{\left<#1\right>}

% Algorithms
\algrenewcommand{\algorithmiccomment}[1]{\hskip 1em \texttt{// #1}}
\algrenewcommand\algorithmicrequire{\textbf{Input:}}
\algrenewcommand\algorithmicensure{\textbf{Output:}}
\newcommand{\parSymbol}{\P}
\renewcommand{\P}{\ptxt{\textbf{P}}}
\newcommand{\NP}{\ptxt{\textbf{NP}}}
\newcommand{\NPC}{\ptxt{\textbf{NP-Complete}}}
\newcommand{\NPH}{\ptxt{\textbf{NP-Hard}}}
\newcommand{\EXP}{\ptxt{\textbf{EXP}}}

%%%%%%%%%%%%%%%%%%%%%%%%%%%%%
% Other commands
%%%%%%%%%%%%%%%%%%%%%%%%%%%%%
\newcommand{\flag}[1]{\textbf{\textcolor{red}{#1}}}

%%%%%%%%%%%%%%%%%%%%%%%%%%%%%
% Make l's curvy in math environments
%%%%%%%%%%%%%%%%%%%%%%%%%%%%%
\mathcode`l="8000
\begingroup
\makeatletter
\lccode`\~=`\l
\DeclareMathSymbol{\lsb@l}{\mathalpha}{letters}{`l}
\lowercase{\gdef~{\ifnum\the\mathgroup=\m@ne \ell \else \lsb@l \fi}}%
\endgroup

\newcommand{\B}{
    \begin{tikzpicture}
    \filldraw [fill=red, draw=black] (0, 0) rectangle (0.37, 0.45);
    \draw [line width=0.5mm, white ] (0.1,0.08) -- (0.1,0.38);
    \draw[line width=0.5mm, white ] (0.1, 0.35) .. controls (0.2, 0.35) and (0.4, 0.2625) .. (0.1, 0.225);
    \draw[line width=0.5mm, white ] (0.1, 0.225) .. controls (0.2, 0.225) and (0.4, 0.1625) .. (0.1, 0.1);
    \end{tikzpicture}
}

\author{Thomas Cohn}
\title{Manifolds-Without-Boundary}
\date{12/3/18} % Can also use \today

\begin{document}
\maketitle
\setlength\RaggedRightParindent{\parindent}
\RaggedRight

\defn{$M\subseteq\R^{n}$ is a \underline{$C^{r}$ $k$-manifold-without-boundary} $\iffdef$ $U^{\ptxt{open}}\subseteq\R^{k}$, $p\in{}V^{\ptxt{rel. open}}\subseteq{}M$, and\n
$\alpha\in{}C^{r}(U,V)$ homeomorphic with $\rank{}D\alpha(\vec{x})=k$ for all $\vec{x}\in{}U$ ($\alpha$ is a coordinate patch).}

\ex{$M=\set{(x,y):x^{2}=y^{3}}$. Let $\begin{array}{rcl}\alpha:\R & \to & M\\ t & \mapsto & (t^{3},t^{2})\end{array}$ homeomorphic, but $\alpha'(0)=(0,0)$. So this fails! We will show $M$ is not a manifold-without-boundary.}

\ex{$M=\set{(x,y):y=\abs{x}}$. Let $\begin{array}{rcl}\alpha:\R & \to & M\\ t & \mapsto & (t^{3},t^{6})\end{array}$ homeomorphic, but $\alpha'(0)=(0,0)$. So this fails! We will show this $M$ is also not a manifold-without-boundary.}

\ex{$M=\set{(x,y):y=x^{2}}$. Let $\begin{array}{rcl}\alpha:\R & \to & M\\ t & \mapsto & (t^{3},t^{6})\end{array}$ homeomorphic, but $\alpha'(0)=(0,0)$. So this fails.\n
But let $\begin{array}{rcl}\beta:\R & \to & M\\ t & \mapsto & (t,t^{2})\end{array}$ homeomorphic, and $\alpha'(0)=(1,0)$. So it is a manifold-without-boundary.}

\ex{Interval to a figure 8 symbol (really only makes sense with the picture).}

\ex{$M=\set{(x,y):x^{2}+y^{2}=1}$. We can't handle this with one coordinate patch, or else $\exists\alpha\inv:M\to{}U^{\ptxt{osso}\R}$ continuous. $M$ is compact, so $\alpha\inv$ can't exist. But we can use $2$ trig parameters.}

\thm{For $M\subseteq\R^{n}$, the following are equivalent:
\begin{enumerate}[label=(\arabic*)]
	\item $M$ is a $k$-mfd-wob
	\item $\forall{}p\in{}M$, $\exists{}V^{\ptxt{rel. open}}\subset{}M$ with $V\ni{}p$ and $\exists{}g\in{}C^{r}(\ptxt{osso}\R^{k},\R^{n-k})$, $\exists\ptxt{Perm}:\R^{n}\to\R^{n}$ coord permutation, such that $v=\ptxt{Perm}(\Graph(g))$.
\end{enumerate}\up\n
Proof (2)$\Rightarrow$(1): Take $\alpha:\ptxt{dom}(g)\to{}V$, $\vec{x}\mapsto\ptxt{Perm}(\vec{x},g(\vec{x}))$. Then $\alpha\inv=(\ptxt{proj onto }\R^{k})\of\ptxt{Perm}\inv$.\n
Proof (1)$\Rightarrow$(2): For $1<s_{1}<\cdots<s_{k}\le{}n$, let $S=\set{s_{1},\ldots,s_{k}}$. Define $P_{S}:\R^{n}\to\R^{k}$ by $\vec{x}\mapsto(x_{s_{1}},\ldots,x_{s_{k}})$. Then $D(P_{S}\of\alpha)(q)=DP_{S}(p)\cdot{}D\alpha(a)=P_{S}\cdot{}D\alpha(p)$. $D\alpha(p)$ has $n$ rows, $k$ columns, and is of rank $k$, so by the rank condition, choose $S$ such that $D(P_{S}\of\alpha)(g)$ invertible.\n
By the inverse function theorem, we can convert $U,V,\Omega$ to $\tilde{U},\tilde{V},\tilde{\Omega}$ \st{} $P_{S}\of\alpha$ is differentiable from $\tilde{U}\to\tilde{\Omega}$.\n
\n
Let $h=\alpha\of(P_{S}\of\alpha)\inv$. Then $P_{S}\of{}h=P_{S}\of\alpha\of(P_{S}\of\alpha)\inv=\Id$. Choose a permutation such that $\ptxt{Perm}(\set{1,\ldots,k})=S$. Then $\tilde{V}=\ptxt{Perm}(\Graph(P_{S}\of{}h))$.\n
\n
Fact 1: $\alpha\inv:\tilde{V}\to\tilde{U}$ extends $(P_{S}\of\alpha)\inv\of{}P_{S}\in{}C^{r}(\ptxt{Perm}(\tilde{\Omega}\times\R^{n-k}),\tilde{U})$.\n
Fact 2: Because of fact 1, $\alpha_{2}\inv\of\alpha_{1}$ (where $\alpha_{2}$ maps from $U_{2}$ to $M$) is $C^{r}$ where defined. So $\alpha_{2}\inv\of\alpha_{1}$ is a $C^{r}$ diffeomorphism. Hence, $D(\alpha_{2}\inv\of\alpha_{1})$ is invertible.\n
Thus, $p\in{}M$, and $M$ is a $k$-mfd-wob.\proven}

\par\noindent After ``\B{}IG ZOOM'', $M$ looks more like an affine set $p+\tau_{p}(M)$ (specifically, the tangent space to $M$ at $p$).\n

\defn{$\tau_{p}(M)\eqdef{}D\alpha_{1}(q_{1})[\R^{k}]$ is the \underline{tangent space} to $M$ at $p$.}

\par\noindent $D\alpha_{2}(q_{2}[\R^{k}])\overset{(2)}{=}D\alpha_{2}(q_{2})\cdot{}D(\alpha_{2}\inv\of\alpha_{1})(q_{1})[\R^{k}]=D\alpha_{1}(q_{1})[\R^{k}]$

\end{document}